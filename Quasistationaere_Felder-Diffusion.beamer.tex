\input{head.inc}

% Präambelbefehle für die Präsentation
\title[TET: Quasistationäre Felder II - Felddiffusion]{Quasistationäre Felder II - Felddiffusion}

\begin{document}
% 
% Frontmatter 
% 
%%%%%%%%%%%%%%%%%%%%%%%%%%%%%%%%%%%%%%%%%%%%%%%%%%%%%%%%%%%%%%%%%%%%%%%%%%%%%%%%%%%%%%%%%%%%%%%%%%%%%%%%%%%%%%%%%%%%%%%%%%%%% 

%% inserts the title page and the table of contents
\maketitle

% 
% Content 
% 
%%%%%%%%%%%%%%%%%%%%%%%%%%%%%%%%%%%%%%%%%%%%%%%%%%%%%%%%%%%%%%%%%%%%%%%%%%%%%%%%%%%%%%%%%%%%%%%%%%%%%%%%%%%%%%%%%%%%%%%%%%%%% 
\section{Quasistationäre Felder II - Felddiffusion}

\begin{frame}
  \frametitle{MQS - Diffusionssgleichung für $\efeld[v]$}
  \begin{itemize}[<+->]
   \item Magneto-Quasistatik (MQS): Maxwell-Gleichungen
    \begin{align*}
	\rotation \efeld[v] &= -\frac{\partial\tetB[v]}{\partial t} & \rotation \magfeld[v] &= \elstromdichte[v]\\
	\divergenz \verschiebung[v] &= \laddichte{V} & \divergenz \tetB[v] &= 0 
\end{align*}
\item Materialgleichungen (lineares, homogenes, isotropes Medium)
\begin{align*}
	\verschiebung[v] &= \varepsilon \efeld[v] & \tetB[v] &= \mu \magfeld[v] & \elstromdichte[v] = \elstromdichte[v]_\mathrm{L} &= \kappa \efeld[v] 
\end{align*}
\item Annahme:  keine makroskopische Ladungsträgerdichte \(\laddichte{V} = 0 \) $\to$ $\divergenz \efeld[v] = 0$
\item Und damit: (Hinweis: $\laplace \vec{a} = \divergenz \gradient \vec{a} = \gradient \divergenz \vec{a}  - \rotation \rotation \vec{a}$)
  \begin{align*}
    \gradient \divergenz \efeld[v] = \vec{0} &= \laplace \efeld[v] + \rotation \rotation \efeld[v]\\
    \laplace \efeld[v] & = - \rotation \rotation \efeld[v] = \rotation \frac{\partial\tetB[v]}{\partial t} = \frac{\partial}{\partial t} \left( \rotation \tetB[v] \right)\\
                                       &= \mu \frac{\partial}{\partial t} \left( \rotation \magfeld[v] \right) = \mu \frac{\partial}{\partial t} \elstromdichte[v] = \mu\kappa \frac{\partial}{\partial t} \efeld[v]\\
    \Aboxed{\laplace \efeld[v](\ortsvektor[v], t) -\mu\kappa \frac{\partial}{\partial t} \efeld[v](\ortsvektor[v], t) &= \vec{0}} \text{ \alert{Diffusionsgleichung} für }\efeld[v] 
\end{align*}  
  \end{itemize}
\end{frame}


\begin{frame}
  \frametitle{MQS - Diffusionssgleichung für $\magvekpot[v]$}
  \begin{itemize}[<+->]
  \item Aus dem Amp{\'e}reschen Durchflutungsgesetz folgt zunächst
    $$
    \rotation \magfeld[v] = \elstromdichte[v] \to \rotation \tetB[v] = \mu \elstromdichte[v] \to \rotation\rotation \magvekpot[v] = \mu \elstromdichte[v]
    $$
  \item Mit $\elstromdichte[v] = \kappa \efeld[v]$ gilt somit
    $$
    \rotation\rotation \magvekpot[v] = \mu\kappa \efeld[v] \to \boxed{ \gradient\divergenz \magvekpot[v] -\laplace \magvekpot[v] = \mu\kappa \efeld[v]}
    $$
  \item Mit Hilfe des Induktionsgesetzes
    $$
    \rotation \efeld[v] = -\frac{\partial\tetB[v]}{\partial t}  \to \rotation \efeld[v] = -\rotation \frac{\partial \magvekpot[v]}{\partial t} \to \rotation \left( \efeld[v] + \frac{\partial \magvekpot[v]}{\partial t}\right) = \vec{0} \to \boxed{\efeld[v] + \frac{\partial\magvekpot[v]}{\partial t} = - grad \phi}  
    $$
  \item Spezialfall $\laddichte{V} = 0$: $\divergenz \efeld[v]=0$ $\to$ $\phi=\text{const.} = 0 $ (o.B.d.A) $\to$ $\boxed{\efeld[v] =- \frac{\partial \magvekpot[v]}{\partial t}}$
  \item Hiermit folgt eine \alert{Diffusionsgleichung} \boxed{\laplace \magvekpot[v] - \mu\kappa \frac{\partial }{\partial t}\magvekpot[v] =\gradient\left( \mu\kappa \phi +\divergenz \magvekpot[v] \right)  }
    \item \alert{Eichung}: $\divergenz \magvekpot[v] = -\mu\kappa \phi$ bzw. $\divergenz \magvekpot[v] = 0$ für $\laddichte{V} = 0$ (Coulomb-Eichung)   
  \end{itemize}
\end{frame}

\begin{frame}
  \frametitle{MQS - weitere Diffusionssgleichungen}
  \begin{itemize}[<+->]
    \item Auf analoge Weise findet man weitere Diffusionsgleichungen:
    \item Magnetische Induktion
      $$
      \boxed{\laplace \tetB[v] - \mu\kappa \frac{\partial }{\partial t}\tetB[v] = \vec{0} }
      $$
    \item Stromdichte
      $$
      \boxed{\laplace \elstromdichte[v] - \mu\kappa \frac{\partial }{\partial t}\elstromdichte[v] = \vec{0} }
      $$
    \item Offensichtlich erhält mal identische pDGL für $\efeld[v]$, $\tetB[v]$, $\elstromdichte[v]$ und $\magvekpot[v]$ $\to$ \alert{Diffusionsgleichungen}
    \item Unterschied: Randbedingungen sind anders!
      \item Wichtige Eigenschaft: \alert{irreversible Prozesse}, d.h. der Übergang  $t \to -t$ liefert eine andere pDGL 
    \end{itemize}
\end{frame}

\begin{frame}
  \frametitle{Exkurs: 3 Typen pDGL 2. Ordnung}
  \begin{itemize}[<+->]
  \item \alert{1. Elliptische partielle Differentialgleichung 2. Ordnung} $\laplace \alpha(\ortsvektor[v]) = \beta(\ortsvektor[v])$
    \begin{itemize}[<+->]
      \item Nur 2. Ableitungen (+,+,+)
    \item z.B. Poisson-Gleichung $\laplace\phi(\ortsvektor[v])=-\frac{\laddichte{V}(\ortsvektor[v])}{\varepsilon}$
  \item zeitunabhängige (stationäre) Probleme
  \item Lösung beschreibt häufig den Zustand \alert{minimaler Energie} $\to$ Variationsproblem, Finite Elemente Method (FEM) 
    \item \alert{Randwert}probleme; Dirichlet bzw. Neumann Randwerte
  \end{itemize}

\item \alert{2. Parabolische partielle Differentialgleichung 2. Ordnung} $\laplace \alpha(\ortsvektor[v], t) - c\frac{\partial}{\partial t}\alpha(\ortsvektor[v], t) = \beta(\ortsvektor[v], t)$
  \begin{itemize}[<+->]
    \item Bezüglich einer Variablen (Zeit) nur einfache Ableitung; 2. Ableitung: Vorzeichen (+,+,+,0)
    \item z.B. Diffusions-Gleichung $\laplace \efeld_x(\ortsvektor[v], t) -\mu\kappa \frac{\partial}{\partial t} \efeld_x(\ortsvektor[v], t) = 0$
  \item kein stationäres Probleme
    \item \alert{Anfangs-Randwert}probleme; Dirichlet bzw. Neumann Randwerte und zusätzlich Anfangswert z.B. $\alpha(\ortsvektor[v], t=0)$ vorgeben.
  \end{itemize}

\item \alert{3. Hyperbolische partielle Differentialgleichung 2. Ordnung} $\laplace \alpha(\ortsvektor[v], t) - c^2\frac{\partial^2}{\partial t^2}\alpha(\ortsvektor[v], t) = \beta(\ortsvektor[v], t)$
  \begin{itemize}[<+->]
    \item 2. Ableitung: Vorzeichen (+,+,+,-)
    \item z.B. Wellen-Gleichung $\to$ \alert{kommt noch}
    \item \alert{Anfangs-Randwert}probleme $\to$ \alert{zusätzlich:} zeitliche Ableitung bei $t=0$
  \end{itemize}
  
\end{itemize}
\end{frame}

\input{finalframe.inc}
   
\end{document}