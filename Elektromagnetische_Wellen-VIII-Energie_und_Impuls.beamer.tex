\input{head.inc}
 \usetikzlibrary{overlay-beamer-styles}
  
% Präambelbefehle für die Präsentation
\title[TET: Elektromagnetische Wellen VIII - Energie und Impuls]{Elektromagnetische Wellen VIII - Energie und Impuls}

\begin{document}
% 
% Frontmatter 
% 
%%%%%%%%%%%%%%%%%%%%%%%%%%%%%%%%%%%%%%%%%%%%%%%%%%%%%%%%%%%%%%%%%%%%%%%%%%%%%%%%%%%%%%%%%%%%%%%%%%%%%%%%%%%%%%%%%%%%%%%%%%%%% 

%% inserts the title page and the table of contents
\maketitle

% 
% Content 
% 
%%%%%%%%%%%%%%%%%%%%%%%%%%%%%%%%%%%%%%%%%%%%%%%%%%%%%%%%%%%%%%%%%%%%%%%%%%%%%%%%%%%%%%%%%%%%%%%%%%%%%%%%%%%%%%%%%%%%%%%%%%%%% 
\section{Elektromagnetische Wellen VIII - Energie und Impuls}

\begin{frame}
  \frametitle{Energiedichte}
  \begin{itemize}[<+->]
  \item Wir betrachten ebene Wellen
    \begin{align*}
      \EFeld[uv](\Ortsr[v], t) & =\EFeld[uv]_0\euler^{\komplex(\omega t - \Wellenzahl[v]\cdot\Ortsr[v])} \quad ;\quad \BFeld[uv](\Ortsr[v], t)  =\BFeld[uv]_0\euler^{\komplex(\omega t - \Wellenzahl[v]\cdot\Ortsr[v])}  \\
      \text{mit } \BFeld[uv]_0 &= \frac{1}{\omega} \left(\Wellenzahl[v] \times \EFeld[uv]_0\right) \ergo |\BFeld[uv]_0|^2 = \frac{1}{\Geschwindigkeit_c^2} |\EFeld[uv]_0|^2 \quad \Geschwindigkeit_c = \frac{1}{\sqrt{\varepsilon\mu}} = \frac{\omega}{\Wellenzahl} =\Geschwindigkeit_p
    \end{align*}
  \item Beliebig veränderliche Felder - Energieerhaltung: \alert{Komplexen Poyntingschen Satz} bei harmonischer Zeitabhängigkeit.
  \item Energiedichte (Mittelwerte über eine Periodendauer):
    \begin{equation*}
      \langle \energiedichte_{em}(\Ortsr[v], t)\rangle = \langle \energiedichte_{e}(\Ortsr[v], t)\rangle + \langle \energiedichte_{m}(\Ortsr[v], t)\rangle = \frac{1}{4}\real{\HFeld[uv]_0 \cdot \BFeld[uv]_0^\star + \EFeld[uv]_0 \cdot \DFeld[uv]_0^\star}
    \end{equation*}
    \item Elektrischer Anteil von \(\langle \energiedichte_{em}\rangle\): \( \langle \energiedichte_{e}\rangle  = \frac{1}{4}\real{\EFeld[uv]_0 \cdot \DFeld[uv]_0^\star} = \frac{1}{4} \varepsilon |\EFeld[uv]_0|^2\) 
    \item Magnetischer Anteil von \(\langle \energiedichte_{em}\rangle\): \( \langle \energiedichte_{m}\rangle  = \frac{1}{4}\real{\HFeld[uv]_0 \cdot \BFeld[uv]_0^\star} = \frac{1}{4} \frac{1}{\mu} |\BFeld[uv]_0|^2 = \frac{1}{4} \varepsilon |\EFeld[uv]_0|^2 \)
    \item Elektrischer und magnetischer Anteil der Welle enthalten gleiche Engieanteile. Die gesamte Energiedichte ist
      \begin{equation*}
        \boxed{\langle \energiedichte_{em}(\Ortsr[v], t)\rangle = \frac{1}{2} \varepsilon |\EFeld[uv]_0|^2 = \frac{1}{2} \frac{1}{\mu} |\BFeld[uv]_0|^2}
      \end{equation*}
    \end{itemize}
  \end{frame}


\begin{frame}
  \frametitle{Energieflussdichte - Poyntingscher Vektor}
  \begin{itemize}[<+->]      
      \item Energieflussdichte, Poynting-Vektor (Mittelwerte über eine Periodendauer): 
    \begin{align*}
      \langle \PoyntingVektor[v](\Ortsr[v], t)\rangle &= \frac{1}{2}\real{\EFeld[uv]_0 \times \HFeld[uv]_0^\star} = \frac{1}{2} \frac{1}{\mu} \frac{1}{\omega} \real{\EFeld[uv]_0 \times (\Wellenzahl[v] \times \EFeld[uv]_0^\star)}\\
                                                   &= \frac{1}{2\omega\mu} \real{ \Wellenzahl[v] |\EFeld[uv]_0|^2 - \EFeld[uv]_0^\star \underbrace{(\EFeld[uv]_0 \cdot \Wellenzahl[v])}_{=0} } \; , \; \Geschwindigkeit_p=\Geschwindigkeit_c = \frac{\omega}{\Wellenzahl} = \frac{1}{\sqrt{\varepsilon\mu}} \ergo \frac{1}{\omega} = \frac{\sqrt{\varepsilon\mu}}{\Wellenzahl}\\
      &= \frac{1}{2} \sqrt{\frac{\varepsilon}{\mu}} |\EFeld[uv]_0|^2 \frac{\Wellenzahl[v]}{\Wellenzahl} = \frac{1}{2} \sqrt{\frac{\varepsilon}{\mu}} |\EFeld[uv]_0|^2 \einheitsvek{k} = \underbrace{\Geschwindigkeit_p}_{=\Geschwindigkeit_c=\Geschwindigkeit_g} \langle \energiedichte_{em}\rangle\; \einheitsvek{k} = \frac{1}{2} \frac{1}{Z} |\EFeld[uv]_0|^2 \einheitsvek{k} 
    \end{align*}
      \item Energiefluss in Ausbreitungsrichtung der Welle. 
    
    \end{itemize}
  \end{frame}
  
\begin{frame}
  \frametitle{Impulsdichte}
  \begin{itemize}[<+->]      
  \item Beliebig veränderliche Felder - Impulserhaltung: Die elektromagnetische Impulsdichte ist gegeben durch:
    \begin{equation*}
      \impulsdichte^{em} = \varepsilon\mu\PoyntingVektor[v] = \frac{1}{\Geschwindigkeit_c^2} \Geschwindigkeit_p \langle \energiedichte_{em}\rangle\; \einheitsvek{k} = \frac{1}{\Geschwindigkeit_c} \langle \energiedichte_{em}\rangle\; \einheitsvek{k}
      \end{equation*}
  \item Im Rahmen der \alert{Speziellen Relativitätstheorie} (SRT) wird folgende \alert{allgemeine Energie-Impulsgleichung} für das Vakuum (\(\Geschwindigkeit_c = \lichtgeschw \)) gefunden
    \begin{equation*}
      \boxed{\energie^2 - \lichtgeschw^2p^2 = m^2\lichtgeschw^4}\quad; \quad \text{ruhende Masse: } \energie = m\lichtgeschw^2   
    \end{equation*}
  \item Für ein \alert{masseloses Teilchen} ergibt sich genau die Beziehung der elektromagnetischen (ebenen) Welle!
    \begin{equation*}
      \boxed{\energie =   \lichtgeschw p }   
    \end{equation*}
    \item \ergo \alert{Welle-Teilchen-Dualismus} \ergo \alert{Photon} \ergo \alert{Photoelektrischer-Effekt} \ergo \alert{Quantenmechanik} (nicht SRT konform) \ergo \alert{relativistische Quantenfelttheorie, Quantenelektrodynamik (QED)}
    \end{itemize}
    
  \end{frame}
  
\input{finalframe.inc}
   
\end{document}