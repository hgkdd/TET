\input{head.inc}
  
% Präambelbefehle für die Präsentation
\title[TET: Elektromagnetische Wellen XI - Erzeugung]{Elektromagnetische Wellen XI - Erzeugung}

\begin{document}
% 
% Frontmatter 
% 
%%%%%%%%%%%%%%%%%%%%%%%%%%%%%%%%%%%%%%%%%%%%%%%%%%%%%%%%%%%%%%%%%%%%%%%%%%%%%%%%%%%%%%%%%%%%%%%%%%%%%%%%%%%%%%%%%%%%%%%%%%%%% 

%% inserts the title page and the table of contents
\maketitle

% 
% Content 
% 
%%%%%%%%%%%%%%%%%%%%%%%%%%%%%%%%%%%%%%%%%%%%%%%%%%%%%%%%%%%%%%%%%%%%%%%%%%%%%%%%%%%%%%%%%%%%%%%%%%%%%%%%%%%%%%%%%%%%%%%%%%%%% 
\section{Elektromagnetische Wellen XI - Erzeugung}

\begin{frame}
  \frametitle{Ausgangspunkt}
  \begin{itemize}[<+->]
  \item Im Abschnitt \alert{Elektromagnetische Wellen I - Grundlagen} hatten wir bereits die Entkopplung der Maxwell-Gleichungen in ihrer allgemeinen Form betrachtet.
  \item Für \alert{Skalarpotential} \(\SkalarPot\) und \alert{Vektorpotential} \(\VektorPot[v]\) mit
    \begin{equation*}
      \BFeld[v]=\rotation\VektorPot[v] \quad\text{und}\quad \EFeld[v]=-\gradient\SkalarPot -\frac{\d \VektorPot[v]}{\d t}
    \end{equation*}
    ergeben sich:
    \begin{itemize}[<+->]
    \item In \alert{Coulomb-Eichung} (Strahlungseichung) (\(\divergenz \VektorPot[v] =0\))
      \begin{align*}
        \laplace\SkalarPot &= -\frac{\laddichte{V}}{\varepsilon} \\
        \laplace\VektorPot[v]-\varepsilon\mu\frac{\d^2 \VektorPot[v]}{\d t^2}-\gradient \varepsilon\mu\frac{\d \SkalarPot}{\d t} &= -\mu \StromDichte[v]
      \end{align*}
    \item In \alert{Lorenz-Eichung} (\(\divergenz\VektorPot[v] +\varepsilon\mu \frac{\d \SkalarPot}{\d t}=0\))
      \begin{align*}
\laplace\SkalarPot-\varepsilon\mu \frac{\d^2}{\d t^2}\SkalarPot &= -\frac{\laddichte{V}}{\varepsilon}\\
\laplace\VektorPot[v]-\varepsilon\mu \frac{\d^2}{\d t^2}\VektorPot[v] &= -\mu \StromDichte[v]
\end{align*}

      \end{itemize}
\end{itemize}
\end{frame}

\begin{frame}
  \frametitle{Lösung für die Potentiale in Lorenzeichung}
  \begin{itemize}[<+->]
  \item Basierend auf den Ableitungen in \alert{Elektromagnetische Wellen VII - Allgemeine Lösung} (Kirchhoffsche-Lösung, Huygens-Fresnelsches-Prinzip) kann die \alert{Greensche-Funktion} (\(\delta\)-Anregung am Ort \(\Ortsr[vs]\) zur Zeit \(t'\)) direkt angegeben werden (auf die formale Herleitung wird hier verzichtet):
    \begin{equation*}
      G(\Ortsr[v]-\Ortsr[vs], t - t')=G_{ret}(\Ortsr[v]-\Ortsr[vs], t - t') = \frac{1}{4\pi |\Ortsr[v]-\Ortsr[vs]|}\delta\left(\frac{|\Ortsr[v]-\Ortsr[vs]|}{\Geschwindigkeit_c}-(t-t')\right) = \frac{\delta(t'-t_{ret})}{4\pi |\Ortsr[v]-\Ortsr[vs]|}
    \end{equation*}
  \item Hierbei ist \(t_{ret} = t - \frac{|\Ortsr[v]-\Ortsr[vs]|}{\Geschwindigkeit_c}\) die \alert{retardierte Zeit} und \(G_{ret}\) die \alert{retardierte Greensche Funktion}.
  \item Die Lösung der inhomogenen Wellengleichung in Lorenz-Eichung sind dann die \alert{retardierten Potentiale}
    \begin{align*}
      \Aboxed{\SkalarPot (\Ortsr[v], t) &= \frac{1}{4\pi\varepsilon} \iiint_V \frac{\laddichte{V}\left(\Ortsr[vs], t-\frac{|\Ortsr[v]-\Ortsr[vs]|}{\Geschwindigkeit_c}\right)}{|\Ortsr[v]-\Ortsr[vs]|} \upd^3\Ortsr[s]}\\
      \Aboxed{\VektorPot[v] (\Ortsr[v], t) &= \frac{\mu}{4\pi} \iiint_V \frac{\StromDichte[v]\left(\Ortsr[vs], t-\frac{|\Ortsr[v]-\Ortsr[vs]|}{\Geschwindigkeit_c}\right)}{|\Ortsr[v]-\Ortsr[vs]|} \upd^3\Ortsr[s]}
      \end{align*}
\end{itemize}
\end{frame}

\begin{frame}
  \frametitle{Bemerkungen zur retardierten Lösung}
  \begin{itemize}[<+->]
  \item Retardierte Potentiale: Lösung wegen Retardierung in der Regel schwierig
    \begin{align*}
      \SkalarPot (\Ortsr[v], t) &= \frac{1}{4\pi\varepsilon} \iiint_V \frac{\laddichte{V}\left(\Ortsr[vs], t-\frac{|\Ortsr[v]-\Ortsr[vs]|}{\Geschwindigkeit_c}\right)}{|\Ortsr[v]-\Ortsr[vs]|} \upd^3\Ortsr[s] &\VektorPot[v] (\Ortsr[v], t) &= \frac{\mu}{4\pi} \iiint_V \frac{\StromDichte[v]\left(\Ortsr[vs], t-\frac{|\Ortsr[v]-\Ortsr[vs]|}{\Geschwindigkeit_c}\right)}{|\Ortsr[v]-\Ortsr[vs]|} \upd^3\Ortsr[s]
      \end{align*}

\begin{tikzpicture}[line width = 1.2pt, line join=round,>=stealth]
	% Differenzvektor R
	\draw [->, color=red!60] (1,1.8) -- (4.5,2.5);
	\draw [color=red!60] (3.8,2.25) node[anchor=south east] {$\Ortsr[v] - \Ortsr[vs] $};
	% Aufpunkt
	\draw [->,color=blue!70] (0,0) -- (4.5,2.5) node[anchor=west] {$\Ortsr[v], t$};
	\filldraw [color=blue!70] (4.5,2.5) circle (1pt);
	% Ladungsdichte
	\coordinate (a) at (2.1,1.8);
	\coordinate (b) at (2,1.2);
	\coordinate (c) at (1.8,0.4);
	\coordinate (d) at (1.3,0.6);
	\coordinate (e) at (0.7,0.5);
	\coordinate (f) at (0.5,2);
	\coordinate (g) at (1.2,2.5);
	\shade[ball color=white!10!green!20,opacity=0.20] plot [smooth cycle, tension = 1] coordinates {(a) (b) (c) (d) (e) (f) (g)};
	\draw [color=darkgreen] plot [smooth cycle, tension = 1] coordinates {(a) (b) (c) (d) (e) (f) (g)} node [sloped, above] {\ $ \laddichte{V}, \StromDichte[v] $};
	\draw [->,color=darkgreen] (0,0) -- (1,1.8);
	\draw [color=darkgreen] (0.5,1.0) node[anchor=north east] {$ \Ortsr[vs] $};
      \end{tikzpicture}
    \item Die Zeit \(\frac{|\Ortsr[v]-\Ortsr[vs]|}{\Geschwindigkeit_c}\) ist gerade die \alert{Laufzeit} vom Quellpunkt zum Beobachtungspunkt
  \item Es gibt formal auch \alert{avancierte} Lösungen \(\to\) nicht kausal
    \item Unterschied zur quasistationären Betrachtung: dort wurde Retardierung vernachlässigt.
\end{itemize}
\end{frame}

\begin{frame}
  \frametitle{Räumlich beschränktes Quellgebiet und harm. ZA}
  \begin{itemize}[<+->]
  \item Die Quellen \(\laddichte{V}, \StromDichte[v]\) seien  von Null verschieden nur innerhalb einer \alert{Kugel mit Durchmesser \(d\)} \(\to\) Es genügt die Betrachtung des Vektorpotentials (\(\BFeld[v]=\rotation\VektorPot[v];\; \frac{\partial \DFeld[v]}{\d t}=\rotation\HFeld[v]\)).
  \item Wir betrachten \alert{harmonische Zeitabhängigkeit}
    \begin{align*}
      \StromDichte[v] (\Ortsr[vs], t) &= \real{\StromDichte[uv](\Ortsr[vs])\euler^{\komplex \omega t}} \\
      \StromDichte[v] (\Ortsr[vs], t-\frac{|\Ortsr[v]-\Ortsr[vs]|}{\Geschwindigkeit_c}) &= \real{\StromDichte[uv](\Ortsr[vs])\euler^{\komplex \omega t}\euler^{-\komplex \omega \frac{|\Ortsr[v]-\Ortsr[vs]|}{\Geschwindigkeit_c}}} = \real{\StromDichte[uv](\Ortsr[vs])\euler^{\komplex \omega t}\euler^{-\komplex \Wellenzahl |\Ortsr[v]-\Ortsr[vs]|}}  
    \end{align*}
  \item Damit gilt für das komplexe Vektorpotential
    \begin{equation*}
      \boxed{%
      \VektorPot[uv] (\Ortsr[v]) = \frac{\mu}{4\pi} \iiint_V \frac{\euler^{-\komplex \Wellenzahl |\Ortsr[v]-\Ortsr[vs]|}}{|\Ortsr[v]-\Ortsr[vs]|} \StromDichte[uv](\Ortsr[vs]) \upd^3 \Ortsr[s]} 
      \end{equation*}
  \item Hierbei ist
    \(\tilde{G}_{ret}(\Ortsr[v]-\Ortsr[vs], \Wellenzahl) = \frac{1}{4\pi} \frac{\euler^{-\komplex \Wellenzahl |\Ortsr[v]-\Ortsr[vs]|}}{|\Ortsr[v]-\Ortsr[vs]|}\)
    die retardierte Greensche-Funktion im Bildraum mit \(\Wellenzahl = \frac{\omega}{\Geschwindigkeit_c} = \omega\sqrt{\varepsilon\mu}\).
    \end{itemize}
\end{frame}

\begin{frame}
  \frametitle{Strahlungszonen}
  \begin{columns}
    \begin{column}{.4\textwidth}
\begin{tikzpicture}[line width = 1.2pt, line join=round,>=stealth]
	% Differenzvektor R
	\draw [->, color=red!60] (1.4,1.8) -- (4.5,2.5);
	\draw [color=red!60] (3.8,2.4) node[anchor=south east, rotate=12] {$\Ortsr[v] - \Ortsr[vs] $};
	% Aufpunkt
	\draw [->,color=blue!70] (1,1) -- (4.5,2.5) node[anchor=west] {$\Ortsr[v], t$};
	\filldraw [color=blue!70] (4.5,2.5) circle (1pt);
	% Ladungsdichte
	\coordinate (a) at (2.1,1.8);
	\coordinate (b) at (2,1.2);
	\coordinate (c) at (1.8,0.4);
	\coordinate (d) at (1.3,0.6);
	\coordinate (e) at (0.7,0.5);
	\coordinate (f) at (0.5,2);
	\coordinate (g) at (1.2,2.5);
	\shade[ball color=white!10!green!20,opacity=0.20] plot [smooth cycle, tension = 1] coordinates {(a) (b) (c) (d) (e) (f) (g)};
	\draw [color=darkgreen] plot [smooth cycle, tension = 1] coordinates {(a) (b) (c) (d) (e) (f) (g)} node [sloped, below] {\ $ \laddichte{V}, \StromDichte[v] $};
	\draw [->,color=darkgreen] (1,1) -- (1.4,1.8);
	\draw [color=darkgreen] (1.2,1.8) node[anchor=north east] {$ \Ortsr[vs] $};
        \draw [dotted, thin] (1.4,1.8) -- +(-65:0.6) node[below,rotate=25,xshift=-5]{$\Ortsr[vs]\cdot\einheitsvek{r}$};
        \draw[dashed,thin] (1.25,1.4) circle (1.2cm);
        \draw[thin] (0.05,1.7) -- (0.05,0); 
        \draw[thin] (2.45,1.7) -- (2.45,0);
        \draw[thin, <->] (0.05, 0.1) -- (2.45, 0.1) node[midway, below]{$d$};  
      \end{tikzpicture}
      \end{column}
    \begin{column}{.6\textwidth}
  \begin{itemize}[<+->]
  \item Näherung für \(|\Ortsr[v]| \gg |\Ortsr[vs]|\) (mit \(\left.\frac{1}{1-x}\right|_{x \ll 1}\simeq 1+x\)):
    \begin{align*}
      |\Ortsr[v] - \Ortsr[vs]| &\simeq |\Ortsr[v]| - \Ortsr[vs]\cdot\einheitsvek{r} = \Ortsr \left(1 - \frac{\einheitsvek{r}\cdot \Ortsr[vs]}{\Ortsr} \right)\\
      \Rightarrow \frac{1}{|\Ortsr[v] - \Ortsr[vs]|} &\simeq \frac{1}{\Ortsr} \left( 1 +\frac{\einheitsvek{r}\cdot \Ortsr[vs]}{\Ortsr} \right)
      \end{align*}
  \end{itemize}
  \end{column}
  \end{columns}
  \begin{itemize}[<+->]
  \item \alert{Nahzone}: \(d \ll \Ortsr \ll \lambda\) (statische Zone)
    \begin{equation*}
      \euler^{-\komplex \Wellenzahl |\Ortsr[v]-\Ortsr[vs]|} \simeq \euler^{-\komplex \frac{2\pi}{\lambda}\Ortsr \left(1 - \frac{\einheitsvek{r}\cdot \Ortsr[vs]}{\Ortsr} \right) } \simeq \euler^{-\komplex 0} =1 \quad \to \text{ keine Retardierung}
    \end{equation*}
  \item \alert{Fernzone}: \(d \ll \Ortsr\) und \(\lambda \ll \Ortsr\) (Strahlungszone)
    \begin{equation*}
      \euler^{-\komplex \Wellenzahl |\Ortsr[v]-\Ortsr[vs]|} \simeq \euler^{-\komplex \Wellenzahl \Ortsr} \quad\to \text{gleiche Retardierung für alle }\Ortsr[vs] \to \text{ vor das Integral}
    \end{equation*}
  \item \alert{Übergangszone}: \(d \simeq \Ortsr\) und/oder \(\Ortsr\simeq\lambda\)

    Dieser Fall ist schwieriger und muss jeweils konkret analysiert werden.
  \end{itemize}
\end{frame}


\input{finalframe.inc}
   
\end{document}