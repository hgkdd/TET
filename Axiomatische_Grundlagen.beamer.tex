\input{head.inc}
\usepackage{tabu}
\usepackage[absolute,overlay]{textpos}

% Präambelbefehle für die Präsentation
\title[TET: Axiomatische Grundlagen]{Axiomatische Grundlagen}

\begin{document}
\maketitle

\section{Axiomatische Grundlagen}

\begin{frame}
  \frametitle{Maxwellsche Gleichungen}
{\tabulinesep=1.2mm
  \begin{tabu}{r|rl}
    \(\rotation\EFeld[v] + \dfrac{\partial \BFeld[v]}{\partial t} = \vec{0} \) & \(\rotation \HFeld[v] -\dfrac{\partial \DFeld[v]}{\partial t} = \StromDichte[v] \) & vektorielle pDGL, zeitabhängig \(\to\) dynamisch \\
\hline
    \( \divergenz\BFeld[v] = 0 \) & \( \divergenz \DFeld[v] =\laddichte{V}\) & skalare pDGL, zeitunabhängig \(\to\) statisch\\
    homogene pDGL & inhomogene pDGL & 
    \end{tabu}}
  
\begin{itemize}[<+->]
\item Historisch wurden die Gleichungen \textbf{unabhängig voneinander} aus \textbf{Beobachtungen} gewonnen (\textbf{Colomb-Gauß}sches-Gesetz, \textbf{Faraday}sches Induktionsgesetz, erweitertes \textbf{Ampère}sches Durchflutungsgesetz, Erhaltung des magnetischen Flusses) 
\item Einführung des Verschiebungsstroms und Zusammenführung durch \textbf{James Clerk Maxwell} (1831-1879): {\tiny Maxwell, James Clerk, and Royal Society (Great Britain). A dynamical theory of the electromagnetic field. The Royal Society, 1865, doi: \url{https://doi.org/10.5479/sil.423156.39088007130693}}
\item Heutige Formulierung durch Oliver Heaviside (1850-1925) nach Einführung der Vektoranalysis (div, rot)
\item Aber: Wo kommen die Gleichungen her? Was ist das tiefere Fundament? Gibt es eine axiomatische Basis?
\end{itemize}
\end{frame}

\begin{frame}
  \frametitle{Literatur}
\begin{itemize}[<+->]
\item Die hier dargestellte Vorgehensweise ist entnommen aus:
  \bigskip
  
  Gronwald F., Hehl F.W., Nitsch J.: \enquote{Axiomatics of classical electrodynamics and ist relation to gauge field theory}, Physics Notes 14, edited by C. Baum, 2005, \url{http://ece-research.unm.edu/summa/notes/Physics/0014.pdf} (heruntergeladen am 19. November 2020)
\end{itemize}
\end{frame}


\begin{frame}
  \frametitle{Ladungsdichte und Ladung}
\begin{itemize}[<+->]
\item Ladung erscheint im Colomb-Gauß-Gesetz in Form der Volumenladungsdichte \(\laddichte{V} \)
\item Ladung ist dann nach Definition das Volumenintegral über die Ladungsdichte; \(\laddichte{V} \) ist somit Integrand eines Volumenintegrals:
  \begin{equation*}
    Q = \iiint_V \laddichte{V} \upd V
  \end{equation*}
\item<0> \(\laddichte{V} \) ist somit Ingetrand eines Volumenintegrals
  \item<0> Mit Poincaré-Lemma(3) folgt unmittelbar das Coulomb-Gauß-Gesetz: \( \boxed{\divergenz\DFeld[v] = \laddichte{V} }\)
  \item<0> Ladungen bewegen sich mit (Material)-Geschwindigkeit \(\vec{u} \to\) Stromdichte \(\StromDichte[v]\)
    \begin{equation*}
\StromDichte[v](\Ortsr[v]) = \laddichte{V}(\Ortsr[v])\vec{u}(\Ortsr[v])
\end{equation*}
\item<0> Strom \(I\) durch 2D-Fläche \(S\):
  \begin{equation*}
    I = \iint_S \StromDichte[v] \cdot \upd\vec{A}
    \end{equation*}

    
\end{itemize}
\end{frame}

\begin{frame}
  \frametitle{Poincaré-Lemma}
\begin{itemize}[<+->]
\item Das Poincaré-Lemma sagt uns, unter welchen \textbf{Bedingungen} eine Größe als \textbf{Ableitung einer anderen Größe} dargestellt (Potential; nicht Eindeutig \(\to\) führt auf den Begriff der Eichung) werden kann.
\item Im Spezialfall 3D-Raum
  \begin{enumerate}[<+->]
    \item Auf einfach zusammenhängenden Gebiet wirbelfreies Vektorfeld = Gradient eines Potentialfeldes
      \begin{equation*}
        \rotation \vec{\alpha} =\vec{0} \to \vec{\alpha} = \gradient f
      \end{equation*}
      \item auf konvexem Gebiet quellenfreies Vektorfeld = Rotation eines Vektorpotentials
      \begin{equation*}
        \divergenz \vec{\beta} =0 \to \vec{\beta} = \rotation \vec{\alpha}
      \end{equation*}
      \item Skalare Felddichte = Divergenz eines Vektorfeldes
      \begin{equation*}
        \gamma \text{ ist ein Volumen Integrand} \to \gamma = \divergenz \vec{\beta}
      \end{equation*}

 \end{enumerate}
\end{itemize}
\end{frame}


\begin{frame}
  \frametitle{Ladungsdichte und Ladung (fortgesetzt)}
\begin{itemize}[<+->]
\item<1-> Ladung erscheint im Colomb-Gauß-Gesetz in Form der Volumenladungsdichte \(\laddichte{V} \)
\item<1-> Ladung ist dann nach Definition das Volumenintegral über die Ladungsdichte; \(\laddichte{V} \) ist somit Integrand eines Volumenintegrals:
  \begin{equation*}
    Q = \iiint_V \laddichte{V} \upd V
  \end{equation*}
\item \(\laddichte{V} \) ist somit Ingetrand eines Volumenintegrals
  \item Mit Poincaré-Lemma(3) folgt unmittelbar das Coulomb-Gauß-Gesetz: \( \boxed{\divergenz\DFeld[v] = \laddichte{V} }\)
  \item Ladungen bewegen sich mit (Material)-Geschwindigkeit \(\vec{u} \to\) Stromdichte \(\StromDichte[v]\)
    \begin{equation*}
\StromDichte[v](\Ortsr[v]) = \laddichte{V}(\Ortsr[v])\vec{u}(\Ortsr[v])
\end{equation*}
\item Strom \(I\) durch 2D-Fläche \(S\):
  \begin{equation*}
    I = \iint_S \StromDichte[v] \cdot \upd\vec{A}
    \end{equation*}
\end{itemize}
\end{frame}


\begin{frame}
  \frametitle{Axiom 1: Ladungserhaltung}
\begin{itemize}[<+->]
\item Wir postulieren Ladungserhaltung: Wenn sich Ladung in einem Volumen ändert, dann geschieht dies nur durch Ladungsstrom durch die Oberfläche des Volumens 
\item Wir betrachten die materielle Ableitung (Beobachter bewegt sich mit) $\frac{\mathrm{D}}{\mathrm{D} t}$ und nutzen das Reynolds-Transport-Theorem (*):
  \begin{align*}
\frac{\mathrm{D} Q}{\mathrm{D} t} & = \frac{\mathrm{D}}{\mathrm{D} t}\left[ \int_{V(t)} \laddichte{V} \upd V \right] \stackrel{!}{=} 0\\
 &\stackrel{*}{=} \int_{V(t)} \frac{\partial \laddichte{V}}{\partial t} \upd V + \oint_{O(V)(t)}\laddichte{V} \vec{u}\cdot \upd\vec{A} \qquad \to \frac{\partial Q}{\partial t} + \oiint_{O(V)} \StromDichte[v]\cdot d\vec{A} = 0\\
& \stackrel{\text{Gauss}}{=} \int_{V(t)} \left(\frac{\partial \laddichte{V}}{\partial t} + \divergenz \left(\laddichte{V} \vec{u} \right) \right) \upd V  \\
&= \int_{V(t)} \left(\frac{\partial \laddichte{V}}{\partial t} + \divergenz \StromDichte[v] \right) \upd V  \qquad\qquad\qquad \to \frac{\partial \laddichte{V}}{\partial t} + \divergenz \StromDichte[v] = 0 \\
&= \int_{V(t)} \divergenz \left( \frac{\partial \DFeld[v]}{\partial t} + \StromDichte[v]\right) \upd V
  \end{align*}
  \item Damit dies für beliebige (zeitlich veränderliche) Volumina gelten kann, folgt:
    \begin{equation*}
      \divergenz \left( \frac{\partial \DFeld[v]}{\partial t} + \StromDichte[v]\right) = 0
      \end{equation*}
    
    \end{itemize}
\only<4>{
\begin{textblock*}{10pt}(240pt,153pt)
{ \color{red} \framebox[22mm]{\rule{0pt}{2em}}\newline Kontinuitätsgleichung }
\end{textblock*}}

\end{frame}

\begin{frame}
  \frametitle{Axiom 1: Ladungserhaltung (...)}
\begin{itemize}[<+->]
\item Aus Pointcaré (2):
  \begin{align*}
    \divergenz \left( \frac{\partial \DFeld[v]}{\partial t} + \StromDichte[v]\right) = 0 \to \frac{\partial \DFeld[v]}{\partial t} + \StromDichte[v] = \rotation\HFeld[v] &\\
    \Aboxed{\rotation\HFeld[v] &- \frac{\partial \DFeld[v]}{\partial t} = \StromDichte[v]} 
    \end{align*}
\end{itemize}
\end{frame}

\begin{frame}
  \frametitle{Zwischenfazit}
\begin{itemize}[<+->]
\item Definition der Ladungsdichte und Poincaré (3) folgt sofort eine Gleichung in der Form des Coulomb-Gauß-Gesetzes (\dots es gibt eine Vektorfeld D, so dass\dots):
  \begin{equation*}
    \divergenz\DFeld[v]=\laddichte{V}
\end{equation*}
  
\item Mit dem Axiom der Ladungserhaltung und Poincaré (2) und der obigen Beziehung folgt (\dots es gibt ein Vektorfeld H, so dass \dots ):
  \begin{equation*}
    \rotation\HFeld[v] - \frac{\partial \DFeld[v]}{\partial t} = \StromDichte[v]
\end{equation*}
  
\item Elektrische Anregung (historisch: dielektrische Verschiebung): \(\DFeld[v] \)
\item Magnetische Anregung (historisch: Magnetfeld): \(\HFeld[v] \) 
\item Bisher keinen Rückgriff auf Kräfte gemacht
\item Ladungserhaltung gilt auch mikrophysikalisch \(\to\) die inhomogenen Maxwellgleichungen gelten auch mikrophysikalisch \(\to\) elektrische und magnetische Anregung sind mikrophysikalische Größen
\item Ladung ist auch relativistisch invariant \(\to\) inhomogene Maxwell-Gleichungen können auch relativistisch invariant formuliert werden
\end{itemize}
\end{frame}


\begin{frame}
  \frametitle{Axiom 2: Lorenzkraft}
\begin{itemize}[<+->]
\item Die experimentell hervorragend bestätigtet \textbf{Lorenzkraft} wird als axiomatisch angenommen:
  \begin{equation*}
    \kraft[v] = q \left( \EFeld[v] + \vec{u} \times \BFeld[v] \right)
  \end{equation*}
\item Hierbei wird die \textbf{elektrische Feldstärke} eingeführt
\item \textbf{Relativitätsprinzip}: Physikalische Gesetze sind unabhängig vom Inertialsystem
\begin{itemize}[<+->]
\item Betrachte zwei Inertialsysteme: Laborsystem (Größen ohne ´) bewegt sich mit v   relativ zum Ruhesystem der Ladung q (Größen mit ´)
\item Annahme: keine elektrische Feldstärke im Ruhesystem; es gilt offensichtlich
 \begin{equation*}
    \kraft[v]' = q \left( \EFeld[v]' + \vec{u}' \times \BFeld[v]' \right) = \vec{0}
  \end{equation*}

\item Dann muss diese Kraft auch im Laborsystem verschwinden:
 \begin{equation*}
    \kraft[v] = q \left( \EFeld[v] + \vec{u} \times \BFeld[v] \right) = \vec{0} \to \EFeld[v] = - \vec{u} \times \BFeld[v]
  \end{equation*}
\item Die beiden Feldstärken \(\EFeld[v]\) und \(\BFeld[v]\) (hist: magnetische Induktion) sind \textbf{nicht unabhängig} voneinander!
\end{itemize}
\end{itemize}
\end{frame}

\begin{frame}
  \frametitle{Axiom 3: Erhaltung des magnetischen Flusses}
\begin{itemize}[<+->]
\item Magnetischer Fluss durch Fläche:
  \(\displaystyle
    \Phi = \iint_S \BFeld[v]\cdot \upd\vec{A}
    \)
\item Flusserhaltung analog zur Ladungserhaltung: \(\displaystyle \dfrac{\partial \Phi}{\partial t} + \oiint_{O(S)} \StromDichte[v]^\Phi \cdot \upd\vec{s} = 0\)

\item Stokes-Theorem: \(\displaystyle \iint_S \frac{\partial \BFeld[v]}{\partial t}\cdot\upd\vec{A} + \iint_{S} \rotation \StromDichte[v]^\Phi \cdot \upd\vec{A} = 0 \to \frac{\partial \BFeld[v]}{\partial t}+ \rotation \StromDichte[v]^\Phi = \vec{0}\)


\item Bildet man die Divergenz folgt (wegen \(\divergenz \rotation \ldots = 0\)):
\begin{equation*}
  \divergenz \frac{\partial \BFeld[v]}{\partial t} = 0 \to \divergenz\BFeld[v] = \rho_\text{mag} , \frac{\partial \rho_\text{mag}}{\partial t}=0
\end{equation*}

\item Die magnetische Ladungsträgerdichte \(\rho_\text{mag} \) ist auf jeden Fall zeitlich invariant! 
\end{itemize}
\end{frame}




\begin{frame}
  \frametitle{Axiom 3: Erhaltung des magnetischen Flusses (\dots)}
\begin{itemize}[<+->]
\item Relativitätsprinzip:
\begin{itemize}[<+->]
\item Sei in einem System \(\displaystyle \frac{\partial \rho_\text{mag} (\Ortsr[v])}{\partial t}=0\), dann gibt es an jedem Ort einen zeitlich konstanten Wert
  
\item Für einen Beobachter in einem anderen Initialsystem würde sich die räumliche Verteilung aber zeitlich ändern.
\item Dieser Widerspruch tritt nur dann nicht auf, wenn die magnetische Ladungsdichte eine räumliche Konstante ist. (für alle Orte und alle Zeiten gleich)
\item Logische Konsequenz: 
\begin{equation*}
\rho_\text{mag}(\Ortsr[v], t)=0 \to \boxed{\divergenz \BFeld[v] = 0}
\end{equation*}
\item Weiterhin noch nicht klar, was  \(\StromDichte[v]^\Phi\) ist (hat die gleiche Dimension wie die elektrische Feldstärke)
\begin{align*}
 \onslide<7->{\frac{\partial \BFeld[v]}{\partial t}+ \rotation \StromDichte[v]^\Phi = \vec{0}}
  && \onslide<8->{\boxed{\rotation\EFeld[v] + \frac{\partial \BFeld[v]}{\partial t} = \vec{0}}}
   && \onslide<9->{\EFeld[v] \stackrel{?}{=} \StromDichte[v]^\Phi \rule{1cm}{0pt}}\\
  \onslide<7->{\text{Aus der Flusserhaltung}}
  && \onslide<8->{\text{MG: Induktion}}
  && \onslide<10->{\text{Ja! Folgt auch aus RP.}}
\end{align*}
\end{itemize}
\end{itemize}
\end{frame}

\begin{frame}
  \frametitle{Bestandsaufnahme -- schon fertig?}
\begin{itemize}[<+->]
\item  4 Maxwellgleichungen abgeleitet aus
\begin{itemize}[<+->]
\item  \textbf{Mathematik, Relativitätsprinzip, Ladungserhaltung, Lorenzkraft, Erhaltung des magnetischen Flusses}
\item Nebenher: Kontinuitätsgleichung der Ladung, E und B sind nicht voneinander unabhängige Phänomene 
\end{itemize}
\item 4 x 3 = 12 Feldkomponenten; 3+3+1+1 = 8 partielle DGLs
\item Nur 2 dynamische MG \(\to\) die anderen 2 Gleichungen (\(\divergenz \ldots\)) sind immer erfüllt, wenn sie zu einem Zeitpunkt erfüllt sind \(\to\) eher Nebenbedingungen
\item Also: \textbf{6 dynamische Gleichungen für 12 Komponenten} 
\item Was noch fehlt sind die sogenannten Materialgleichungen (constitutive relations) \(\to\) 6 weitere Gleichungen, z.B. für das Vakuum:
  \begin{equation*}
    \boxed{\DFeld[v] = \varepsilon_0\EFeld[v] \qquad
    \HFeld[v] = \frac{1}{\mu_0} \BFeld[v]}
    \end{equation*}
\end{itemize}
\end{frame}


\begin{frame}
  \frametitle{Axiom 4: Die Materialgleichungen des Vakuums}
\begin{itemize}[<+->]
\item  Man kann zeigen, dass die Materialgleichungen des Vakuums genau diese Form haben müssen, wenn

\begin{itemize}[<+->]
\item  Die MGl des Vakuums sind invariant bei Translation und Rotation \(\to\) eine Eigenschaft des Vakuums
\item Die MGl des Vakuums sind lokal und linear (Felder am gleichen Ort und zur gleichen Zeit werden verknüpft) \(\to\) eine Eigenschaft des Vakuums
\item Kein Mixen von elektrischen und magnetischen Effekten \(\to\) eine Eigenschaft des Vakuums
\end{itemize}
\item Dann (und in einer flachen Raumzeit), ergeben sich die Materialgleichungen in der bekannten Form:

  \begin{equation*}
    \boxed{\DFeld[v] = \varepsilon_0\EFeld[v] \qquad
    \HFeld[v] = \frac{1}{\mu_0} \BFeld[v]}
    \end{equation*}
\end{itemize}
\end{frame}

\begin{frame}
  \frametitle{Bemerkenswert}
\begin{itemize}[<+->]
\item Die Materialgleichungen (des Vakuums) verknüpfen nicht nur die Anregungen und die Feldstärken.
\item  Sie verknüpfen auch \textbf{elektromagnetische Felder} mit der \textbf{Struktur der Raumzeit}.
Tatsächlich gibt es Hinweise dafür, das die Propagation elektromagnetischer Felder, die metrische Struktur der Raumzeit bestimmt (und nicht umgekehrt).
\begin{itemize}
\item Hehl,F.W. and Obukhov, Yu.N.: Foundations of Classical Electrodynamics: Charge, Flux, and Metric (Birkhäuser, Boston, 2003)
\item Lämmerzahl, C. and Hehl, F.W.: \enquote{Riemannian light cone from vanishing birefringence in premetric vacuum electrodynamics}, Phys.Rev. D, vol.70 (2004) 105022 (7pages); arXiv.org/gr-qc/0409072
\end{itemize}
\end{itemize}
\end{frame}

 \input{finalframe.inc}
\end{document}