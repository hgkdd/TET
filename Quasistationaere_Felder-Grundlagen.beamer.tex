\input{head.inc}

% Präambelbefehle für die Präsentation
\title[TET: Quasistationäre Felder I - Grundlagen]{Quasistationäre Felder I - Grundlagen}

\begin{document}
% 
% Frontmatter 
% 
%%%%%%%%%%%%%%%%%%%%%%%%%%%%%%%%%%%%%%%%%%%%%%%%%%%%%%%%%%%%%%%%%%%%%%%%%%%%%%%%%%%%%%%%%%%%%%%%%%%%%%%%%%%%%%%%%%%%%%%%%%%%% 

%% inserts the title page and the table of contents
\maketitle

% 
% Content 
% 
%%%%%%%%%%%%%%%%%%%%%%%%%%%%%%%%%%%%%%%%%%%%%%%%%%%%%%%%%%%%%%%%%%%%%%%%%%%%%%%%%%%%%%%%%%%%%%%%%%%%%%%%%%%%%%%%%%%%%%%%%%%%% 
\section{Quasistationäre Felder I - Grundlagen}

\begin{frame}

  \frametitle{Maxwell-Gleichungen}

  \begin{itemize}[<+->]
  \item Voller Satz der Maxwell-Gleichungen; Zeitableitungen bisher vernachlässigt
\begin{align*}
	\divergenz \BFeld[v] &= 0 & \rotation \EFeld[v] &= - \textcolor{green}{\dfrac{\partial \BFeld[v]}{\partial t}} \\
	\divergenz \DFeld[v] &= \laddichte{V} & \rotation \HFeld[v] &= \StromDichte[v] + \textcolor{red}{\dfrac{\partial \DFeld[v]}{\partial t}}
\end{align*}
\item Wir betrachten nun den Fall, dass \alert{einer} der Terme $\textcolor{green}{\frac{\partial \BFeld[v]}{\partial t}}$ (Induktion) oder $\textcolor{red}{\frac{\partial \DFeld[v]}{\partial t}}$ (Verschiebungsstromdichte) vernachlässigt werden kann.
\item Wir werden sehen: dies liefert noch \alert{keine elektromagnetischen Wellen} als Lösung!
  \item \alert{Magneto-Quasistatik}: $\textcolor{red}{\frac{\partial \DFeld[v]}{\partial t}} = \vec{0}$ oder $\left|\textcolor{red}{\frac{\partial \DFeld[v]}{\partial t}}\right|\ll |\StromDichte[v]|$ (Verschiebungsstrom vernachlässigen)
\item \alert{Elektro-Quasistatik}: $\textcolor{green}{\frac{\partial \BFeld[v]}{\partial t}}=\vec{0}$ (Induktion vernachlässigen) 
  \end{itemize}
\end{frame}

\begin{frame} \frametitle{Harmonische Zeitabhängigkeit - Komplexe
Feldgrößen}
  \begin{itemize}[<+->]
  \item Quasistatische Probleme treten meist im Zusammenhang mit \alert{technischen Quellen} auf.
  \item Es ist daher naheliegend, alle Größen im \alert{Frequenzbereich} darzustellen und beliebige Zeitabhängigkeiten im Sinne von \alert{Fourier-Reihen} darzustellen.
    \item \alert{Komplexe Feldgrößen} (i-te Komponente):
\begin{align*}
		\EFeld_i(\Ortsr[v],\,t) &= \hat{\EFeld}_i(\Ortsr[v])  \cos(\omega  t + \varphi_{E,i}) = \real{\hat{\EFeld}_i(\Ortsr[v]) \euler^{\komplex  (\omega  t +  \varphi_{E,i})} } = \real{\EFeld[u]_i(\Ortsr[v]) \euler^{\komplex  \omega  t } }\\
		\HFeld_i(\Ortsr[v],\,t) &= \hat{\HFeld}_i(\Ortsr[v])  \cos(\omega  t + \varphi_{H,i}) = \real{\hat{\HFeld}_i(\Ortsr[v]) \euler^{\komplex  (\omega  t +  \varphi_{H,i})} } = \real{\HFeld[u]_i(\Ortsr[v]) \euler^{\komplex  \omega  t } }
\end{align*}
\item Dabei werden die \alert{Phasoren} (ruhenden Zeiger) eingeführt:
\begin{align*}
		\EFeld[u]_i(\Ortsr[v]) &= \hat{\EFeld}_i(\Ortsr[v]) \cdot \euler^{\komplex  \varphi_{E,i}} \\
		\HFeld[u]_i(\Ortsr[v]) &= \hat{\HFeld}_i(\Ortsr[v]) \cdot \euler^{\komplex  \varphi_{H,i}}
\end{align*}
\item Offensichtlicher Vorteil: $\frac{\partial}{\partial t} \to \text{ Faktor }\komplex\omega$ $\to$ DGL einfacher!
  \item Wir werden ohne gesonderte Bemerkungen auch zwischen Frequenzbereich und Zeitbereich wechseln!
  \end{itemize}
\end{frame}


\begin{frame}
  \frametitle{Elektro-Quasistatik (EQS)}
  \begin{itemize}[<+->]
  \item Annahme: $\frac{\partial \BFeld[v]}{\partial t} = \vec{0}$ und $\frac{\partial \DFeld[v]}{\partial t} \neq \vec{0}$
    \begin{align*}
 	\rotation \EFeld[v] & = \vec{0}  & \rotation \EFeld[uv] & = \vec{0} \\
 	\rotation \HFeld[v] & = \StromDichte[v] + \frac{\partial \DFeld[v]}{\partial t} & \rotation \HFeld[uv] & = \StromDichte[uv] + \komplex\omega \DFeld[uv]\\
 	\divergenz \DFeld[v] &= \laddichte{V} & \divergenz \DFeld[uv] &= \underline{\laddichte{V}}\\
      \divergenz \BFeld[v] &= 0 & \divergenz \BFeld[uv] &= 0\\
      \frac{\partial \laddichte{V}}{\partial t} + \divergenz \StromDichte[v] &=0 & \komplex\omega \underline{\laddichte{V}} + \divergenz \StromDichte[uv] &=0
    \end{align*}
  \item Beiträge zur Stromdichte:
    \begin{itemize}[<+->]
      \item  $\StromDichte[v] = \StromDichte[v]_\mathrm{L} + \StromDichte[v]_\mathrm{E} + \StromDichte[v]_\mathrm{K} = \kappa \cdot \EFeld[v] + \StromDichte[v]_\mathrm{E} + \StromDichte[v]_\mathrm{K} $
 	\item Leitungsstromdichte \(\StromDichte[v]_\mathrm{L} \) mit $\StromDichte[v]_\mathrm{L} = \kappa \cdot \EFeld[v]$
 	\item Eingeprägte Stromdichte \(\StromDichte[v]_\mathrm{E} \) unabhängig von den Feldgrößen; Ursache EMK, Urspannung
 	\item Konvektionsstromdichte \(\StromDichte[v]_\mathrm{K} \) mit \(\StromDichte[v]_\mathrm{K} = \const \cdot \gradient \laddichte{V}\)
 \end{itemize}
  \end{itemize}
\end{frame}

\begin{frame}
  \frametitle{Elektro-Quasistatik (EQS), Komplexes Potential}
  \begin{itemize}[<+->]
  \item Wegen $\rotation \EFeld[uv]  = \vec{0}$:
    $\boxed{\EFeld[uv] = -\gradient \SkalarPot[u]} \text{ komplexes Skalarpotential} $
  \item $\rotation \HFeld[uv]  = \StromDichte[uv] + \komplex\omega \DFeld[uv]$:
    \begin{align*}
      \divergenz \rotation \HFeld[uv] = 0 &=   \divergenz \left( \StromDichte[uv] + \komplex\omega \DFeld[uv] \right) = \divergenz \left( (\kappa \EFeld[uv] + \StromDichte[uv]_\mathrm{E} + \StromDichte[uv]_\mathrm{K})  + \komplex\omega \varepsilon \EFeld[uv] \right)\\
                                            &= \divergenz \left( (\kappa+\komplex\omega \varepsilon) \EFeld[uv] + \StromDichte[uv]_\mathrm{E} + \StromDichte[uv]_\mathrm{K} \right)\\
                                            &= -(\kappa+\komplex\omega \varepsilon) \laplace \SkalarPot[u] + \divergenz (\StromDichte[uv]_\mathrm{E} + \StromDichte[uv]_\mathrm{K} ) \\
      \Aboxed{\laplace \SkalarPot[u] &= \frac{1}{\kappa+\komplex\omega \varepsilon}  \divergenz (\StromDichte[uv]_\mathrm{E} + \StromDichte[uv]_\mathrm{K} ) } \text{ Poisson-Gleichung (komplex)}
    \end{align*}
  \item Im Frequenzbereich ist die \alert{erscheint} die Poisson-Gleichung statisch
  \item Einige Anwendungsbereiche (nicht verwindende Verschiebungsströme; keine Induktion):
      \begin{itemize}[<+->]
      \item Hoch- und Höchstspannungstechnik bei (sehr) niedrigen Frequenzen
      \item Feldprobleme in Halbleiterbauelementen (z.B. FET)
        \item Biophysik; Nervenleitung
        \end{itemize}
        \item Tiefe Frequenzen: Felder erscheinen statisch + Magnetfeld verschwindet $\to$ \alert{EQS}
  \end{itemize}
\end{frame}


\begin{frame}
  \frametitle{Magneto-Quasistatik (MQS)}
  \begin{itemize}[<+->]
  \item Annahme: $\frac{\partial \BFeld[v]}{\partial t} \ne \vec{0}$ und $\frac{\partial \DFeld[v]}{\partial t} = \vec{0}$
    \begin{align*}
 	\rotation \EFeld[v] & = -\frac{\partial \BFeld[v]}{\partial t}  & \rotation \EFeld[uv] & = -\komplex\omega\BFeld[uv] \\
 	\rotation \HFeld[v] & = \StromDichte[v] & \rotation \HFeld[uv] & = \StromDichte[uv] \\
 	\divergenz \DFeld[v] &= \laddichte{V} & \divergenz \DFeld[uv] &= \underline{\laddichte{V}}\\
      \divergenz \BFeld[v] &= 0 & \divergenz \BFeld[uv] &= 0\\
      \divergenz \StromDichte[v] &=0 & \divergenz \StromDichte[uv] &=0
    \end{align*}
  \item Beiträge zur Stromdichte: $\StromDichte[v] = \StromDichte[v]_\mathrm{L} + \StromDichte[v]_\mathrm{E} + \StromDichte[v]_\mathrm{K} = \kappa \cdot \EFeld[v] + \StromDichte[v]_\mathrm{E} + \StromDichte[v]_\mathrm{K} $
  \item $\rotation \HFeld[v]  = \StromDichte[v]$ und $\divergenz \BFeld[v] = 0$ $\to$ Magnetfelder eindeutig bestimmt
    \item Zusammen mit $\divergenz \StromDichte[v] =0$ $\to$ Formeln der Magnetostatik strukturgleich in MQS 
  \item Einige Anwendungsbereiche (verwindende Verschiebungsströme; Induktion):
      \begin{itemize}[<+->]
      \item Skin-Effekt (Übertragungsleitungen, insbesondere bei höheren Frequenzen)
      \item Induktion (klassische elektrotechnische Antriebe und Generatoren)
        \end{itemize}
        \item Tiefe Frequenzen: Felder erscheinen statisch + E-Feld verschwindet $\to$ \alert{MQS}
  \end{itemize}
\end{frame}


\begin{frame}
  \frametitle{Verwendungsbereich EQS, MQS - was ist langsam?}
  \begin{itemize}[<+->]
  \item Betrachte Gebiet mit \alert{Linearausdehnung} $L$
  \item Typische \alert{Zeitkonstante}: $\tau = 1/\omega$
  \item räumliche Ableitungen ($\divergenz, \rotation, \gradient$): $\propto 1/L$
  \item zeitliche Ableitung: $\sim 1 /\tau = \omega$
  \item $\rotation \HFeld[v]  = \StromDichte[v] \to \frac{H}{L} = J \to B  = \mu H = \mu L J \to \alert{\dot{B}  =  \omega \mu L J}$
  \item $\rotation \EFeld[v]  = -\frac{\partial \BFeld[v]}{\partial t} \to \frac{E}{L} = - \dot{B} = - \omega \mu L J \to D = - \omega \mu\varepsilon L^2 J \to \alert{\dot{D} = - \omega^2 \mu\varepsilon L^2 J}$
  \item Forderung: $|\dot{\vec{D}}| \ll |\vec{J}|$: $\omega^2 \mu\varepsilon L^2 J \ll J \to \omega^2 \ll \frac{1}{\varepsilon \mu} \frac{1}{L^2}$
    $$
    \boxed{\omega \ll \frac{1}{\sqrt{\varepsilon \mu}} \frac{1}{L}} = \frac{v_c}{L} \le \frac{c}{L} \quad c: \text{ Vakuum-Lichtgeschwindigkeit; } c\approx 3\cdot 10^8 \text{m/s}
    $$
  \item $L=1$ m: $f=\frac{\omega}{2\pi} \ll 50$ MHz
  \item $2$ m $\to$ $25$ MHz; $10$ m $\to$ $5$ MHz, \dots
    \item Umgekehrt betrachtet: $50$ Hz $\to$ $1000$ km
  \end{itemize}
\end{frame}



\input{finalframe.inc}
   
\end{document}