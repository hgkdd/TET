\input{head.inc}
\usepackage{pdfpages}
\usepackage{fancyhdr}
\usepackage{xcolor}

\fancypagestyle{importedpages}{%
  \fancyhf{}% Clear header/footer
  \renewcommand{\headrulewidth}{0pt}% Remove header rule
  \renewcommand{\footrulewidth}{0pt}% Remove footer rule (default)
  \fancyfoot[R]{\raisebox{.75\baselineskip}[0pt][0pt]{{\color{gray}\fontsize{6}{7} \selectfont Seite \thepage}}}% Lower page number into position
}

\begin{document}
\fancyfootoffset{-1.26cm}
\includepdfset{pages=-,fitpaper,pagecommand={\thispagestyle{importedpages}}}
\includepdf{Titel-tet1.handout.pdf} 
\section{Begriffsbestimmung}
\includepdf{Begriffsbestimmung.handout.pdf}
\section{Vorwissen}
\includepdf{Vorwissen.handout.pdf}
\section{Axiomatische Grundlagen}
\includepdf{Axiomatische_Grundlagen.handout.pdf}
\section{Einführung}
\includepdf{Einfuehrung.handout.pdf}
\section{Verhalten an Grenzflächen}
\includepdf{Verhalten_an_Grenzflaechen.handout.pdf}
\section{Einteilung elektromagnetischer Felder}
\includepdf{Einteilung_elektromagnetischer_Felder.handout.pdf}
\section{Elektrostatik}
\subsection{Grundgleichungen, Größen und Begriffe}
\includepdf{Elektrostatik-I.handout.pdf}
\subsection{Monopol bis Multipolentwicklung}
\includepdf{Elektrostatik-II.handout.pdf}
\subsection{Randwertprobleme}
\includepdf{Elektrostatik-III.handout.pdf}
\subsection{Formale Lösung}
\includepdf{Elektrostatik-IV-Formale_Loesung.handout.pdf}
\subsection{Spiegelungsmethode}
\includepdf{Elektrostatik-V-Spiegelungsmethode.handout.pdf}
\subsection{Orthogonale Funktionensysteme}
\includepdf{Elektrostatik-VI-Orthogonale_Funktionensysteme.handout.pdf}
\subsection{Separationsverfahren}
\includepdf{Elektrostatik-VII-Separationsverfahren.handout.pdf}
\subsection{Materie}
\includepdf{Elektrostatik-VIII-Materie.handout.pdf}
\section{Stationäres Elektrisches Strömungsfeld}
\includepdf{Stationaeres-Stroemungsfeld.handout.pdf}
\section{Magnetostatik}
\subsection{Grundlagen}
\includepdf{Magnetostatik-I_Grundlagen.handout.pdf}
\subsection{Materie}
\includepdf{Magnetostatik-II_Materie.handout.pdf}
\section{Quasistationäre Felder}
\subsection{Grundlagen}
\includepdf{Quasistationaere_Felder-Grundlagen.handout.pdf}
\subsection{Felddiffusion}
\includepdf{Quasistationaere_Felder-Diffusion.handout.pdf}
\subsection{Felddiffusion im Halbraum}
\includepdf{Quasistationaere_Felder-III-Felddiffusion_im_Halbraum.handout.pdf}
\subsection{Induktion}
\includepdf{Quasistationaere_Felder-IV-Induktion.handout.pdf}
\input{finalframe-handout.inc}
\end{document}