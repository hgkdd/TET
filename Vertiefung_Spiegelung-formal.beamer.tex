\input{head-vertiefung.inc}
\usepackage{tikz-3dplot}
% Präambelbefehle für die Präsentation
\title[TET Vertiefung: Spiegelungsmethode -- formal]{Spiegelungsmethode -- formal}

\begin{document}
% 
% Frontmatter 
% 
%%%%%%%%%%%%%%%%%%%%%%%%%%%%%%%%%%%%%%%%%%%%%%%%%%%%%%%%%%%%%%%%%%%%%%%%%%%%%%%%%%%%%%%%%%%%%%%%%%%%%%%%%%%%%%%%%%%%%%%%%%%%% 

%% inserts the title page and the table of contents
\maketitle

% 
% Content 
% 
%%%%%%%%%%%%%%%%%%%%%%%%%%%%%%%%%%%%%%%%%%%%%%%%%%%%%%%%%%%%%%%%%%%%%%%%%%%%%%%%%%%%%%%%%%%%%%%%%%%%%%%%%%%%%%%%%%%%%%%%%%%%% 
\section{Spiegelung - formal}

\begin{frame}
  \frametitle{Ausgangspunkt und Zielsetzung}

    \begin{itemize}[<+->]
    \item Die \alert{Spiegelungsmethode} wurde zur Lösung elektrostatischer Randwertprobleme schon vorher betrachtet (\url{https://bildungsportal.sachsen.de/opal/auth/RepositoryEntry/27455913992/CourseNode/103138906469436/Theoretische\%20Elektrotechnik/Elektrostatik/TET-Elektrostatik-V-Spiegelungsmethode.pdf})
      \item Beschränkung dabei: \alert{idealleidende} Ränder 
    \item Zielstellung: Erweiterung der Methode auf Grenzen zwischen \alert{dielektrischen} Materialien
    \item Hierfür wird eine \alert{stärker formalisierte Herleitung} benötigt
    \end{itemize}
\end{frame}

\begin{frame}
  \frametitle{Zunächst: Berandung durch idealen Leiter, geerdet}

  \begin{columns}
    \begin{column}{.45\textwidth}
      reales Problem:

      
      \begin{tikzpicture}[line width = 1.2pt, line join=round,>=stealth]
        \filldraw [color=blue!70] (2.5,2.5) circle (1pt) node[above] {$Q_i, \Ortsr[vs]_i$};
        \filldraw [color=blue!70] (2.9,2.1) circle (1pt); 
        \filldraw [color=blue!70] (2.7,1.9) circle (1pt); 
        % Ladungsdichte
        \coordinate (a) at (2.1,1.8);
        \coordinate (b) at (2,1.2);
        \coordinate (c) at (1.8,0.4);
        \coordinate (d) at (1.3,0.6);
        \coordinate (e) at (0.7,0.5);
        \coordinate (f) at (0.5,2);
        \coordinate (g) at (1.2,2.5);
        \shade[ball color=white!10!green!20,opacity=0.20] plot [smooth cycle, tension = 1] coordinates {(a) (b) (c) (d) (e) (f) (g)};
        \draw [color=darkgreen] plot [smooth cycle, tension = 1] coordinates {(a) (b) (c) (d) (e) (f) (g)} node [sloped, above] {$O(V_2)$};
        \draw (1.3,-0.2) node[align=center] {$\laddichte{F} \neq 0 $ auf $O(V_2)$\\
        $\SkalarPot = 0$ auf $O(V_2)$};
        \draw(1,1.1) node {$V_2$};
        \draw(1.2,0.8) node{$\kappa_2 \to \infty$};
        \draw(0,2.2) node[align=center] {$V_1$\\$\varepsilon=\varepsilon_1$}; 
      \end{tikzpicture}
\end{column}
    \begin{column}{.45\textwidth}
      Ersatzproblem:

      
      \begin{tikzpicture}[line width = 1.2pt, line join=round,>=stealth]
        \filldraw [color=blue!70] (2.5,2.5) circle (1pt) node[above] {$Q_i, \Ortsr[vs]_i$};
        \filldraw [color=blue!70] (2.9,2.1) circle (1pt); 
        \filldraw [color=blue!70] (2.7,1.9) circle (1pt); 
        \filldraw [color=red!70] (1.3,1.9) circle (1pt) node[above] {$Q^\star_i, \Ortsr[v]^\star_i$};
        \filldraw [color=red!70] (1.2,1.7) circle (1pt); 
        \filldraw [color=red!70] (1.7,1.6) circle (1pt); 
        % Ladungsdichte
        \coordinate (a) at (2.1,1.8);
        \coordinate (b) at (2,1.2);
        \coordinate (c) at (1.8,0.4);
        \coordinate (d) at (1.3,0.6);
        \coordinate (e) at (0.7,0.5);
        \coordinate (f) at (0.5,2);
        \coordinate (g) at (1.2,2.5);
        \shade[ball color=white!10!green!20,opacity=0.20] plot [smooth cycle, tension = 1] coordinates {(a) (b) (c) (d) (e) (f) (g)};
        \draw [color=darkgreen] plot [smooth cycle, tension = 1] coordinates {(a) (b) (c) (d) (e) (f) (g)} node [sloped, above] {$O(V_2)$};
        \draw (1.3,-0.2) node[align=center] {$\laddichte{F} = 0 $ auf $O(V_2)$\\
        $\SkalarPot \stackrel{!}{=} 0$ auf $O(V_2)$};
        \draw(1,1.1) node {$V_2$};
        \draw(1.2,0.8) node{$\varepsilon=\varepsilon_1$};
        \draw(0,2.2) node[align=center] {$V_1$\\$\varepsilon=\varepsilon_1$}; 
      \end{tikzpicture}
\end{column}
    \end{columns}

\pause
\begin{block}{Finden der Ersatzladungen und deren Positionen}
  \begin{enumerate}[<+->]
  \item Abbildung der Quellpunkte: $T: \mathbb{R}^3 \to \mathbb{R}^3:\; \Ortsr[vs] \to \Ortsr[v]^\star = T( \Ortsr[vs])$ (stetig diff.-bar)\\
    Bildladungen sollen in $V_2$ liegen: $T( V_1) \subset V_2$\\
    $O(V_2)$ soll auf sich selbst abgebildet werden: $T(\Ortsr[v] )= \Ortsr[v]$ für $\Ortsr[v] \in O(V_2)$ 
  \item Faktor für Bildladungen: $\lambda: V_1 \cup O(V_2) \to \mathbb{R}: \Ortsr[vs] \to \lambda(\Ortsr[vs])$ (stetig diff.-bar)\\
    Ladung auf $O(V_2)$ nicht skalieren: $\lambda(\Ortsr[vs]) = 1$ für $\Ortsr[vs] \in O(V_2)$
  \end{enumerate}
  \pause
  Insgesamt: $(Q_i, \Ortsr[vs]_i) \to (Q_i^\star, \Ortsr[v]_i^\star) $ mit $Q_i^\star = -\lambda(\Ortsr[vs]_i)\, Q_i$ und $\Ortsr[v]_i^\star = T(\Ortsr[vs]_i)$ (\enquote{-}: Konvention)
\end{block}
  
  \end{frame}

  \begin{frame}
    \frametitle{Ersatzproblem}
    \begin{itemize}[<+->]
    \item Für das \alert{Ersatzproblem} kann das elektrische Skalarpotential $\SkalarPot(\Ortsr[v])$ hervorgerufen von den \alert{Originalladungen} $(Q_i, \Ortsr[vs]_i)$ und den \alert{Ersatzladungen} $(Q_i^\star, \Ortsr[v]_i^\star)$ direkt mit der \alert{Greenschen Funktion}
      \begin{equation*}
        G(\Ortsr[v],\Ortsr[vs]) = \frac{1}{4\pi\varepsilon_1} \frac{1}{\left|\Ortsr[v] - \Ortsr[vs] \right|}
      \end{equation*}
      des homogenen Freiraums $V_2\cup V_1$ berechnet werden.
    \item Es gilt somit:
      \begin{eqnarray*}
        \SkalarPot(\Ortsr[v]) & = &\sum_i \left[ Q_i G(\Ortsr[v],\Ortsr[vs]_i) + Q_i^\star G(\Ortsr[v],\Ortsr[v]_i^\star)\right]\\
                                     & = &  \sum_i \left[ Q_i G(\Ortsr[v],\Ortsr[vs]_i) -\lambda(\Ortsr[vs]_i) Q_i G(\Ortsr[v],T(\Ortsr[vs]_i))\right]\\
                                     & = & \sum_i Q_i \left[G(\Ortsr[v],\Ortsr[vs]_i) -\lambda(\Ortsr[vs]_i) G(\Ortsr[v],T(\Ortsr[vs]_i))\right]
      \end{eqnarray*}
      \item Für $\Ortsr[v] \in O(V_2)$ muss gelten: $\SkalarPot(\Ortsr[v]) = 0$
      \item Für $\Ortsr[v] \in O(V_2)$ gilt auch: $\lambda(\Ortsr[v]) = 1$ und $T(\Ortsr[v]) = \Ortsr[v]$ 
 \end{itemize}   
    
    \end{frame}

    \begin{frame}
      \frametitle{Bedingung für Greensche Funktion}
    \begin{itemize}[<+->]
\item Wir betrachten das Skalarpotential für das Ersatzproblem:
      \begin{equation*}
        \SkalarPot(\Ortsr[v]) =  \sum_i Q_i \left[G(\Ortsr[v],\Ortsr[vs]_i) -\lambda(\Ortsr[vs]_i) G(\Ortsr[v],T(\Ortsr[vs]_i))\right]
      \end{equation*}
      \item Auf der Oberfläche von $V_2$ muss das Potential verschwinden:
      \begin{equation*}
        \SkalarPot(\Ortsr[v]) =  \sum_i Q_i \left[G(\Ortsr[v],\Ortsr[vs]_i) -\lambda(\Ortsr[vs]_i) G(\Ortsr[v],T(\Ortsr[vs]_i))\right] \stackrel{!}{=} 0 \text{ für } \Ortsr[v] \in O(V_2)
      \end{equation*}
    \item \alert{Hinreichend} dafür ist, dass gilt
      \begin{equation*}
        \boxed{\lambda(\Ortsr[vs]_i) G(\Ortsr[v],T(\Ortsr[vs]_i)) \stackrel{!}{=}  \lambda(\Ortsr[v]) G(T(\Ortsr[v]), \Ortsr[vs]_i)} 
      \end{equation*}
    \item Denn dann gilt auf $O(V_2)$
      \begin{equation*}
        \lambda(\Ortsr[vs]_i) G(\Ortsr[v],T(\Ortsr[vs]_i)) =  \underbrace{\lambda(\Ortsr[v])}_{=1} G(\underbrace{T(\Ortsr[v])}_{=\Ortsr[v]},\Ortsr[vs]_i) =  G(\Ortsr[v],\Ortsr[vs]_i) \text{ für } \Ortsr[v] \in O(V_2)
      \end{equation*}
  \item Man bezeichnet dies als \alert{Umwältzen vom Quellpunkt zum Aufpunkt}    
 \end{itemize}   
      
\end{frame}


\begin{frame}
  \frametitle{Problemangepasste Greensche Funktion}
  \begin{itemize}[<+->]
  \item Mit der Eigenschaft
      \begin{equation*}
        \boxed{\lambda(\Ortsr[vs]_i) G(\Ortsr[v],T(\Ortsr[vs]_i)) \stackrel{!}{=}  \lambda(\Ortsr[v]) G(T(\Ortsr[v]), \Ortsr[vs]_i)} 
      \end{equation*}
ist das Skalarpotential    
      \begin{equation*}
        \SkalarPot(\Ortsr[v]) =  \sum_i Q_i \left[G(\Ortsr[v],\Ortsr[vs]_i) -\lambda(\Ortsr[vs]_i) G(\Ortsr[v],T(\Ortsr[vs]_i))\right] =\sum_i Q_i \left[G(\Ortsr[v],\Ortsr[vs]_i) -\lambda(\Ortsr[v]) G(T(\Ortsr[v]),\Ortsr[vs]_i)\right]
      \end{equation*}
      Lösung des \alert{ursprünglichen Problems}.
    \item Denn für $\Ortsr[v] \in V_1 \cup O(V_2)$ gehen hier nur die Originalladungen an den Originalorten ein und die Randbedingung ist automatisch erfüllt.
    \item Die \alert{problemangepasste Greensche Funktion} ist damit
      \begin{equation*}
        \boxed{G^\star(\Ortsr[v],\Ortsr[vs]_i) = G(\Ortsr[v],\Ortsr[vs]_i) -\lambda(\Ortsr[v]) G(T(\Ortsr[v]),\Ortsr[vs]_i)}
      \end{equation*}
    \item Das Problem reduziert sich somit auf das Finden der Abbildungen $T$ und $\lambda$ \\
      $\to$ einfache Beispiele
    \end{itemize}
\end{frame}


      
\input{finalframe.inc}
\end{document}