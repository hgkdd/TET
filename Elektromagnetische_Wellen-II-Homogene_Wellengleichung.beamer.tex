\input{head.inc}
  
% Präambelbefehle für die Präsentation
\title[TET: Elektromagnetische Wellen II - Homogene Wellengleichung]{Elektromagnetische Wellen II - Homogene Wellengleichung}

\begin{document}
% 
% Frontmatter 
% 
%%%%%%%%%%%%%%%%%%%%%%%%%%%%%%%%%%%%%%%%%%%%%%%%%%%%%%%%%%%%%%%%%%%%%%%%%%%%%%%%%%%%%%%%%%%%%%%%%%%%%%%%%%%%%%%%%%%%%%%%%%%%% 

%% inserts the title page and the table of contents
\maketitle

% 
% Content 
% 
%%%%%%%%%%%%%%%%%%%%%%%%%%%%%%%%%%%%%%%%%%%%%%%%%%%%%%%%%%%%%%%%%%%%%%%%%%%%%%%%%%%%%%%%%%%%%%%%%%%%%%%%%%%%%%%%%%%%%%%%%%%%% 
\section{Elektromagnetische Wellen II - Homogene Wellengleichung}

\begin{frame}
  \frametitle{Ausgangspunkt}
  \begin{itemize}[<+->]
  \item Entkoppelte Gleichungen für Felder und Potentiale in \alert{Lorenzeichung}:
    \begin{align*}
      \laplace \EFeld[v]-\varepsilon\mu \frac{\d^2}{\d t^2}\EFeld[v] &= \gradient \frac{\laddichte{V}}{\varepsilon}+\mu \frac{\d \StromDichte[v]}{\d t} && \laplace\SkalarPot-\varepsilon\mu \frac{\d^2}{\d t^2}\SkalarPot = -\frac{\laddichte{V}}{\varepsilon}\\
      \laplace\HFeld[v]-\varepsilon\mu\frac{\d^2}{\d t^2}\HFeld[v]& = -\rotation\StromDichte[v] && \laplace\VektorPot[v]-\varepsilon\mu \frac{\d^2}{\d t^2}\VektorPot[v] = -\mu \StromDichte[v]
      \end{align*}
  \item Für \(\laddichte{V}=0\) und \(\StromDichte[v]=\vec{0}\) (\(\rightarrow\) keine Quellen im Lösungsgebiet) besteht immer die gleiche formale Struktur:
\begin{equation*}
(\laplace -\varepsilon\mu\frac{\d^2}{\d t^2})\Psi = \square\Psi = 0\pointspace ; \pointspace \square = \laplace - \varepsilon\mu\frac{\d^2}{\d t^2} \text{ \alert{Wellenoperator}, \alert{D'Alembert-O.}, \alert{Quabla}} 
\end{equation*}    
\item   \(\Psi = \Psi(\Ortsr[v], t)\): Eine Komponente von \(\EFeld[v]\), \(\BFeld[v]\), \(\DFeld[v]\), \(\HFeld[v]\), \(\VektorPot[v]\) oder \(\SkalarPot\) (Potentiale in Lorenz-Eichung)
\item Im Vakuum: \(\varepsilon\mu = \varepsilon_0 \mu_0 = \frac{1}{\lichtgeschw^2}\); \(\mu_0 = \SI{4\pi e-7}{\henry\per\metre}\) (rel. Fehler \(\approx \SI{2e-10}{}\))
 \begin{equation*}
 \text{\alert{Vakuum-Lichtgeschwindigkeit}: }\boxed{\lichtgeschw = \SI{299792458}{\metre\per\second}} \approx \SI{3e8}{\metre\per\second}
\end{equation*}
\item Sonst: \(\varepsilon\mu = \frac{1}{\Geschwindigkeit_{c}^2}\); \(\Geschwindigkeit_{c}\) Lichtgeschwindigkeit im Medium
 \begin{equation*}
c\ge \Geschwindigkeit_{c} = \frac{1}{\sqrt{\varepsilon\mu}} = \frac{1}{\sqrt{\varepsilon_0\mu_0}\sqrt{\varepsilon_\mathrm{r}\mu_\mathrm{r}}} = \frac{\lichtgeschw}{\sqrt{\varepsilon_\mathrm{r}\mu_\mathrm{r}}} =\frac{\lichtgeschw}{\brechind} \pointspace ; \pointspace \brechind = \sqrt{\varepsilon_\mathrm{r}\mu_\mathrm{r}} \text{ \alert{Brechungsindex}}
\end{equation*}
  \end{itemize}
\end{frame}


\begin{frame}
  \frametitle{Homogene Wellengleichung - Ebene Wellen}
  \begin{itemize}[<+->]
\item Wir betrachten im Folgenden die \alert{Homogene Wellengleichung}
    \begin{equation*}
\laplace\Psi (\Ortsr[v], t) - \frac{1}{\Geschwindigkeit_{c}^2}\frac{\d ^2}{\d t^2}\Psi (\Ortsr[v], t) = 0 \qquad \boxed{\square \Psi (\Ortsr[v], t) = 0} 
\end{equation*}
\item Wir suchen und analysieren Lösungen der Wellengleichung
\item Zunächst: Lösungen die sich nur in \alert{einer Dimension ausbreiten} \(\to\) \alert{Ebene Wellen}
\item Ansatz:
  \begin{equation*}
    \Psi(\Ortsr[v], t) = \Psi (\omega t + \Wellenzahl[v]\cdot\Ortsr[v]) \pointspace ; \pointspace \varphi (\Ortsr[v], t) = \omega t + \Wellenzahl[v]\cdot\Ortsr[v] \quad \alert{Phasenfunktion} 
  \end{equation*}
\item Einsetzen in die homogene Wellengleichung:
  \begin{align*}
& \laplace\Psi = \Wellenzahl^2\cdot\frac{\d^2 \Psi}{\d \varphi^2}\pointspace ; \pointspace \frac{\d^2}{\d t^2}\Psi = \omega^2\frac{\d^2 \Psi}{\d \varphi^2}\quad\Wellenzahl[v]:\text{ \alert{Wellenvektor}};\quad\Wellenzahl:\text{ \alert{Wellenzahl}}\\
\Rightarrow &\left( \Wellenzahl^2 -\frac{\omega^2}{\Geschwindigkeit_{c}^2}\right)\frac{\d^2 \Psi}{\d \varphi^2} = 0\\
\Rightarrow & \Wellenzahl^2 = \frac{\omega^2}{\Geschwindigkeit_{c}^2} \to \quad\Wellenzahl_{1,2}=\Wellenzahl_\pm=\pm\frac{\omega}{\Geschwindigkeit_{c}}=\pm \frac{\brechind\cdot \omega}{\lichtgeschw}\pointspace;\quad \text{oBdA:}\pointspace \omega\geq 0
\end{align*}
  \end{itemize}
\end{frame}


\begin{frame}
  \frametitle{Homogene Wellengleichung - Ebene Wellen (\dots)}
  \begin{itemize}[<+->]
  \item Für jeden Wert des Parameters \(\omega\) (\alert{Kreisfrequenz}) gibt es zwei mögliche Werte für die \alert{Wellenzahl} \(\Wellenzahl = |\Wellenzahl[v]|\). Die allgemeine Lösung (zum gewählten Ansatz) der homogenen Wellengleichung ist somit
    \begin{equation*}
       \boxed{ \Psi (\Ortsr[v], t) = \sum_\omega \left( \Psi_+ (\omega t + \Wellenzahl[v]\cdot\Ortsr[v]) + \Psi_- (\omega t - \Wellenzahl[v]\cdot\Ortsr[v]) \right) } \pointspace ; \pointspace \Wellenzahl[v] = \Wellenzahl \frac{\Wellenzahl[v]}{\Wellenzahl} = \frac{\omega}{\Geschwindigkeit_{c}}\frac{\Wellenzahl[v]}{\Wellenzahl} =\frac{\omega}{\Geschwindigkeit_{c}} \vu{\Wellenzahl}  
     \end{equation*}
  \item Betrachtung der \alert{Phasenfunktion} \(\varphi_\pm (\Ortsr[v], t) = \omega t \pm \Wellenzahl[v]\cdot\Ortsr[v] \)
  \item Offenbar: \( \varphi (\Ortsr[v], t) = \text{const} \Rightarrow \Psi = \text{const} \to\) \alert{\enquote{Flächen gleicher Phase = Flächen gleicher Werte}}
  \item Momentaufnahme zum Zeitpunkt \(t=t_0\):
    \begin{equation*}
      \varphi_\pm (\Ortsr[v], t_0) = \omega t_0 \pm \Wellenzahl[v]\cdot\Ortsr[v] \to \varphi_\pm=\text{const} \Leftrightarrow  \Wellenzahl[v]\cdot\Ortsr[v]=\text{const}
      \end{equation*}
  \item Bedeutung von \( \Wellenzahl[v]\cdot\Ortsr[v]\): \(|\Wellenzahl[v]|\) mal Projektion von \(\Ortsr[v]\)  auf \(\Wellenzahl[v]\)
    \begin{columns}<+->
      \begin{column}{.25\linewidth}
        \xincfig{ebeneWelle}{\columnwidth}
      \end{column}
      \begin{column}{.75\linewidth}
        \begin{itemize}[<+->]
        \item Orte mit \(\Wellenzahl[v]\cdot\Ortsr[v]=\text{const} \to\) Ebenen \(\perp \Wellenzahl[v]\)
        \item Orte, die zu einem Zeitpunkt \(t=t_0\) gleiche Werte \(\Psi\) haben, liegen auf Ebenen senkrecht zu \(\Wellenzahl[v]\)
          \item \(\to\) \alert{ebene Wellenfront} \(\to\) \alert{ebene Wellen}
  \end{itemize}
        \end{column}
\end{columns}
  \end{itemize}
\end{frame}

\begin{frame}
  \frametitle{Homogene Wellengleichung - Ebene Wellen (\dots)}
  \begin{columns}<+->
    \begin{column}{.4\linewidth}
      \centerline{\xincfig{ebeneWelle2}{\columnwidth}}
    \end{column}
    \begin{column}{.6\linewidth}
        \begin{itemize}[<+->]
        \item Betrachte \(t_1 > t_0\) mit gleichem Wert für $\Psi$:
        \item Dies sind Orte wiederkehrender Phase:
          \begin{align*}
            \varphi_0 &= \omega t_0 + \Wellenzahl[v]\cdot\Ortsr[v]_0 \stackrel{!}{=} \omega t_1 \cpm \Wellenzahl[v]\cdot\Ortsr[v]_1\\
            \Rightarrow &\cpm \Wellenzahl[v]\cdot\Ortsr[v]_1 = \varphi_0 - \omega t_1\pointspace ; \pointspace \Wellenzahl[v]\cdot\Ortsr[v]_1 = k r_{1\parallel} \text{ mit } r_{1\parallel}=\frac{\Wellenzahl[v]\cdot\Ortsr[v]_1}{\Wellenzahl}  \\
            \Rightarrow &\cpm \Ortsr_{1\parallel} =\frac{\varphi_0}{\Wellenzahl} - \frac{\omega}{k} t_1\Rightarrow \Ortsr_{1\parallel} = \cpm\frac{\varphi_0}{\Wellenzahl} \cmp \frac{\omega}{k} t_1\\
            \end{align*}
  \end{itemize}
        \end{column}
    \end{columns}
  \begin{itemize}[<+->]
  \item \textcolor{green!70}{Lösungen} \((\omega t \textcolor{green!70}{+} \Wellenzahl[v]\cdot\Ortsr[v])\): Wellenfront bewegt sich in Richting \(-\Wellenzahl[v]\)
  \item \textcolor{red}{Lösungen} \((\omega t \textcolor{red}{-} \Wellenzahl[v]\cdot\Ortsr[v])\): Wellenfront bewegt sich in Richting \(+\Wellenzahl[v]\)
  \item Für die weitere Analyse: \((\omega t \textcolor{red}{-} \Wellenzahl[v]\cdot\Ortsr[v])\): Wellenfront bewegt sich in Richting \(+\Wellenzahl[v]\)
    \item \alert{Phasengeschwindigkeit}: \(\boxed{\Geschwindigkeit_p =\frac{\upd r_{\parallel}}{\upd t} = \frac{\omega}{\Wellenzahl}} \stackrel{\text{hier}}{=}{\Geschwindigkeit_c}\)
  \end{itemize}
\end{frame}


\input{finalframe.inc}
   
\end{document}