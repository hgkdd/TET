\begin{tikzpicture}[line width = 1.2pt, line join=round,x=0.5cm,y=0.5cm,>=stealth]
	% Mittellinie zeichnen
	\draw (0,-6) -- (0,7);
	% Schleife zeichnen
	\draw (-1.5,-5) rectangle (1.5,5);
	% Richtungen der Schleife
	\draw [->] (-1.5,3) -- (-1.5,2);
	\draw [->] (1.5,1) -- (1.5,2);
	% Vektoren für ds Zeichnen
	\draw [->,line width = 2pt] (-1.5,0) -- (-1.5,-2);
	\draw (-1.5,-1) node[anchor=east] {$\upd \Weg[v]_1$};
	\draw [->,line width = 2pt] (1.5,-2) -- (1.5,0);
	\draw (1.5,-1) node[anchor=west] {$\upd \Weg[v]_2$};
	% Breite der Schleife
	\draw [|<->|] (1.5,5.5) -- (-1.5,5.5) node[anchor=south west] {$ h $};
	% Höhe der Schleife
	\draw [|<->|] (-3.5,-5) -- (-3.5,5);
	\draw (-3.5,0) node[anchor=east] {$ l $};
	% Normalenvektor
	\draw [->] (0,6.5) -- (2,6.5) node[anchor=west] {$ \normalenvektor $};
	% Tangentialvektor
	\draw [->] (0,6.5) -- (0,8.5) node[anchor=south] {$ \tangentialvektor $};
	% Flächenbezeichnung und Randbezeichnung
	\draw (0,-4.3) node[anchor=west] {$ \Flaeche $};
	\draw (1.5,-4.3) node[anchor=west] {$ \rand(\Flaeche) $};
	% elektrisches Feld
	\draw [->] (-2.5,-6) -- (-1,-5.6) node[anchor=north east] {$ \EFeld[v]_1 $};
	\draw [->] (1,-5.9) -- (2.55,-5.7) node[anchor=north east] {$ \EFeld[v]_2 $};
	% Definition der Seiten
	\draw (-4,8) node[anchor=west] {\circled{1}};
	\draw (4,8) node[anchor=east] {\circled{2}};
\end{tikzpicture}