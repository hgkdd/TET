\begin{tikzpicture}[scale=0.8, line width = 1.2pt, line join=round,x=1cm,y=1cm,>=stealth, circuit ee IEC]
	% Koordinaten für die Spiegelfläche
	\coordinate (a) at (0,2);
	\coordinate (b) at (0,-4);
	\coordinate (c) at (4,2);
	% Ladungsdichten 1. Spiegelung
        \onslide<4->{
	\draw (3.5,0) node[anchor=east] {$\laddichte{V}$};
	\draw (3.5,0.75) -- (2,0) -- (3.5,-0.75) -- cycle;}
      \onslide<5->{
	\draw (-3.5,0) node[anchor=west] {$-\laddichte{V}$};
	\draw (-3.5,0.75) -- (-2,0) -- (-3.5,-0.75) -- cycle;}
      \onslide<6->{
	\draw (4.5,0) node[anchor=west] {$-\laddichte{V}$};
	\draw (4.5,0.75) -- (6,0) -- (4.5,-0.75) -- cycle;}
      % Ladungsdichten 2. Spiegelung
      \onslide<7->{
	\draw [color=red] (-4,2) -- ++(b);
	\draw [color=red] ({3.5-8},0) node[anchor=east] {$\laddichte{V}$};
	\draw [color=red] ({3.5-8},0.75) -- ({2-8},0) -- ({3.5-8},-0.75) -- cycle;}
      \onslide<8->{
	\draw [color=red] ({3.5+8},0) node[anchor=east] {$\laddichte{V}$};
	\draw [color=red] (8,2) -- ++(b);
	\draw [color=red] ({3.5+8},0.75) -- ({2+8},0) -- ({3.5+8},-0.75) -- cycle;}
      % Spiegelflächen
      \onslide<4->{
	\draw (a) -- ++(b);
	\draw[decoration=zigzag,decorate] (-0.15,2) -- ++(b);
	\draw (c) -- ++(b);
	\draw[decoration=zigzag,decorate] (4.15,2) -- ++(b);
	% Erdung
	\coordinate (erde) at (0,-1.5);
	\draw (erde) -- ++(0.4,0) to [ground={pos=1}] (0.4,-1.9);
	\filldraw (erde) circle (1.5pt);
	\coordinate (erdez) at (4,-1.5);
	\draw (erdez) -- ++(-0.4,0) to [ground={pos=1}] (3.6,-1.9);
	\filldraw (erdez) circle (1.5pt);}
\end{tikzpicture}