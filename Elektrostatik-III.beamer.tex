\input{head.inc}

% Präambelbefehle für die Präsentation
\title[TET: Elektrostatik-III: Randwertprobleme]{Elektrostatik-III: Randwertprobleme}

\begin{document}
% 
% Frontmatter 
% 
%%%%%%%%%%%%%%%%%%%%%%%%%%%%%%%%%%%%%%%%%%%%%%%%%%%%%%%%%%%%%%%%%%%%%%%%%%%%%%%%%%%%%%%%%%%%%%%%%%%%%%%%%%%%%%%%%%%%%%%%%%%%% 

%% inserts the title page and the table of contents
\maketitle

% 
% Content 
% 
%%%%%%%%%%%%%%%%%%%%%%%%%%%%%%%%%%%%%%%%%%%%%%%%%%%%%%%%%%%%%%%%%%%%%%%%%%%%%%%%%%%%%%%%%%%%%%%%%%%%%%%%%%%%%%%%%%%%%%%%%%%%% 
\section{Elektrostatik-III: Randwertprobleme}

\begin{frame}

  \frametitle{Elektrostatik: Randwertprobleme}

  \begin{block}{Poisson-Gleichung}<+->
    Das Grundproblem der Elektrostatik ist die Lösung der
    \alert{Poisson-Gleichung}
    $$
    \laplace \phi(\Ortsr[v]) = -\frac{1}{\varepsilon_0}
    \laddichte{V} (\Ortsr[v])
    $$
    Falls \alert{keine Randbedingungen auf Grenzflächen im Endlichen}
    erfüllt sein müssen kann das Potential bei bekannter
    Ladungsverteilung aus dem \alert{Coulomb-Integral} (manchmal auch:
    \alert{Poisson-Integral}) berechnet werden:
    $$
   \phi(\Ortsr[v]) = \frac{1}{4\pi\varepsilon_0} \int
   \frac{\laddichte{V}(\Ortsr[vs])}{|\Ortsr[v]-\Ortsr[vs]|}
   \upd^3\Ortsr[s]
   $$
 \end{block}

 \begin{block}{Randwertproblem}<+->
   \alert{Gegeben} sei $\laddichte{V}(\Ortsr[vs])$ in einem Volumen $V$ und \alert{$\phi$} und/oder \alert{$\frac{\partial\phi}{\partial n}=\gradient\phi\cdot\vec{n}=-\EFeld[v] \cdot \vec{n}$} auf Grenz- oder Randflächen in $V$. \alert{Gesucht} ist $\phi(\Ortsr[v])$ für $\Ortsr[v] \in V$.   
   \end{block}
  
 \end{frame}

 \begin{frame}

   \begin{itemize}[<+->]
   \item Warum diesen beiden Randbedingungen: $\phi$ und/oder $\frac{\partial\phi}{\partial n}$? $\to$ Mathematik
   \item Greensche Identitäten
     \begin{align*}
     \int_V \left[ f\laplace g + (\gradient g\cdot \gradient f) \right] \upd^3r' &= \oint_{O(V)} f\frac{\partial g}{\partial n} \upd^2r' &\text{ 1. Greensche Identität} \\
     \int_V \left[ f\laplace g - g \laplace f \right] \upd^3r' &= \oint_{O(V)} \left[ f \frac{\partial g}{\partial n} - g \frac{\partial f}{\partial n} \right] \upd^2r' &\text{ 2. Greensche Identität}
     \end{align*}
   \item Normalenableitung: $\frac{\partial f}{\partial n} = \gradient f \cdot \vec{n}$
     \item Beziehung für die $\delta$-Funktion (Distribution): $\delta^3(\Ortsr[v] - \Ortsr[vs]) = -\frac{1}{4\pi}\laplace\frac{1}{|\Ortsr[v] - \Ortsr[vs]|}$
     \item Um die Poisson-Gleichung in eine Integralgleichung umzuwandeln, setzen wir in der 2. Greenschen Identität $f\to\phi(\Ortsr[vs])$ und $g\to \frac{1}{|\Ortsr[v] - \Ortsr[vs]|}$:
       \begin{align*}
         \int_V \left[  \phi(\Ortsr[vs]) \laplace_{\Ortsr[s]} \frac{1}{|\Ortsr[v] - \Ortsr[vs]|}  - \frac{1}{|\Ortsr[v] - \Ortsr[vs]|} \laplace_{\Ortsr[s]}  \phi(\Ortsr[vs]) \right] \upd^3\Ortsr[s] &=  -4\pi \int_V  \phi(\Ortsr[vs]) \delta^3 (\Ortsr[v] - \Ortsr[vs]) \upd^3r' \\
         &+ \frac{1}{\varepsilon_0} \int_V \frac{\rho(\Ortsr[vs])}{|\Ortsr[v] - \Ortsr[vs]|} \upd^3r'\\
                                                                                                                                                                                                                                                   &= \oint_{O(V)} \left[\phi\frac{\partial}{\partial n'}\frac{1}{|\Ortsr[v] - \Ortsr[vs]|}
                                                                                                                                                                                                                                                     - \frac{1}{|\Ortsr[v] - \Ortsr[vs]|}\frac{\partial\phi}{\partial n'} \right] \upd^2r'
       \end{align*}
     \item Somit:
       $$
      \boxed{ \phi(\Ortsr[v]) = \frac{1}{4\pi\varepsilon_0} \int_V
   \frac{\laddichte{V}(\Ortsr[vs])}{|\Ortsr[v]-\Ortsr[vs]|}
   \upd^3\Ortsr[s] + \frac{1}{4\pi} \oint_{O(V)} \left[\frac{1}{|\Ortsr[v] - \Ortsr[vs]|}\frac{\partial\phi}{\partial n'} - \phi\frac{\partial}{\partial n'}\frac{1}{|\Ortsr[v] - \Ortsr[vs]|} \right] \upd^2r'}
       $$
     \end{itemize}
   
   \end{frame}

   \begin{frame}
     \frametitle{Diskussion der Integralgleichung für $\phi$}
     \begin{itemize}[<+->]
     \item Wir haben folgende Integralgleichung für $\phi$ gefunden:
       $$
      \phi(\Ortsr[v]) = \frac{1}{4\pi\varepsilon_0} \int_V
   \frac{\laddichte{V}(\Ortsr[vs])}{|\Ortsr[v]-\Ortsr[vs]|}
   \upd^3\Ortsr[s] + \frac{1}{4\pi} \oint_{O(V)} \left[\frac{1}{|\Ortsr[v] - \Ortsr[vs]|}\frac{\partial\phi}{\partial n'} - \phi\frac{\partial}{\partial n'}\frac{1}{|\Ortsr[v] - \Ortsr[vs]|} \right] \upd^2r'
       $$
       \item Nur Ladungen \alert{im} Volumen $V$ sowie $\phi$
         bzw. $\frac{\partial\phi}{\partial n}=\gradient\phi\cdot
         \vec{n}=-\EFeld[v]\cdot\vec{n}$ bestimmen das Skalarpotential
         $\phi$ in $V$. Ladungen außerhalb von $V$ wirken nur indirekt.
         \item Spezialfall: Wenn keine Ladungen in $V$ vorhanden sind,
           ist das Skarpotential vollständig durch die Randwerte
           festgelegt.
           \item Liegen die Ränder von $V$ im Unendlichen ($V$ ist der
             ganze Raum), verschwindet das Oberflächenintegral, weil
             der Integrand mit $1/r'^3$ abfällt. Es ergibt sich das
             bereits bekannte Ergebnis für das Skalarpotential im
             freien Raum.
             \item Wir werden als nächstes zeigen, dass die Lösung
               (für das E-Feld) bereits eindeutig bestimmt ist, wenn
               für Punkte $\in O(V)$
               \begin{itemize}
                 \item überall $\phi$ gegeben ist: \alert{Dirichlet
                     Randbedingungen}
                   \item oder überall $\frac{\partial\phi}{\partial n}=\gradient\phi\cdot
         \vec{n}=-\EFeld[v]\cdot\vec{n}$ gegeben ist: \alert{Neumann
           Randbedingungen}
         \item oder für einen Teil der $O_1(V)\subseteq O(V)$ Dirichlet
           Randbedingungen und für einen anderen Teil $O_2(V)\subseteq
           O(V),\; O_1(V) \mathbin{\dot{\cup}} O_2(V) = O(V)$ Neumannsche Randbedingungen gegeben sind:
           \alert{Gemischte Randbedingungen}   
                 \end{itemize}
         \end{itemize}

       \end{frame}

       \begin{frame}
         \frametitle{Eindeutigkeit}
         \begin{itemize}[<+->]
           \item Seien $\phi_1$ und $\phi_2$ Lösungen der Poisson-Gleichung
             $\laplace\phi_{1,2}(\Ortsr[v]) =
             -\frac{1}{\varepsilon_0}\laddichte{V}(\Ortsr[v])$
             mit gleichen Werten oder gleichen Normalenableitungen auf
             $O(V)$.
             \item Dann ist $\psi=\phi_1-\phi_2$ Lösung der
               Laplace-Gleichung $\laplace\psi(\Ortsr[v]) = 0$
               mit verschwindenden Werten bzw. verschwindender
               Normalenableitung auf $O(V)$.
               \item Wir wenden die \alert{1. Greenschen Identität} für
                 den Fall $f=g=\psi$ an:
                 $$
                 \only<3>{\int_V \left[ \psi\laplace \psi + (\gradient
                     \psi\cdot \gradient \psi) \right] \upd^3r' =
                   \oint_{O(V)} \psi\frac{\partial \psi}{\partial n}
                   \upd^2r'}
                 \only<4->{\int_V (\gradient
                     \psi)^2  \upd^3r' = 0}\only<5->{\Rightarrow \gradient\psi =
                     \vec{0}}\only<6->{\Rightarrow \psi=\text{const.}}
                   $$
                   \item<7-> \alert{Dirichlet}: $\psi=0$ auf dem
                     Rand. $\Rightarrow \psi = 0$ in $V$. $\to
                     \alert{\phi_1=\phi_2}\to
                     \alert{\EFeld[v]_1=\EFeld[v]_2}$
                     \item<8-> \alert{Neumann}:
                       $\frac{\partial\psi}{\partial n}=0$ auf dem
                     Rand. $\Rightarrow \psi = \text{const.}$ in $V$. $\to
                     \alert{\phi_1=\phi_2 +C} \to
                     \alert{\EFeld[v]_1=\EFeld[v]_2}$
                     \item<9-> Durch Angabe von $\phi$ \alert{UND}
                       $\frac{\partial\phi}{\partial n}$ an irgendeinem
                       Punkt von $O(V)$ wäre das Problem bereits
                       \alert{physikalisch überbestimmt}!
                     \item<10-> Die Gleichung
                       \vspace*{-1em}$$
                       \phi(\Ortsr[v]) = \frac{1}{4\pi\varepsilon_0} \int_V
   \frac{\laddichte{V}(\Ortsr[vs])}{|\Ortsr[v]-\Ortsr[vs]|}
   \upd^3\Ortsr[s] + \frac{1}{4\pi} \oint_{O(V)} \left[\frac{1}{|\Ortsr[v] - \Ortsr[vs]|}\frac{\partial\phi}{\partial n'} - \phi\frac{\partial}{\partial n'}\frac{1}{|\Ortsr[v] - \Ortsr[vs]|} \right] \upd^2r'
   $$
   ist somit noch nicht die Lösung, sondern \alert{\enquote{nur}} äquivalent zur Poisson-Gleichung.
           \end{itemize}
         \end{frame}
        
\input{finalframe.inc}
   
\end{document}