\input{head.inc}
  
% Präambelbefehle für die Präsentation
\title[TET: Beliebig veränderliche Felder - Impulserhaltung]{Beliebig veränderliche Felder - Impulserhaltung}

\begin{document}
% 
% Frontmatter 
% 
%%%%%%%%%%%%%%%%%%%%%%%%%%%%%%%%%%%%%%%%%%%%%%%%%%%%%%%%%%%%%%%%%%%%%%%%%%%%%%%%%%%%%%%%%%%%%%%%%%%%%%%%%%%%%%%%%%%%%%%%%%%%% 

%% inserts the title page and the table of contents
\maketitle

% 
% Content 
% 
%%%%%%%%%%%%%%%%%%%%%%%%%%%%%%%%%%%%%%%%%%%%%%%%%%%%%%%%%%%%%%%%%%%%%%%%%%%%%%%%%%%%%%%%%%%%%%%%%%%%%%%%%%%%%%%%%%%%%%%%%%%%% 
\section{Beliebig veränderliche Felder - Impulserhaltung}

\begin{frame}
  \frametitle{Impulserhaltung}
  \begin{itemize}[<+->]
  \item Warum betrachten wir den Impuls?
  \item Mechanik \(\to\) Beziehung zwischen \alert{Impuls} \(\vec{p}^\text{mech}\) und \alert{Kraft} \(\kraft[v]\):
    \begin{equation*}
      \kraft[v] = \frac{\upd}{\upd t} \vec{p}^\text{mech}
    \end{equation*}
  \item Diese Beziehung gilt auch für den \alert{relativistischen Impuls} (bei sehr hohen Geschwindigkeiten, im Rahmen der Speziellen Relativitätstheorie) und ist somit wesentlich fundamentaler als die klassische Formulierung \(\kraft[v] = m\vec{a}\); \(\vec{p}^\text{mech} = m \vec{v}\).
  \item Und wir wissen schon, dass die elektromagnetische Feldtheorie eine relativistische Theorie ist \(\to\) Quasistationäre Felder VIII - Induktion.
  \item In der elektromagnetischen Feldtheorie kann prinzipiell die \alert{Lorenzkraft} herangezogen werden:
    \begin{equation*}
      \kraft[v] = q (\EFeld[v] + \vec{v} \times \BFeld[v])
    \end{equation*}
  \item Für viele konkrete Berechnungen (z.B. elektrische Antriebe) ist eine \alert{Formulierung nur mit Feldern} häufig zweckmäßiger
  \item Wir starten hierfür mit der lokalen \alert{Kraftdichte} \(\kraftdichte[v]\) in Abhängigkeit von der \alert{Impulsdichte} \(\impulsdichte\)
    \begin{equation*}
      \kraftdichte[v] = \frac{\upd}{\upd t} \impulsdichte \quad \kraft[v]=\iiint_V \kraftdichte[v]\, \upd V
      \end{equation*}
  \end{itemize}
\end{frame}


\begin{frame}
  \frametitle{Kraftdichte}
  \begin{itemize}[<+->]
  \item Wir betrachten die Kraftdichte
    \begin{equation*}
      \kraftdichte[v] = \frac{\upd}{\upd t}\impulsdichte =\laddichte{V} (\EFeld[v] + \vec{v} \times \BFeld[v]) = \textcolor{green}{\laddichte{V}} \EFeld[v] + \textcolor{red}{\StromDichte[v]} \times \BFeld[v]
    \end{equation*}
  \item Durch Einsetzen der Maxwell-Gleichungen folgt:
    \begin{align*}
      \kraftdichte[v] &=  \EFeld[v]\, \textcolor{green}{\divergenz\DFeld[v]} + \textcolor{red}{(\rotation\HFeld[v]-\frac{\partial \DFeld[v]}{\partial t})} \times \BFeld[v]\\
                      &= \EFeld[v]\, \divergenz\DFeld[v] + (\rotation\HFeld[v]) \times \BFeld[v] - \frac{\partial \DFeld[v]}{\partial t}  \times \BFeld[v]\\
                      &= \EFeld[v]\, \divergenz\DFeld[v] + (\rotation\HFeld[v]) \times \BFeld[v] - \left[ \frac{\upd }{\upd t}(\DFeld[v] \times \BFeld[v]) \textcolor{orange}{-} \DFeld[v] \times \textcolor{orange}{\frac{\partial \BFeld[v]}{\partial t}} \right]\\
                      &= \EFeld[v]\, \divergenz\DFeld[v] + (\rotation\HFeld[v]) \times \BFeld[v] - \left[ \frac{\upd }{\upd t}(\DFeld[v] \times \BFeld[v]) \textcolor{orange}{+} \DFeld[v] \times \textcolor{orange}{\rotation \EFeld[v]} \right]\\
      \Aboxed{\kraftdichte[v] &= \left[ \EFeld[v]\, \divergenz\DFeld[v] - \DFeld[v] \times \rotation \EFeld[v]\right] + \left[-\BFeld[v]\times \rotation\HFeld[v])\right] - \frac{\upd }{\upd t}(\DFeld[v] \times \BFeld[v]) } \text{ alle Materialien}\\\pause
       &= \left[ \EFeld[v]\, \divergenz\DFeld[v] - \DFeld[v] \times \rotation \EFeld[v]\right] + \left[\HFeld[v]\,\textcolor{cyan}{\divergenz\BFeld[v]}-\BFeld[v]\times \rotation\HFeld[v]\right] - \frac{\upd }{\upd t}(\DFeld[v] \times \BFeld[v]) 
      \end{align*}
  \end{itemize}
\end{frame}

\begin{frame}
  \frametitle{Kraftdichte - homogene, lineare, isotrope Medien}
  \begin{itemize}[<+->]
  \item Wir hatten die Kraftdichte
    \begin{align*}
      \kraftdichte[v] &= \left[ \EFeld[v]\, \divergenz\DFeld[v] - \DFeld[v] \times \rotation \EFeld[v]\right] + \left[\HFeld[v]\,\divergenz\BFeld[v]-\BFeld[v]\times \rotation\HFeld[v]\right] - \frac{\upd }{\upd t}(\DFeld[v] \times \BFeld[v])  \text{ alle Materialien}\\\pause
      \Aboxed{\kraftdichte[v] &= \varepsilon\left[ \EFeld[v]\, \divergenz\EFeld[v] - \EFeld[v] \times \rotation \EFeld[v]\right] + \frac{1}{\mu}\left[\BFeld[v]\,\divergenz\BFeld[v]-\BFeld[v]\times \rotation\BFeld[v]\right] - \varepsilon\frac{\upd }{\upd t}(\EFeld[v] \times \BFeld[v])}  \text{ h, l, i Materialien}
    \end{align*}
  \item Terme \(\vec{a}\times\rotation\vec{a} \to\) kleiner mathematischer Einschub:
    \begin{align*}
      \gradient(\vec{a}\cdot\vec{b}) &= \vec{a}\times\rotation\vec{b} + \vec{b}\times\rotation\vec{a} + (\vec{a}\cdot\gradient)\vec{b} + (\vec{b}\cdot\gradient)\vec{a}\\
      \gradient(\underbrace{\vec{a}\cdot\vec{a}}_{|\vec{a}|^2}) &= 2 (\vec{a}\times\rotation\vec{a}) + 2 (\vec{a}\cdot\gradient)\vec{a}\\
      \Rightarrow - \EFeld[v] \times \rotation \EFeld[v] &= (\EFeld[v]\cdot\gradient)\EFeld[v] - \frac{1}{2}\gradient|\EFeld[v]|^2\quad \text{ für }\BFeld[v]\text{ analog}
    \end{align*}
  \item Damit ist die \alert{Kraftdichte in homogenen, lineare, isotropen Medien}:
    \begin{equation*}
        \boxed{%
      \begin{split}
        \kraftdichte[v] =  \varepsilon\left[ \EFeld[v]\, \divergenz\EFeld[v] +(\EFeld[v]\cdot\gradient)\EFeld[v]\right] + \frac{1}{\mu}\left[\BFeld[v]\,\divergenz\BFeld[v] + (\BFeld[v]\cdot\gradient)\BFeld[v]\right] \\-\gradient\left[ \only<4>{\frac{1}{2}\varepsilon|\EFeld[v]|^2 +\frac{1}{2\mu} |\BFeld[v]|^2}\only<5-|handout:0>{w_\text{em}}   \right] - \only<4,5>{\varepsilon\frac{\upd }{\upd t}(\EFeld[v] \times \BFeld[v])}\only<6-|handout:0>{\varepsilon\mu\frac{\upd }{\upd t}\PoyntingVektor[v]}
        \end{split}}
      \end{equation*}
  \end{itemize}
\end{frame}

\begin{frame}
  \frametitle{Maxwellscher Spannungstensor}
  \begin{itemize}[<+->]
  \item Um die Kraftdichte einfacher schreiben zu können, führen wir den \alert{Maxwellschen Spannungstensor} \(\mathbf{T}\) ein. Der Maxwellsche Spannungstensor ist ein dreidimensionaler Tensor 2. Stufe (\enquote{\(3 \times 3\)-Matrix}):
    \begin{equation*}
      \mathbf{T} = \left(T_{ij}\right) \text{ mit } \boxed{T_{ij} = \varepsilon\left[ \EFeld_i\EFeld_j-\frac{1}{2}\delta_{ij} |\EFeld[v]|^2\right] + \frac{1}{\mu} \left[\BFeld_i\BFeld_j-\frac{1}{2}\delta_{ij} |\BFeld[v]|^2\right] } \quad [T_{ij}] = \si{\newton\per\metre\squared} 
    \end{equation*}
  \item Explizit:
    \begin{equation*}
      \begin{split}
      \mathbf{T} =
      \begin{pmatrix}
        \varepsilon\EFeld_x\EFeld_x + \frac{1}{\mu}\BFeld_x\BFeld_x & \varepsilon\EFeld_x\EFeld_y + \frac{1}{\mu}\BFeld_x\BFeld_y &\varepsilon\EFeld_x\EFeld_z + \frac{1}{\mu}\BFeld_x\BFeld_z \\
        \varepsilon\EFeld_y\EFeld_x + \frac{1}{\mu}\BFeld_y\BFeld_x & \varepsilon\EFeld_y\EFeld_y + \frac{1}{\mu}\BFeld_y\BFeld_y &\varepsilon\EFeld_y\EFeld_z + \frac{1}{\mu}\BFeld_y\BFeld_z \\
        \varepsilon\EFeld_z\EFeld_x + \frac{1}{\mu}\BFeld_z\BFeld_x & \varepsilon\EFeld_z\EFeld_y + \frac{1}{\mu}\BFeld_z\BFeld_y &\varepsilon\EFeld_z\EFeld_z + \frac{1}{\mu}\BFeld_z\BFeld_z 
      \end{pmatrix}
      \\ -\underbrace{\frac{1}{2}\left(\varepsilon |\EFeld[v]|^2 + \frac{1}{\mu}|\BFeld[v]|^2\right)}_{w_\text{em}}
      \begin{pmatrix}
        1 & 0 &0\\
        0 & 1 & 0\\
        0& 0& 1
      \end{pmatrix}
      \end{split}
      \end{equation*}
  \end{itemize}
\end{frame}

\begin{frame}
  \frametitle{Maxwellscher Spannungstensor und Kraftdichte}
  \begin{itemize}[<+->]
  \item Wir bilden die \(j\)-te Komponente der Divergenz von \(\mathbf{T}\):
    \begin{align*}
      (\divergenz\mathbf{T})_j &= \sum_{i=1}^3 \frac{\partial}{\partial x_i} T_{ij} = \partial_i T_{ij} \\
                              &= \sum_{i=1}^3 \left\{ \varepsilon\left( \frac{\partial \EFeld_i}{\partial x_i} \EFeld_j + \EFeld_i\frac{\partial \EFeld_j}{\partial x_i} \right) +
\frac{1}{\mu}\left( \frac{\partial \BFeld_i}{\partial x_i} \BFeld_j + \BFeld_i\frac{\partial \BFeld_j}{\partial x_i} \right)
                                \right\} - \left(\frac{\varepsilon}{2}\frac{\partial |\EFeld[v]|^2}{\partial x_j} + \frac{1}{2\mu}\frac{\partial |\BFeld[v]|^2}{\partial x_j}\right)\\
                              &=\varepsilon\left[\EFeld_j \divergenz\EFeld[v] + (\EFeld[v]\cdot\gradient )\EFeld_j - \frac{1}{2}\gradient_j |\EFeld[v]|^2 \right] +\\
                               &\qquad \frac{1}{\mu}\left[\BFeld_j \divergenz\BFeld[v] + (\BFeld[v]\cdot\gradient )\BFeld_j - \frac{1}{2}\gradient_j |\BFeld[v]|^2 \right]
    \end{align*}
  \item Dies ist gerade die \(j\)-te Komponente der ersten drei Terme der Kraftdichte. Somit:
    \begin{equation*}
      \boxed{\kraftdichte[v] = \divergenz \mathbf{T} - \varepsilon\frac{\upd}{\upd t}\left( \EFeld[v] \times \BFeld[v]\right) 
                   = \divergenz \mathbf{T} - \varepsilon\mu\frac{\upd}{\upd t}\left( \EFeld[v] \times \HFeld[v]\right) 
      = \divergenz \mathbf{T} - \varepsilon\mu\frac{\upd \PoyntingVektor[v]}{\upd t}}
  \end{equation*}
\item Die Gesamtkraft ist somit:
  \begin{equation*}
  \boxed{\kraft[v] = \oiint_{O(V)} \mathbf{T} \cdot \upd\vec{A} - \varepsilon\mu\frac{\upd}{\upd t} \iiint_V \PoyntingVektor[v] \upd V}
\end{equation*}
\end{itemize}
\end{frame}

\begin{frame}
  \frametitle{Interpretation}
  \begin{equation*}
  \boxed{\kraft[v] = \oiint_{O(V)} \mathbf{T} \cdot \upd\vec{A} - \varepsilon\mu\frac{\upd}{\upd t} \iiint_V \PoyntingVektor[v] \upd V}
\end{equation*}
\begin{columns}
  \begin{column}<+->[t]{.5\linewidth}
    Oberflächenintegral
    \begin{itemize}[<+->]
    \item Wirkt nur auf die Oberfläche des betrachteten Volumens
      \item \(T_{ij}\): $i$-te Komponente (\alert{Wirkrichtung}) der Kraft pro Flächeneinheit (Druck) auf Flächenelement mit Normale \(\vu{j}\) (\alert{Normalenrichtung}) 
      \item \(i=j\): Die Komponenten \(T_{ii}\) sind die \alert{Normalspannungskomponenten}, bei denen Normalen- und Wirkrichtung gleich sind.
        \item \( i \ne j \): Die anderen Komponenten sind \alert{Scherspannungskomponenten}, bei den Normalen- und Wirkrichtung unterschiedlich sind.
      \end{itemize}
    \end{column}
    \begin{column}<+->[t]{.5\linewidth}
      Volumenintegral
    \begin{itemize}[<+->]
    \item Verschwindet im stationären Fall \(\to\) dann Gesamtkraft nur über Oberfläche!
    \item Mechanik: \(\kraft[v] = \frac{\upd }{\upd t}\vec{p}^\text{mech} \), \(\kraftdichte[v] = \frac{\upd }{\upd t}\vec{p}^\text{mech}_V \)
    \item \alert{Elektromagnetischer Impuls bzw. Impulsdichte}
      \begin{align*}
        \Aboxed{\vec{p}^\text{em} &= \iiint_V \varepsilon\mu\PoyntingVektor[v] \upd V}\\
        \Aboxed{\vec{p}^\text{em}_V &= \varepsilon\mu\PoyntingVektor[v]} 
      \end{align*}
      \item Wieder Poynting-Vektors! \(\to\) SRT
    \end{itemize}
    \end{column}
\end{columns}
\end{frame}

\begin{frame}
  \frametitle{Impulserhaltung}
    \begin{itemize}[<+->]
    \item Die Formel für die Kraftdichte kann nun umgeformt werden:
      \begin{align*}
        \kraftdichte[v] = \frac{\upd }{\upd t}\vec{p}^\text{mech}_V &= \divergenz\mathbf{T} - \frac{\upd }{\upd t}\left( \varepsilon\mu\PoyntingVektor[v]\right) = \divergenz\mathbf{T} - \frac{\upd }{\upd t}\vec{p}^\text{em}_V\\
        \frac{\upd }{\upd t}\left(\vec{p}^\text{mech}_V +  \varepsilon\mu\PoyntingVektor[v]\right)&=\divergenz\mathbf{T} \\
        \Aboxed{\frac{\upd }{\upd t}\left(\vec{p}^\text{mech}_V + \vec{p}^\text{em}_V\right)&=\divergenz\mathbf{T}} \text{ lokale Impulsbilanz} 
      \end{align*}
    \item Volumenintegration liefert die integrale \alert{Impulsbilanz}:
      \begin{equation*}
        \boxed{\frac{\upd }{\upd t} \iiint_V \left(\vec{p}^\text{mech}_V + \vec{p}^\text{em}_V\right)\upd V = \oiint_{O(V)}\mathbf{T} \cdot \upd \vec{A}} \text{ integrale Impulsbilanz} 
      \end{equation*}
    \item \alert{Strahlungsdruck}: Kepler: Komentenschweife; Maxwell: Normalkomponente und volumetrische Energiedichte; Wellenmodell: Spannungstensor; Teilchenmodell: Planck, Photoelektrischer Effekt, Energie-Impuls-Relation
      \item Anwendungen: Sonnensegel als Antrieb; zunehmend: Mikromanipulation mit Licht 
    \end{itemize}
\end{frame}



\input{finalframe.inc}
   
\end{document}