\input{head.inc}
\usepackage{pdfpages}
\usepackage{fancyhdr}
\usepackage{xcolor}

\fancypagestyle{importedpages}{%
  \fancyhf{}% Clear header/footer
  \renewcommand{\headrulewidth}{0pt}% Remove header rule
  \renewcommand{\footrulewidth}{0pt}% Remove footer rule (default)
  \fancyfoot[R]{\raisebox{.75\baselineskip}[0pt][0pt]{{\color{gray}\fontsize{6}{7} \selectfont Seite \thepage}}}% Lower page number into position
}

\begin{document}
\fancyfootoffset{-1.26cm}
\includepdfset{pages=-,fitpaper,pagecommand={\thispagestyle{importedpages}}}
\includepdf{Titel.handout.pdf}
% TET-1
%\tableofcontents
\section{Begriffsbestimmung}
\includepdf{Begriffsbestimmung.handout.pdf}
\section{Vorwissen}
\includepdf{Vorwissen.handout.pdf}
\section{Axiomatische Grundlagen}
\includepdf{Axiomatische_Grundlagen.handout.pdf}
\section{Einführung}
\includepdf{Einfuehrung.handout.pdf}
\section{Verhalten an Grenzflächen}
\includepdf{Verhalten_an_Grenzflaechen.handout.pdf}
\section{Einteilung elektromagnetischer Felder}
\includepdf{Einteilung_elektromagnetischer_Felder.handout.pdf}
\section{Elektrostatik}
\subsection{Grundgleichungen, Größen und Begriffe}
\includepdf{Elektrostatik-I.handout.pdf}
\subsection{Monopol bis Multipolentwicklung}
\includepdf{Elektrostatik-II.handout.pdf}
\subsection{Randwertprobleme}
\includepdf{Elektrostatik-III.handout.pdf}
\subsection{Formale Lösung}
\includepdf{Elektrostatik-IV-Formale_Loesung.handout.pdf}
\subsection{Spiegelungsmethode}
\includepdf{Elektrostatik-V-Spiegelungsmethode.handout.pdf}
\subsection{Orthogonale Funktionensysteme}
\includepdf{Elektrostatik-VI-Orthogonale_Funktionensysteme.handout.pdf}
\subsection{Separationsverfahren}
\includepdf{Elektrostatik-VII-Separationsverfahren.handout.pdf}
\subsection{Materie}
\includepdf{Elektrostatik-VIII-Materie.handout.pdf}
\section{Stationäres Elektrisches Strömungsfeld}
\includepdf{Stationaeres-Stroemungsfeld.handout.pdf}
\section{Magnetostatik}
\subsection{Grundlagen}
\includepdf{Magnetostatik-I_Grundlagen.handout.pdf}
\subsection{Materie}
\includepdf{Magnetostatik-II_Materie.handout.pdf}
\section{Quasistationäre Felder}
\subsection{Grundlagen}
\includepdf{Quasistationaere_Felder-Grundlagen.handout.pdf}
\subsection{Felddiffusion}
\includepdf{Quasistationaere_Felder-Diffusion.handout.pdf}
\subsection{Felddiffusion im Halbraum}
\includepdf{Quasistationaere_Felder-III-Felddiffusion_im_Halbraum.handout.pdf}
\subsection{Induktion}
\includepdf{Quasistationaere_Felder-IV-Induktion.handout.pdf}
% TET-2
\section{Beliebig veränderliche Felder}
\subsection{Ladungserhaltung und Energieerhaltung}
\includepdf{Beliebig_veraenderliche_Felder-Energieerhaltung.handout.pdf}
\subsection{Impulserhaltung}
\includepdf{Beliebig_veraenderliche_Felder-Impulserhaltung.handout.pdf}
\section{Elektromagnetische Wellen}
\subsection{Grundlagen}
\includepdf{Elektromagnetische_Wellen-I-Grundlagen.handout.pdf}
\subsection{Homogene Wellengleichung}
\includepdf{Elektromagnetische_Wellen-II-Homogene_Wellengleichung.handout.pdf}
\subsection{Harmonische Ebene Wellen}
\includepdf{Elektromagnetische_Wellen-III-Harmonische_Ebene_Wellen.handout.pdf}
\subsection{Polarisation Ebener Wellen}
\includepdf{Elektromagnetische_Wellen-IV-Polarisation_Ebener_Wellen.handout.pdf}
\subsection{Wellenpakete}
\includepdf{Elektromagnetische_Wellen-V-Wellenpakete.handout.pdf}
\subsection{Kugelwellen}
\includepdf{Elektromagnetische_Wellen-VI-Kugelwellen.handout.pdf}
\subsection{Allgemeine Lösung}
\includepdf{Elektromagnetische_Wellen-VII-Allgemeine_Loesung.handout.pdf}
\subsection{Energie und Impuls}
\includepdf{Elektromagnetische_Wellen-VIII-Energie_und_Impuls.handout.pdf}
\subsection{Leitende Medien}
\includepdf{Elektromagnetische_Wellen-IX-Leitende_Medien.handout.pdf}
\subsection{Reflexion und Brechung}
\includepdf{Elektromagnetische_Wellen-X-Reflexion_Brechung.handout.pdf}
\subsection{Erzeugung elektromagnetischer Wellen}
\includepdf{Elektromagnetische_Wellen-XI-Erzeugung.handout.pdf}
\subsection{Hertzscher Diplol}
\includepdf{Antennen-I-Hertzscher_Diplol.handout.pdf}
\subsection{Zylindrische Wellenleiter}
\includepdf{Wellenleiter-I-Zylindrische_Wellenleiter.handout.pdf}
\subsection{Klassische Leitungstheorie}
\includepdf{Wellenleiter-II-Klassische_Leitungstheorie.handout.pdf}
\input{finalframe-handout.inc}
\end{document}
