\input{head.inc}

% Präambelbefehle für die Präsentation
\title[TET: Elektrostatik-VIII: Materie]{Elektrostatik-VIII: Materie}

\begin{document}
% 
% Frontmatter 
% 
%%%%%%%%%%%%%%%%%%%%%%%%%%%%%%%%%%%%%%%%%%%%%%%%%%%%%%%%%%%%%%%%%%%%%%%%%%%%%%%%%%%%%%%%%%%%%%%%%%%%%%%%%%%%%%%%%%%%%%%%%%%%% 

%% inserts the title page and the table of contents
\maketitle

% 
% Content 
% 
%%%%%%%%%%%%%%%%%%%%%%%%%%%%%%%%%%%%%%%%%%%%%%%%%%%%%%%%%%%%%%%%%%%%%%%%%%%%%%%%%%%%%%%%%%%%%%%%%%%%%%%%%%%%%%%%%%%%%%%%%%%%% 
\section{Elektrostatik-VIII: Materie}

\begin{frame}

  \frametitle{mikroskopische vs makroskopische Betrachtung}
  \begin{itemize}[<+->]
  \item Die Maxwellgleichungen für das Vakuum gelten zunächst auch \alert{unverändert} in Materie
  \item Denn: Materie ist auch nur \enquote{Nichts} plus \enquote{Elementarteilchen} (Protonen, Neutronen, Elektronen).
  \item Beispiel
      \begin{itemize}[<+->]
      \item Das dichteste Element ist \alert{Osmium} (Os): OZ: 76, Dichte: $22.6\; \frac{\text{g}}{\text{cm}^3}$, Atommasse: $190.23$ u
      \item $8.4\; \text{cm}^3$ entspricht 1 Mol Osmium
      \item Mit den Radien von Elektron ($\approx 10^{-19}$\;m), Proton und Neutron (jeweils $\approx 10^{-15}$\;m) rechnet man aus, dass 1 Mol Osmium etwa $0.5\cdot 10^{-12}\;\text{cm}^3$ Materie enthält.
        \item Oder anders ausgedrückt: In $16.8\;(\text{hm})^3$ ist $1\;\text{cm}^3$ Materie
        \end{itemize}
      \item Man könnte mit den Maxwell-Gleichungen des Vakuums rechnen, müsste dann aber über alle Ladungen superponieren und dabei die Orte und Ortsveränderung aller Ladungen immer berücksichtigen.
      \item Die Informationen hierzu sind erstens gar nicht einfach verfügbar und zweitens sind die dann berechenbaren Details gar nicht beobachtbar
        \item Deshalb: Übergang zu \alert{makroskopischen Betrachtungen}
  \end{itemize}
  
  \end{frame}

\begin{frame}

  \frametitle{Ladungsdichte und Polarisation von \enquote{Teilchen}}
  \begin{itemize}[<+->]
  \item Größere Ladungsansammlungen (Atome, Ionen, Moleküle, ...) können als Ladungsanhäufungen (\enquote{Teilchen}) betrachtet werden, zwischen denen fast nichts ist.
  \item Befinden sich im $k$-ten \enquote{Teilchen} die Ladungen  $q_i^k$ (gebunden oder frei) an den Orten $r_i = r_i(t)$, so ist die \alert{Ladungsdichte im Bereich des $k$-ten Teilchens} in guter Näherung
    $$
    \rho^k(\Ortsr[v], t) = \sum_i q_i^k \delta^3(\Ortsr[v] - \Ortsr[v]_i(t) ) \quad \stackrel{\text{Mittelung}}{\longrightarrow} \quad \boxed{\rho^k(\Ortsr[v]) = \sum_i q_i^k \delta^3(\Ortsr[v] - \Ortsr[v]_i )} 
    $$
  \item Ist $\vec{R}^k=\vec{R}^k(t)$ der \alert{Ladungsschwerpunkt} des $k$-ten Teilchens, so ergibt sich das \alert{Dipolmoment} zu:
    $$
    \vec{p}^k(t) = \int_{V^k} \rho^k(\Ortsr[v], t) (\Ortsr[v] - \vec{R}^k(t) ) \upd^3r \quad \stackrel{\text{Mittelung}}{\longrightarrow} \quad \boxed{\vec{p}^k = \int_{V^k} \rho^k(\Ortsr[v]) (\Ortsr[v] - \vec{R}^k ) \upd^3r}
    $$
  \end{itemize}
  
  \end{frame}

\begin{frame}

  \frametitle{Effektive Ladungsdichte, Dipolmoment und Potential}
  \begin{itemize}[<+->]
  \item Abstände innerhalb der \enquote{Teilchen} sind in aller Regel sehr klein im Vergleich zum Abstand zwischen Ladungsschwerpunkt und Beobachtungspunkt: $|\Ortsr[v]^k -  \vec{R}^k| \ll  |\Ortsr[v] - \vec{R}^k | \text{ für } \Ortsr[v]^k \in V^k $
  \item \alert{Skalarpotential des $k$-ten Teilchens}: Multipolentwicklung nach Dipolterm abbrechen:
    $$
    \phi^k(\Ortsr[v]) \approx \frac{1}{4\pi\varepsilon_0} \left[ \frac{q^k}{|\Ortsr[v] - \vec{R}^k |} + \frac{\vec{p}^k \cdot (\Ortsr[v] - \vec{R}^k )}{|\Ortsr[v] - \vec{R}^k |^3} \right] 
    $$
  \item Für $N$ \enquote{Teilchen} ergeben sich dann \alert{effektive} Ladungsdichte, Dipoldichte und Potential in  der Form
    \begin{align*}
      \rho_{\text{eff}}(\Ortsr[v]) & = \sum_{k=1}^N q^k\delta (\Ortsr[v] - \vec{R}^k ) \quad\text{mit Gesamtladung } q^k\\
      \vec{p}_{\text{eff}}(\Ortsr[v]) &= \sum_{k=1}^N \vec{p}^k\delta (\Ortsr[v] - \vec{R}^k ) \quad\text{mit Gesamtdipolmoment } \vec{p}^k\\
      \phi _{\text{eff}}(\Ortsr[v]) &= \frac{1}{4\pi\varepsilon_0} \int_V \left[ \frac{\rho_{\text{eff}}(\Ortsr[vs])}{|\Ortsr[v] - \Ortsr[vs] |} + \frac{\vec{p}_{\text{eff}}(\Ortsr[vs])  \cdot (\Ortsr[v] - \Ortsr[vs] )}{|\Ortsr[v] - \Ortsr[vs] |^3} \right] \upd^3 r'  
      \end{align*}
 \end{itemize}
  
  \end{frame}

\begin{frame}

  \frametitle{Makroskopische Ladungsdichte, Polarisation und Potential}
  \begin{itemize}[<+->]
  \item Aus den effektiven Größen ergeben sich die \alert{makroskopischen} Größen durch Mittelung.
  \item \alert{Makroskopische Ladungsdichte} $\laddichte{V}(\Ortsr[v]) = \overline{\rho_{\text{eff}}(\Ortsr[v])} $ $\to$ alle Ladungen berücksichtigt, aber die meisten kompensieren sich in ihrer Wirkung
  \item \alert{Makroskopische Polarisation} $\vec{P}(\Ortsr[v]) = \overline{\vec{p}_{\text{eff}}(\Ortsr[v])} $ $\to$ \alert{Modelle} werden benötigt, um die makroskopische Polarisation als Antwort auf interne und externe Felder vorherzusagen 
  \item Das \alert{makroskopische Skalarpotential} ist dann
    \begin{align*}
      \phi (\Ortsr[v]) = \overline{\phi _{\text{eff}}(\Ortsr[v])} &= \frac{1}{4\pi\varepsilon_0} \int_V \left[ \frac{\laddichte{V}(\Ortsr[vs])}{|\Ortsr[v] - \Ortsr[vs] |} + \frac{\vec{P}(\Ortsr[vs])  \cdot (\Ortsr[v] - \Ortsr[vs] )}{|\Ortsr[v] - \Ortsr[vs] |^3} \right] \upd^3 r' \\
      &= \frac{1}{4\pi\varepsilon_0} \int_V \left[ \frac{\laddichte{V}(\Ortsr[vs])}{|\Ortsr[v] - \Ortsr[vs] |} + \vec{P}(\Ortsr[vs])  \cdot \nabla' \frac{1}{|\Ortsr[v] - \Ortsr[vs] |} \right] \upd^3 r'
      \end{align*}
\end{itemize}
  
  \end{frame}

\begin{frame}
\frametitle{Dielektrische Verschiebung}
\begin{itemize}[<+->]
  \item Makroskopische Skalarpotential:
    $$
      \phi (\Ortsr[v]) = \frac{1}{4\pi\varepsilon_0} \int_V \left[ \frac{\laddichte{V}(\Ortsr[vs])}{|\Ortsr[v] - \Ortsr[vs] |} + \vec{P}(\Ortsr[vs])  \cdot \nabla' \frac{1}{|\Ortsr[v] - \Ortsr[vs] |} \right] \upd^3 r'
     $$
   \item  Wir berechnen $\divergenz \EFeld[v](\Ortsr[v]) = \divergenz(-\gradient\phi(\Ortsr[v])) = -\laplace_r \phi(\Ortsr[v])$:
     \begin{align*}
       \divergenz \EFeld[v](\Ortsr[v]) & = - \frac{1}{4\pi\varepsilon_0} \int_V [ \laddichte{V}(\Ortsr[vs]) \underbrace{\laplace_r\frac{1}{|\Ortsr[v] - \Ortsr[vs] |}}_{-4\pi\delta^3(\Ortsr[v] - \Ortsr[vs])} + \vec{P}(\Ortsr[vs])  \cdot \nabla'\underbrace{ \laplace_r\frac{1}{|\Ortsr[v] - \Ortsr[vs] |}}_{-4\pi\delta^3(\Ortsr[v] - \Ortsr[vs])} ] \upd^3 r'\\
                               &= \frac{1}{\varepsilon_0} [ \laddichte{V}(\Ortsr[v]) + \int_V \vec{P}(\Ortsr[vs])  \cdot \underbrace{\nabla'\delta^3(\Ortsr[v] - \Ortsr[vs])}_{-\nabla\delta^3(\Ortsr[v] - \Ortsr[vs])}  \upd^3 r' ] = \frac{1}{\varepsilon_0} [ \laddichte{V}(\Ortsr[v]) - \underbrace{\nabla\cdot \vec{P}(\Ortsr[v])}_{\divergenz\vec{P}(\Ortsr[v])} ] 
       \end{align*}
 \item Hieraus folgt das makroskopische Äquivalent des Colomb-Gauss-Gesetzes in Materie mit der \alert{Dielektrischen Verschiebung} $\vec{D}(\Ortsr[v])$:
$$                                
                                 \divergenz(\varepsilon_0 \EFeld[v](\Ortsr[v]) + \vec{P}(\Ortsr[v]) ) = \boxed{\divergenz \vec{D}(\Ortsr[v]) = \laddichte{V}(\Ortsr[v])} \text{ mit } \boxed{\vec{D}(\Ortsr[v]) = \varepsilon_0 \EFeld[v](\Ortsr[v]) + \vec{P}(\Ortsr[v])} 
                                 $$
                               \end{itemize}
\end{frame}

\begin{frame}
\frametitle{Maxwell Gleichungen und Stetigkeitsbedingungen}
\begin{itemize}[<+->]
  \item Grundgleichungen der Elektrostatik
\begin{align*}
	& \rotation \EFeld[v] = \vec{0}
		&&\oint\limits_{\rand(\Flaeche)} \EFeld[v]
                   \cdot \intweg[v] = 0\\
	& \divergenz \DFeld[v] = \laddichte{V}
		&&\oiint\limits_{\oberfl(\volumen)}
                   \DFeld[v]\cdot \intflaeche[v] =
                   \iiint\limits_{\volumen} \laddichte{V} \intvolumen
\end{align*}
\item Stetigkeitsbedingungen ($\vec{n}$ Normalenvektor von Volumen 1 $\to$ zeigt von 1 nach 2):
\begin{align*}
  \vec{n} \cdot (\DFeld[v]_2 -
              \DFeld[v]_1) & = \DFeld{}_2\normal -
                                    \DFeld{}_1\normal = \laddichte{F}\\
  \vec{t} \cdot (\EFeld[v]_2 - \EFeld[v]_1) &= \EFeld{}_2\tangential - \EFeld{}_1\tangential = 0 \quad \forall \vec{t} \text{ mit } \vec{t} \perp \vec{n}
  \end{align*}
 \end{itemize}
\end{frame}

\begin{frame}
\frametitle{Typen von Dielektrika}
\begin{itemize}[<+->]
\item \alert{gewöhnliche Dielektrika:}
  \begin{itemize}[<+->]
  \item Keine internen Dipolmomente ohne äußeres Feld
  \item Äußeres Feld verschiebt Ladungen in den \enquote{Teilchen} und erzeugt damit eine Polarisation
  \item $\to$ \alert{Deformationspolarisation}
  \end{itemize}
\item \alert{Paraelektrika:}
    \begin{itemize}[<+->]
    \item Es gibt permanente interne Dipole (z.B. Wasser) auch ohne äußeres Feld
    \item Durch die thermische Unordnung ergibt sich keine makroskopische Polarisation
    \item Ein äußeres Feld führt zu einer (temperaturabhängigen) makroskopischen Polarisation
      \item $\to$ \alert{Orientierungspolarisation}
      \end{itemize}
      \item \alert{Ferroektrika:}
    \begin{itemize}[<+->]
    \item Es gibt permanente interne Dipole auch ohne äußeres Feld
    \item Unterhalb einer kritischen Temperatur (Curie-Temperatur) richten sich die internen Dipole \alert{spontan} zueinander aus (z.B. Bariumtitanat)
    \item Oberhalb der kritischen Temperatur erscheinen sie paraelektrisch
    \item Im ferroelektrischen Bereich kann die Richtung der spontanen Polarisation durch ein äußeres Feld umgekehrt werden $\to$ \alert{Hysterese} von P vs E
      \item Ferroelektrische Stoffe zeigen \alert{pyroelektrische} und \alert{piezoelektrische} Effekte   
      \end{itemize}
\end{itemize}
\end{frame}

\begin{frame}
\frametitle{Gewöhnliche Dielektrika und Paraelektrika}
\begin{itemize}[<+->]
\item Die Polarisation hängt von E-Feld ab: $\vec{P} = \vec{P}(\vec{E})$ mit $\vec{P}(\vec{E} = \vec{0}) = \vec{0}$
\item Entwicklung nach Potenzen von E ($P_i$: i-te Komponente; Summenkonvention):
  $$
  P_i = \gamma_{ij} E_j + \beta_{ijk} E_j E_k + \ldots
  $$
\item Die Tensoren $\gamma_{ij}$, $\beta_{ijk}$, $\ldots$ sind \alert{Materialkonstanten}
\item \alert{lineares Dielektrikum}: $P_i = \gamma_{ij} E_j$ ist ausreichend gute Näherung
\item \alert{isotropes Dielektrikum}: $P_i = \gamma_{ij} E_j + \beta_{ijk} E_j E_k + \ldots = \gamma E_i +\beta E_i^2 +\ldots$ richtungsunabhängig
\item Für lineare und isotrope Dielektrika definiert man die \alert{elektrische Suszeptibilität} $\chi_e$:
  $$
\vec{P} = \chi_e\varepsilon_0 \vec{E}
$$
\item Somit folgt dann für die Dielektrische Verschiebung:
  $$
\boxed{\vec{D} = (1+\chi_e)\varepsilon_0 \vec{E} = \varepsilon_r\varepsilon_0 \vec{E} = \varepsilon \vec{E}} 
$$
\item Die Größe $\varepsilon_r = (1+\chi_e)$ ist die \alert{relative Dielekrizitätskonstante}.
\end{itemize}
\end{frame}

\begin{frame}
\frametitle{Atomare Polarisierbarkeit - Clausius-Mosotti-Gleichung}
\begin{itemize}[<+->]
\item Die lokale Antwort der Materie auf das tatsächliche lokale elektrische Feld kann nur sehr aufwendig ermittelt werden
\begin{itemize}[<+->]
\item Festkörperphysikalische Modelle
\item Quantenmechanische Rechnungen
  \item Numerische Berechnung (prinzipiell)
\end{itemize}
\item Modelle liefern dann die \alert{atomare/molekulare Polarisierbarkeit} $\alpha$.
\item Verknüpfung zwischen atomaren und makroskopischen Größen liefert die \alert{Clausius-Mosotti-Gleichung} (ohne Herleitung):
  $$
  \boxed{\alpha = \frac{3\varepsilon_0}{n} \frac{\varepsilon_r -1}{\varepsilon_r +2} } \text{ mit } n=\frac{N_A\rho}{M_m}
  $$
  wobei $N_A$ die \alert{Avogadrokonstante}, $\rho$ die \alert{Dichte} und $M_m$ die \alert{molare Masse} ist. 
\end{itemize}
\end{frame}


\begin{frame}
\frametitle{Elektrostatische Energie und Energiedichte}
\begin{itemize}[<+->]
\item Wir hatten die elektrostatische Energie bereits für das \alert{Vakuum} berechnet:
       \begin{align*}
      W_e = & -\frac{\varepsilon_0}{2}\left(\int \divergenz(\phi\gradient\phi)
      \upd^3 r - \int (\gradient\phi)^2 \upd^3 r\right)  =
                     \frac{\varepsilon_0}{2} \int (\gradient\phi)^2 \upd^3 r\\
        = & \frac{\varepsilon_0}{2} \int |\EFeld[v]|^2 \upd^3 r
            \quad\alert{Energiedichte:} \boxed{w_e = \frac{\varepsilon_0}{2} |\EFeld[v]|^2}
      \end{align*}
 \item Eine ganz analoge Herleitung ergibt das \alert{verallgemeinerte Resultat}:
     $$
      \boxed{W_e = \frac{1}{2}\int \vec{E}\cdot\vec{D} \upd^3 r}
            \quad\alert{Energiedichte:} \boxed{w_e = \frac{1}{2} \vec{E}\cdot\vec{D}}
      $$
 \end{itemize}
\end{frame}
  
\input{finalframe.inc}
   
\end{document}