\input{head.inc}
\usepackage{tikz-3dplot}
\usepackage[outline]{contour} % glow around text
\colorlet{veccol}{green!50!black}
\colorlet{projcol}{blue!70!black}
\colorlet{myblue}{blue!80!black}
\colorlet{myred}{red!90!black}
\colorlet{mydarkblue}{blue!50!black}
\tikzset{>=latex} % for LaTeX arrow head
\tikzstyle{proj}=[projcol!80,line width=0.08] %very thin
\tikzstyle{area}=[draw=veccol,fill=veccol!80,fill opacity=0.6]
\tikzstyle{vector}=[-stealth,myblue,thick,line cap=round]
\tikzstyle{unit vector}=[->,veccol,thick,line cap=round]
\tikzstyle{dark unit vector}=[unit vector,veccol!70!black]
\usetikzlibrary{angles,quotes} % for pic (angle labels)
\contourlength{1.3pt}

% Präambelbefehle für die Präsentation
\title[TET: Vorwissen]{Vorwissen}

\begin{document}
% 
% Frontmatter 
% 
%%%%%%%%%%%%%%%%%%%%%%%%%%%%%%%%%%%%%%%%%%%%%%%%%%%%%%%%%%%%%%%%%%%%%%%%%%%%%%%%%%%%%%%%%%%%%%%%%%%%%%%%%%%%%%%%%%%%%%%%%%%%% 

%% inserts the title page and the table of contents
\maketitle

% 
% Content 
% 
%%%%%%%%%%%%%%%%%%%%%%%%%%%%%%%%%%%%%%%%%%%%%%%%%%%%%%%%%%%%%%%%%%%%%%%%%%%%%%%%%%%%%%%%%%%%%%%%%%%%%%%%%%%%%%%%%%%%%%%%%%%%% 
\section{Vorwissen}

\begin{frame}
  \frametitle{Ausgangspunkt}

  \begin{block}{Maxwell-Gleichungen}
    Ausgangspunkt aller Betrachtungen sind die
    \alert{Maxwell-Gleichungen}:
    \begin{align*}
      \rotation \vec{E} + \frac{\partial
      \vec{B}}{\partial t} &= \vec{0} &
                                  \rotation\vec{H}-\frac{\partial\vec{D}}{\partial
                                  t}&=\vec{J} \\
      \divergenz\vec{B}&=0 & \divergenz\vec{D}&=\rho_\text{V}
    \end{align*} \pause
    Offenbar ist Wissen in folgenden Bereichen notwendig:
    \begin{itemize}[<+->]
    \item Felder (skalar und vektorwertig; reell und
      komplexwertig)
    \item Differentialoperatoren auf diesen
      Feldern
    \item partielle Differentialgleichungen
    \item beim Übergang von lokalen zu
      integralen Aussagen: Integralrechnung
      (Linien-, Flächen-, Volumenintegrale; in
      verschiedenen Koordinatensystemen
      (insb. kartesische, zylindrische und
      sphärische Koordinaten, Umrechnung
      dazwischen, Metrik, Integralsätze (Gauß
      und Stokes), Kronecker $\delta_{ij}$,
      Delta-Funktion $\delta$
    \end{itemize}

    
  \end{block}
\end{frame}

\section{Differentialoperatoren}
\begin{frame}
  \frametitle{Differentialoperatoren (I)}

  \begin{block}{Divergenz}
    Divergenz eines Vektorfeldes,
    \enquote{Quellendichte} $\to$ Skalarfeld\pause

    \begin{align*}
      \divergenz \vec{F} := \nabla \cdot \vec{F} =
      \sum_{i=1}^3 \frac{\partial}{\partial x_i} F_i
      = \partial_i F_i & \qquad \text{in kartesischen
                         Koordinaten}\\ \pause
      \divergenz \vec{F} := \lim_{V \to 0}\left(
      \frac{1}{V} \oint_{O(V)} \vec{F}\cdot \vec{n}
      dS\right) & \qquad\text{koordinatenfrei}
    \end{align*}
    
  \end{block}
\end{frame}
\begin{frame}
  \frametitle{Differentialoperatoren (II)}

  \begin{block}{Rotation}
    Rotation eines Vektorfeldes,
    \enquote{Wirbelstärke} $\to$ Vektorfeld\pause

    \begin{align*}
      \rotation \vec{F} := \nabla \times \vec{F} =
      \sum_{i=1}^3\vu{i}\sum_{j=1}^3\sum_{k=1}^3
      \epsilon_{ijk}\frac{\partial}{\partial x_j}
      F_k = \epsilon_{ijk} \vu{i} \partial_j F_k & \qquad \text{in kartesischen
                                                      Koordinaten}\\ \pause
      \rotation \vec{F} := \lim_{V \to 0}\left(
      \frac{1}{V} \oint_{O(V)} \vec{n}\times \vec{F}
      dS\right)
                                                    & \qquad\text{koordinatenfrei}
    \end{align*}
  \end{block}\pause

  \begin{block}{Videos von Grant Sanderson
      (3blue1brown)}
    \begin{itemize}
    \item Kanal \url{https://www.youtube.com/channel/UCYO_jab_esuFRV4b17AJtAw}
    \item  Video zu Divergenz und Rotation:
      \url{https://youtu.be/rB83DpBJQsE}
    \item Animation Package: manim \url{https://pypi.org/project/manimlib}
    \end{itemize}
  \end{block}
  
\end{frame}

\begin{frame}
  \frametitle{Differentialoperatoren (III)}

  \begin{block}{Gradient}
    Gradient eines Skalarfeldes,
    \enquote{Richtungsfeld des stärksten Anstiegs} $\to$ Vektorfeld\pause

    \begin{align*}
      \gradient f := \nabla f = \sum_{i=1}^3 \frac{\partial}{\partial x_i}f \vu{i} = \vu{i} \partial_i f & \qquad \text{in kartesischen
                                                                                                              Koordinaten}\\ \pause
      \gradient f := \lim_{V \to 0}\left( \frac{1}{V} \oint_{O(V)} f \vec{n}  dS\right)
                                                                                                            & \qquad\text{koordinatenfrei}
    \end{align*}
  \end{block}\pause

  \begin{block}{Videos und Übungen auf Khan Academy}
    \begin{itemize}
    \item Khan Academy
      \url{https://www.khanacademy.org}
    \item  Video zum Gradienten:
      \url{https://www.khanacademy.org/math/multivariable-calculus/multivariable-derivative},
      dann zu Gradient...
    \end{itemize}
  \end{block}
  
\end{frame}

\begin{frame}
  \frametitle{Koordinatensysteme}
  \begin{itemize}[<+->]
  \item Natürlich insbesondere \alert{kartesisch, zylindrisch, sphärisch} (2D und 3D)
  \item Umrechnung von Komponenten
  \item Jacobi Matrix
  \item Umrechnung von Differentialen und Integralen
    \begin{itemize}
    \item Partielle Ableitung, div, rot, grad
    \item Volumenelemente, Flächenelements, Linienelemente
    \end{itemize}
  \item Metrik (metrischer Tensor)

  \end{itemize}
\end{frame}

\begin{frame}
  \frametitle{Koordinatensysteme}
  \begin{columns}[t]
    \begin{column}{.33\linewidth}
      Kartesisches Koordinaten
% 3D AXIS with kartesian coordinates with dark unit vectors
\tdplotsetmaincoords{60}{110}
\begin{tikzpicture}[scale=2.2,tdplot_main_coords]
  
  % VARIABLES
  \def\l{0.30} % length scale dark unit vector
  \def\rtheta{0.7*\l} % length theta arc
  \def\rvec{1.2}
  \def\phivec{46}
  \def\thetavec{48}
  
  % AXES
  \coordinate (O) at (0,0,0);
  \tdplotsetcoord{P}{\rvec}{\phivec}{\thetavec};
  %\draw[dashed,myblue] (O)  -- (Pxy);
  \draw[thick,->] (0,0,0) -- (1,0,0) node[below left=-3]{$x$};
  \draw[thick,->] (0,0,0) -- (0,1,0) node[right=-1]{$y$};
  \draw[thick,->] (0,0,0) -- (0,0,1) node[above=-1]{$z$};
  %\draw (Pxy)++(0,0,0.12) --++ (\thetavec+180:0.12) --++ (0,0,-0.12);
  
  % VECTORS & DASHED
    \node[circle,inner sep=0.9,fill=myblue]
    (P) at ({\rvec*sin(\thetavec)*cos(\phivec)},{\rvec*sin(\thetavec)*sin(\phivec)},{\rvec*cos(\thetavec)}) {};
  \draw[dashed,mydarkblue] (P)  -- (Pz);
    %node[pos=0.55,above right=-6] {\contour{white}{$\rho$}};
    \draw[dashed,mydarkblue] (Py) -- (Pxy) -- (Px);
  
  % MEASURES
  \draw[dashed,mydarkblue] (0,0,0) -- (Pxy);
  \draw[<->,veccol] (\thetavec:{\rvec*sin(\phivec)}) -- (P)
    node[myblue,pos=0.55,scale=0.9]{\contour{white}{$z$}};
  \draw[<->,veccol] (Py) -- (Pxy)
    node[myblue,pos=0.55,scale=0.9]{\contour{white}{$x$}};
  \draw[<->,veccol] (Px) -- (Pxy)
    node[myblue,pos=0.55,scale=0.9]{\contour{white}{$y$}};
  \draw[->] (\rtheta,0,0) arc(0:\thetavec:\rtheta)
    node[pos=0.35,below=-2,scale=0.9] {$\varphi$};
  
   % UNITT VECTORS
   \draw[dark unit vector] (0,0,0) -- (\l,0,0)
    node[left=3,below=-8,scale=0.8]{$\vu{x}$};
  \draw[dark unit vector] (0,0,0) -- (0,\l,0)
    node[above=3.5,right=-4.5,scale=0.8]{$\vu{y}$};
  \draw[dark unit vector] (0,0,0) -- (0,0,\l)
    node[left,scale=0.8]{$\vu{z}$};
  
  % VECTORS
  \draw[vector] (O)  -- (P)
    node[pos=0.5,above left=-3] {$r$}
    node[right=1,above right=-3] {$\vec{r} = (x,y,z)$};

    % Wertebereiche
    \node at (-.3,.5,1) {$x,y,z \in \IR$};
\end{tikzpicture}
$\vu{x}, \vu{y}, \vu{z}:$ konstante\\
\phantom{$\vu{x}, \vu{y}, \vu{z}: $} Vektoren

    \end{column}
    \begin{column}{.33\linewidth}
      Zylinder Koordinaten

% 3D AXIS with cylindrical coordinates with dark unit vectors
\tdplotsetmaincoords{60}{110}
\begin{tikzpicture}[scale=2.2,tdplot_main_coords]
  
  % VARIABLES
  \def\l{0.30} % length scale dark unit vector
  \def\rtheta{0.7*\l} % length theta arc
  \def\rvec{1.2}
  \def\phivec{46}
  \def\thetavec{48}
  
  % AXES
  \coordinate (O) at (0,0,0);
  \tdplotsetcoord{P}{\rvec}{\phivec}{\thetavec};
  \draw[dashed,myblue] (O)  -- (Pxy);
  \draw[thick,->] (0,0,0) -- (1,0,0) node[below left=-3]{$x$};
  \draw[thick,->] (0,0,0) -- (0,1,0) node[right=-1]{$y$};
  \draw[thick,->] (0,0,0) -- (0,0,1) node[above=-1]{$z$};
  %\draw (Pxy)++(0,0,0.12) --++ (\thetavec+180:0.12) --++ (0,0,-0.12);
  
  % VECTORS & DASHED
    \node[circle,inner sep=0.9,fill=myblue]
    (P) at ({\rvec*sin(\thetavec)*cos(\phivec)},{\rvec*sin(\thetavec)*sin(\phivec)},{\rvec*cos(\thetavec)}) {};
  \draw[dashed,mydarkblue] (P)  -- (Pz);
    %node[pos=0.55,above right=-6] {\contour{white}{$\rho$}};
    \draw[dashed,mydarkblue] (Py) -- (Pxy) -- (Px);
  
  % MEASURES
  \draw[->,veccol] (0,0,0) -- (\thetavec:{\rvec*sin(\phivec)})
    node[myblue,pos=0.65,scale=0.9]{\contour{white}{$\rho$}};
  \draw[<->,veccol] (\thetavec:{\rvec*sin(\phivec)}) -- (P)
    node[myblue,pos=0.55,scale=0.9]{\contour{white}{$z$}};
  \draw[->] (\rtheta,0,0) arc(0:\thetavec:\rtheta)
    node[pos=0.35,below=-2,scale=0.9] {$\varphi$};
  
  % UNITT VECTORS
  \draw[dark unit vector] (0,0,0) -- (\thetavec:1.2*\l)
    node[left=3,below=-3,scale=0.8]{$\vu{\rho}$};
  \draw[dark unit vector] (0,0,0) -- (\thetavec+90:\l)
    node[above=2.5,right=-2.5,scale=0.8]{$\vu{\varphi}$};
  \draw[dark unit vector] (0,0,0) -- (0,0,\l)
    node[left,scale=0.8]{$\vu{z}$};
  
  % VECTORS
  \draw[vector] (O)  -- (P) node[pos=0.5,above left=-3] {$r$} node[right=1,above right=-3] {$\vec{r} = (\rho,\varphi,z)$};

  % Wertebereiche
    \node[align=left] at (-.3,.5,.85) {$\rho \in \IR_0^+$\\$\varphi \in [0, 2\pi]$\\$z \in \IR$};

\end{tikzpicture}
$\vu{\rho}=\vu{\rho}(\varphi)$ , $\vu{\varphi}=\vu{\varphi}(\varphi)$\\
$\vu{z}:$ konstant


      \end{column}
    \begin{column}{.33\linewidth}
      Kugel Koordinaten
% 3D AXIS with cylindrical coordinates with dark unit vectors
\tdplotsetmaincoords{60}{110}
\begin{tikzpicture}[scale=2.2,tdplot_main_coords]
  
  % VARIABLES
  \def\l{0.30} % length scale dark unit vector
  \def\rtheta{0.7*\l} % length theta arc
  \def\rvec{1.2}
  \def\phivec{46}
  \def\thetavec{48}
  
  % AXES
  \coordinate (O) at (0,0,0);
  \tdplotsetcoord{P}{\rvec}{\phivec}{\thetavec};
  \draw[dashed,myblue] (O)  -- (Pxy);
  \draw[thick,->] (0,0,0) -- (1,0,0) node[below left=-3]{$x$};
  \draw[thick,->] (0,0,0) -- (0,1,0) node[right=-1]{$y$};
  \draw[thick,->] (0,0,0) -- (0,0,1) node[above=-1]{$z$};
  %\draw (Pxy)++(0,0,0.12) --++ (\thetavec+180:0.12) --++ (0,0,-0.12);
  
  % VECTORS & DASHED
    \node[circle,inner sep=0.9,fill=myblue]
    (P) at ({\rvec*sin(\thetavec)*cos(\phivec)},{\rvec*sin(\thetavec)*sin(\phivec)},{\rvec*cos(\thetavec)}) {};
  \draw[dashed,mydarkblue] (P)  -- (Pz);
    %node[pos=0.55,above right=-6] {\contour{white}{$\rho$}};
    \draw[dashed,mydarkblue] (Py) -- (Pxy) -- (Px);
  \draw[dashed,mydarkblue] (Pxy) -- (P);
  
  % MEASURES
  %\draw[->,veccol] (0,0,0) -- (\thetavec:{\rvec*sin(\phivec)})
   % node[myblue,pos=0.65,scale=0.9]{\contour{white}{$\rho$}};
  %\draw[<->,veccol] (\thetavec:{\rvec*sin(\phivec)}) -- (P)
  %  node[myblue,pos=0.55,scale=0.9]{\contour{white}{$z$}};
  \draw[->] (\rtheta,0,0) arc(0:\thetavec:\rtheta)
    node[pos=0.35,below=-2,scale=0.9] {$\varphi$};
  
  UNITT VECTORS
  \draw[dark unit vector] (0,0,0) -- ({\rvec*sin(\thetavec)*cos(\phivec)*2*\l}, {\rvec*sin(\thetavec)*sin(\phivec)*2*\l}, {\rvec*cos(\thetavec)*2*\l})
    node[left=3,below=-9,scale=0.8]{$\vu{r}$};
  \draw[dark unit vector] (0,0,0) -- (\thetavec+90:\l)
    node[above=2.5,right=-2.5,scale=0.8]{$\vu{\varphi}$};
  %\draw[dark unit vector] (0,0,0) -- (0,0,\l)
  % node[left,scale=0.8]{$\vu{z}$};
    \tdplotsetcoord{Ptheta}{\l}{90+\thetavec}{\phivec}
  \draw[dark unit vector] (0,0,0) -- (Ptheta)
    node[above=2.5,right=-2.5,scale=0.8]{$\vu{\vartheta}$};


  %   # ARC
  \tdplotsetthetaplanecoords{\phivec}
  \tdplotdrawarc[->,tdplot_rotated_coords]{(0,0,0)}{0.36*\rvec}{0}{\thetavec}{right=2,above=-1}{$\vartheta$}

  % % VECTORS
  \draw[vector] (O)  -- (P) node[pos=0.8,above left=-3] {$r$} node[right=1,above right=-3] {$\vec{r} = (r,\vartheta,\varphi)$};
  % Wertebereiche
    \node[align=left] at (-.3,.5,.85) {$r \in \IR_0^+$\\$\vartheta \in [0, \pi]$\\$\varphi \in [0, 2\pi]$};

\end{tikzpicture}
$\vu{r}=\vu{r}(\vartheta, \varphi)$ , $\vu{\vartheta}=\vu{\vartheta}(\vartheta,\varphi), \vu{\varphi}=\vu{\varphi}(\varphi)$

\end{column}
\end{columns}
\pause

\rule[1ex]{\linewidth}{1pt}

\begin{columns}[t]
  \begin{column}{.45\linewidth}
    \alert{Kartesisch -- Zylinder}
    \begin{equation*}
    \begin{bmatrix}
      \vu{\rho}\\
      \vu{\varphi}\\
      \vu{z}
    \end{bmatrix}
    =
    \begin{bmatrix}
      \cos\varphi & \sin\varphi & 0\\
      -\sin\varphi & \cos\varphi & 0\\
      0 & 0 & 1
    \end{bmatrix}
    \begin{bmatrix}
      \vu{x}\\
      \vu{y}\\
      \vu{z}
    \end{bmatrix}    
    \end{equation*}
    \end{column}
    \begin{column}{.52\linewidth}
      \alert{Kartesisch -- Kugel}
    \begin{equation*}
    \begin{bmatrix}
      \vu{r}\\
      \vu{\vartheta}\\
      \vu{\varphi}
    \end{bmatrix}
    =
    \begin{bmatrix}
      \sin\vartheta\cos\varphi & \sin\vartheta\sin\varphi & \cos\vartheta\\
      \cos\vartheta\cos\varphi & \cos\vartheta\sin\varphi & - \sin\vartheta\\
      -\sin\varphi & \cos\varphi & 0
    \end{bmatrix}
    \begin{bmatrix}
      \vu{x}\\
      \vu{y}\\
      \vu{z}
    \end{bmatrix}    
    \end{equation*}

    \end{column}
  \end{columns}

\end{frame}


\begin{frame}
  \frametitle{Integralsätze}
  \begin{block}{Satz von Gauss}
Das Oberflächenintegral eines Vektorfeldes $\vec{F}$ über eine geschlossene Fläche O(V) ist
gleich dem Volumenintegral der Divergenz von
$\vec{F}$, erstreckt über das von der
Fläche O(V) eingeschlossene Volumen V:
\begin{align*}
\oint_V \divergenz\vec{F} \; dV = \oint_{O(V)} \vec{F} \cdot \vec{n} \; dS
\end{align*}\pause
Das geht natürlich auch in 2D: Volumen $\to$ Fläche; Oberflächenintegral
$\to$ geschlossenes Wegintegral
\end{block}\pause

    \begin{block}{Satz von Stokes}
      Das Kurven- oder Linienintegral eines Vektorfeldes $\vec{F}$ längs einer einfach
geschlossenen Kurve C(A) ist gleich dem Oberflächenintegral der Rotation von
$\vec{F}$ über eine beliebige Fläche A, die durch die Kurve C berandet wird:
\begin{equation*}
  \oint_{C(A)} \vec{F}\cdot d\vec{s} = \int_A\rotation\vec{F} \cdot
\;  d\vec{S} = \int_A\rotation\vec{F} \cdot
 \vec{n}\; dS 
 \end{equation*}
\end{block}
\end{frame}


\begin{frame}
  \frametitle{$\delta$-Funktion (I)}
  \begin{block}{Motivation}
    Die Berechnung von Differentialen in der Nähe von Singularitäten
    des Feldes ist häufig problematisch. Beispiel Coulomb Feld einer
    Punktladung Q im Ursprung:
    $$
    \vec{E} = \frac{1}{4\pi\varepsilon_0} \frac{Q}{r^2} \vu{r}
    $$
    Die Divergenz (Quellstärke) sollte nur am Ursprung ungleich Null
    sein und am Ursprung einen definierten Wert haben.
    \begin{align*}
      \divergenz \left( \frac{1}{r^2}\vu{r}\right) & = \frac{1}{r^2}
                                                  \partial_r \left(
                                                  r^2 \frac{1}{r^2}
                                                  \right)\\\pause
                                                  &= \frac{1}{r^2}
                                                  \partial_r \left(
                                                  1
                                                  \right)\pause \mbox{???}
      \end{align*}
    \end{block}
  
  \end{frame}
\begin{frame}
  \frametitle{$\delta$-Funktion (II)}
  \begin{block}{Definition über ihre Eigenschaften}
    Es soll gelten
    $$
    \delta(x)=0 \text{ für } x\neq 0 \text{ und }
    \int_{-\infty}^{+\infty} \delta(x) dx = 1
    $$\pause

  Solche Funktionen existieren nur im Grenzwert von $n\to\infty$ von
  Funktionen $f_n(x)$ mit $\int_{-\infty}^{+\infty} f_n(x) dx = 1$ und
  $f_n(x) \stackrel{n\to\infty}{\to}0$ für $x\neq 0$ und $f_n(0) \stackrel{n\to\infty}{\to}\infty$ 
\end{block}\pause

\begin{block}{Beispiele}
  \begin{align*}
    f_n(x)  = &
             \begin{cases}
               n & \text{ für } |x|\le\frac{1}{2n}\\
               0 & \text{ sonst } 
 \end{cases}\\\pause
   f_n(x)&= n e^{-\pi n^2x^2}\\\pause
   f_n(x) &= \frac{n}{\pi}\left( \frac{\sin nx}{nx} \right)^2
   \end{align*}
 \end{block}


  \end{frame}
\begin{frame}
  \frametitle{$\delta$-Funktion (III)}
  \begin{block}{Wichtige Beziehungen}
\pause Da die $\delta$-Funktion für alle $x\neq 0$ verschwindet und aufgrund
der Normierung folgt unmittelbar:
$$
\int_{-\infty}^{+\infty} f(x) \delta(x) dx = f(0) \int_{-\infty}^{+\infty}
 \delta(x) dx =  f(0)
 $$\pause
 Dies kann einfach für eine Verschiebung $x\to x-a$ verallgemeinert
 werden:
 $$
\int_{-\infty}^{+\infty} f(x) \delta(x-a) dx = f(a) \int_{-\infty}^{+\infty}
 \delta(x-a) dx =  f(a)
 $$\pause
 Wichtig ist auch die Skalierungseigenschaft (z.B. auch für
 dimensionsbehaftete Größen; k = Einheit von x):
 $$
\delta(kx)  = \frac{1}{|k|}\delta(x)
$$\pause
Verallgemeinerung auf 3D:
$$
\Ortsr[v] = x\vu{x} +y\vu{y}+z\vu{z} \to
\delta^3(\Ortsr[v])=\delta(x) \delta(y) \delta(z) \to
\int_{\mathbb{R}^3} \delta^3(\Ortsr[v]) dV = 1 
$$
 
 

\end{block}
\end{frame}
\begin{frame}
  \frametitle{$\delta$-Funktion (IV)}
Mit dem Gaußschen Integralsatz und der $\delta$-Funktion lässt sich
$\divergenz\left(\frac{1}{r^2}\vu{r}\right)$ leicht bestimmen:
\begin{align*}
  \int_{K(R)} \divergenz\left(\frac{1}{r^2}\vu{r}\right) dV = &
 \oint_{O(K)} \frac{1}{r^2}\vu{r}\cdot d\vec{A}\pause \qquad r=R,\pause \qquad d\vec{A} =R^2 \sin\vartheta d\vartheta d\varphi \vu{r}\\
  \pause
  = & 2\pi \int_0^\pi \frac{1}{R^2} R^2 \sin\vartheta d\vartheta \\
  \pause
  = & 4\pi \\
  \pause
  \divergenz\left(\frac{1}{r^2}\vu{r}\right)  =& 4\pi\delta^3(\Ortsr[v])
\end{align*}\pause

Einfach berechnen lässt sich:
$$
\gradient \frac{1}{r} = \vu{r} \partial_r \frac{1}{r} + \vec{0} + \vec{0} =-\frac{1}{r^2} \vu{r}
$$\pause
Hiermit folgt die wichtige Beziehung für den \alert{Laplace-Operator}
$\Delta = \divergenz\gradient$:
$$
\Delta \frac{1}{r} = \divergenz\gradient \frac{1}{r} = -4\pi\delta^3(\Ortsr[v]) 
$$
\end{frame}
\begin{frame}
  \frametitle{Zerlegung von Vektorfeldern (I)}
  \begin{block}{Helmholtz-Theorem}
    Ein Vektorfeld $\vec{F}(\Ortsr[v])$, welches für $r\to\infty$
    stärker als $1/r$ gegen Null abfällt (das ist für praktisch
    relevante Probleme regelmäßig erfüllt), lässt sich als Summe eines
    rotationsfreien Anteils $\vec{a}(\Ortsr[v])$ und eines
    divergenzfreien Anteils $\vec{b}(\Ortsr[v])$ darstellen:
    $$
    \vec{F}(\Ortsr[v]) = \vec{a}(\Ortsr[v]) + \vec{b}(\Ortsr[v]) \text{ mit
    } \rotation  \vec{a} = \vec{0} \land \divergenz  \vec{b} = 0 
    $$\pause
    \begin{align*}
    \text{Gradientenfelder sind rotationsfrei: } & \rotation\gradient \phi =\vec{0}\\ \pause
      \text{Rotationsfelder sind divergenzfrei: } & \divergenz\rotation \vec{A}
                                                    = 0
    \end{align*}\pause
    Hiermit folgt die Darstellung mit einen \alert{Skalarpotential}
    $\phi$ und einem \alert{Vektorpotential} $\vec{A}$:
    $$
    \vec{F}(\Ortsr[v]) = - \gradient \phi (\Ortsr[v]) + \rotation
    \vec{A}(\Ortsr[v]) \text{ (\enquote{-} ist Konvention)}
    $$
  \end{block}
  \end{frame}
\begin{frame}
  \frametitle{Zerlegung von Vektorfeldern (II)}
  \begin{block}{Skalar- und Vektorpotential}
    \alert{Skalar-} $\phi (\Ortsr[v])$ und \alert{Vektorpotential}
    $\vec{A}(\Ortsr[v])$ lassen sich direkt aus dem Feld
    $\vec{F}(\Ortsr[v])$ berechnen. Es gilt:

    \begin{align*}
  \phi (\Ortsr[v]) = & \frac{1}{4\pi} \int_V \frac{\divergenz'
                     \vec{F}(\Ortsr[vs])}{\left|\Ortsr[v]-\Ortsr[vs]\right|}
                     d^3r'  \uncover<2->{- \frac{1}{4\pi} \oint_{O(V)}
                     \vec{n}' \cdot \frac{\vec{F}(\Ortsr[vs])}{\left|\Ortsr[v]-\Ortsr[vs]\right|}
                     d^2r'}\\
  \vec{A} (\Ortsr[v]) = & \frac{1}{4\pi} \int_V \frac{\rotation'
                     \vec{F}(\Ortsr[vs])}{\left|\Ortsr[v]-\Ortsr[vs]\right|}
                     d^3r'  \uncover<2->{- \frac{1}{4\pi} \oint_{O(V)}
                     \vec{n}' \times \frac{\vec{F}(\Ortsr[vs])}{\left|\Ortsr[v]-\Ortsr[vs]\right|}
                     d^2r'}
    \end{align*}\pause
  \end{block}\pause

  \begin{block}{Longitudinale und transversale Komponente}
    Häufig bezeichnet man die rotationsfreie Komponente $ - \gradient
    \phi (\Ortsr[v])$ als \alert{longitudinale Komponente} und die
    divergrenzfreie Komponente $\rotation
    \vec{A}(\Ortsr[v])$ als \alert{transversale Komponente} des
    Vektorfeldes $\vec{F}(\Ortsr[v])$.
    \end{block}
  \end{frame}

\input{finalframe.inc}
\end{document}