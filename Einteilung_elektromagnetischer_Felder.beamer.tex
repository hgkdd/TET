\input{head.inc}
% Präambelbefehle für die Präsentation
\title[TET: Einteilung elektromagnetischer Felder]{Einteilung elektromagnetischer Felder}

\begin{document}
% 
% Frontmatter 
% 
%%%%%%%%%%%%%%%%%%%%%%%%%%%%%%%%%%%%%%%%%%%%%%%%%%%%%%%%%%%%%%%%%%%%%%%%%%%%%%%%%%%%%%%%%%%%%%%%%%%%%%%%%%%%%%%%%%%%%%%%%%%%% 

%% inserts the title page and the table of contents
\maketitle

% 
% Content 
% 
%%%%%%%%%%%%%%%%%%%%%%%%%%%%%%%%%%%%%%%%%%%%%%%%%%%%%%%%%%%%%%%%%%%%%%%%%%%%%%%%%%%%%%%%%%%%%%%%%%%%%%%%%%%%%%%%%%%%%%%%%%%%% 
\section{Einteilung elektromagnetischer Felder}

\begin{frame}
  \frametitle{A keine Einschränkungen - Maxwell-Gleichungen}

\begin{align*}
	& \rotation \efeld[v] = -\dfrac{\partial}{\partial t} \tetB[v]
		&&	\oint\limits_{\rand(\flaeche)} \efeld[v]
                   \cdot \intweg[v] = -\dfrac{\upd}{\upd t}
                   \iint\limits_{\flaeche} \tetB[v] \cdot \intflaeche[v]\\
	& \rotation \magfeld[v] = \elstromdichte[v] +\dfrac{\partial}{\partial t} \verschiebung[v]
		&&	\oint\limits_{\rand(\flaeche)} \magfeld[v]
                   \cdot \intweg[v] = \iint\limits_{\flaeche}
                   \elstromdichte[v] \cdot \intflaeche[v] + \dfrac{\upd}{\upd t}
                   \iint\limits_{\flaeche} \verschiebung[v] \cdot \intflaeche[v]\\
	& \divergenz \tetB[v] = 0
		&&	\oiint\limits_{\oberfl(\volumen)} \tetB[v]\cdot \intflaeche[v] = 0\\
	& \divergenz \verschiebung[v] = \laddichte{V}
		&&	\oiint\limits_{\oberfl(\volumen)}
                   \verschiebung[v]\cdot \intflaeche[v] =
                   \iiint\limits_{\volumen} \laddichte{V} \intvolumen
  \\
  \\
  \hline
  \\
  & \dfrac{\partial \laddichte{V}}{\partial t} + \divergenz
    \elstromdichte[v] = 0 && \dfrac{\upd}{\upd t}
                             \iiint\limits_{\volumen} \laddichte{V}
                             \intvolumen +
                             \oiint\limits_{\oberfl(\volumen)}
                             \elstromdichte[v] \cdot   \intflaeche[v] = 0
\end{align*}

 \end{frame}

\begin{frame}
  \frametitle{B Statischer Grenzfall: $\dfrac{\partial ...}{\partial t} = 0$}

\begin{align*}
	& \rotation \efeld[v] = \only<-1|handout:0>{-\dfrac{\partial}{\partial t} \tetB[v]}\only<2->{\alert{\vec{0}}}
		&&	\oint\limits_{\rand(\flaeche)} \efeld[v]
                   \cdot \intweg[v] = \only<-2|handout:0>{-\dfrac{\upd}{\upd t}
                   \iint\limits_{\flaeche} \tetB[v] \cdot
                   \intflaeche[v]}
                   \only<3->{\alert{0}}\\
	& \rotation \magfeld[v] = \elstromdichte[v]
   +\only<-3|handout:0>{\dfrac{\partial}{\partial t} \verschiebung[v]}
   \only<4->{\alert{\vec{0}}}
		&&	\oint\limits_{\rand(\flaeche)} \magfeld[v]
                   \cdot \intweg[v] = \iint\limits_{\flaeche}
                   \elstromdichte[v] \cdot \intflaeche[v] + \only<-4|handout:0>{\dfrac{\upd}{\upd t}
                   \iint\limits_{\flaeche} \verschiebung[v] \cdot \intflaeche[v]}\only<5->{\alert{0}}\\
	& \divergenz \tetB[v] = 0
		&&	\oiint\limits_{\oberfl(\volumen)} \tetB[v]\cdot \intflaeche[v] = 0\\
	& \divergenz \verschiebung[v] = \laddichte{V}
		&&	\oiint\limits_{\oberfl(\volumen)}
                   \verschiebung[v]\cdot \intflaeche[v] =
                   \iiint\limits_{\volumen} \laddichte{V} \intvolumen
  \\
  \\
  \hline
  \\
  & \only<-5|handout:0>{\dfrac{\partial \laddichte{V}}{\partial t}}\only<6->{\alert{0}} + \divergenz
    \elstromdichte[v] = 0 && \only<-6|handout:0>{\dfrac{\upd}{\upd t}
                             \iiint\limits_{\volumen} \laddichte{V}
                             \intvolumen}\only<7->{\alert{0}} +
                             \oiint\limits_{\oberfl(\volumen)}
                             \elstromdichte[v] \cdot   \intflaeche[v] = 0
\end{align*}
\end{frame}
\begin{frame}
  \frametitle{B.1 Elektrostatik: $\elstromdichte[v] = \vec{0}$ }

\uncover<2->{Grundgleichungen der Elektrostatik}
\only<1|handout:0>{
\begin{align*}
	& \rotation \efeld[v] = \vec{0}
		&&\oint\limits_{\rand(\flaeche)} \efeld[v]
                   \cdot \intweg[v] = 0\\
	& \rotation \magfeld[v] = \elstromdichte[v]
		&&	\oint\limits_{\rand(\flaeche)} \magfeld[v]
                   \cdot \intweg[v] = \iint\limits_{\flaeche}
                   \elstromdichte[v] \cdot \intflaeche[v] \\
	& \divergenz \tetB[v] = 0
		&&	\oiint\limits_{\oberfl(\volumen)} \tetB[v]\cdot \intflaeche[v] = 0\\
	& \divergenz \verschiebung[v] = \laddichte{V}
		&&\oiint\limits_{\oberfl(\volumen)}
                   \verschiebung[v]\cdot \intflaeche[v] =
                   \iiint\limits_{\volumen} \laddichte{V} \intvolumen\\
  \\
 \hline
 \\
  & \divergenz
    \elstromdichte[v] = 0 && 
                             \oiint\limits_{\oberfl(\volumen)}
                             \elstromdichte[v] \cdot   \intflaeche[v] = 0
\end{align*}
}
\only<2->{
\begin{align*}
	& \rotation \efeld[v] = \vec{0}
		&&\oint\limits_{\rand(\flaeche)} \efeld[v]
                   \cdot \intweg[v] = 0\\
	& \divergenz \verschiebung[v] = \laddichte{V}
		&&\oiint\limits_{\oberfl(\volumen)}
                   \verschiebung[v]\cdot \intflaeche[v] =
                   \iiint\limits_{\volumen} \laddichte{V} \intvolumen
\end{align*}
 }
\uncover<3->{für \alert{homogene, lineare, isotrope} Medien:
  $\verschiebung[v]=\varepsilon_0\varepsilon_r \efeld[v]=\varepsilon \efeld[v]$
\begin{align*}
	& \rotation \efeld[v] = \vec{0}
		&&\oint\limits_{\rand(\flaeche)} \efeld[v]
                   \cdot \intweg[v] = 0\\
	& \divergenz \efeld[v] = \frac{1}{\varepsilon}\laddichte{V}
		&&\oiint\limits_{\oberfl(\volumen)}
                   \efeld[v]\cdot \intflaeche[v] = \frac{1}{\varepsilon}
                   \iiint\limits_{\volumen} \laddichte{V} \intvolumen
\end{align*}
}
\uncover<4->{$\rotation \efeld[v] = \vec{0} \Rightarrow \efeld[v] = -\gradient
  \phi \to -\divergenz\efeld[v] = \divergenz\gradient
  \phi = \alert{\Delta\phi = -
    \frac{1}{\varepsilon}\laddichte{V}}$ (\alert{Poisson-Gleichung})}
\end{frame}

\begin{frame}
  \frametitle{B.2 Magnetostatik: $\laddichte{V} = 0$}

\uncover<2->{Grundgleichungen der Magnetostatik}
\only<1|handout:0>{
\begin{align*}
	& \rotation \efeld[v] = \vec{0}
		&&\oint\limits_{\rand(\flaeche)} \efeld[v]
                   \cdot \intweg[v] = 0\\
	& \rotation \magfeld[v] = \elstromdichte[v]
		&&	\oint\limits_{\rand(\flaeche)} \magfeld[v]
                   \cdot \intweg[v] = \iint\limits_{\flaeche}
                   \elstromdichte[v] \cdot \intflaeche[v] \\
	& \divergenz \tetB[v] = 0
		&&	\oiint\limits_{\oberfl(\volumen)} \tetB[v]\cdot \intflaeche[v] = 0\\
	& \divergenz \verschiebung[v] = \laddichte{V}
		&&\oiint\limits_{\oberfl(\volumen)}
                   \verschiebung[v]\cdot \intflaeche[v] =
                   \iiint\limits_{\volumen} \laddichte{V} \intvolumen\\
  \\
 \hline
 \\
  & \divergenz
    \elstromdichte[v] = 0 && 
                             \oiint\limits_{\oberfl(\volumen)}
                             \elstromdichte[v] \cdot   \intflaeche[v] = 0
\end{align*}
}
\only<2->{
\begin{align*}
	& \rotation \magfeld[v] = \elstromdichte[v]
		&&	\oint\limits_{\rand(\flaeche)} \magfeld[v]
                   \cdot \intweg[v] = \iint\limits_{\flaeche}
                   \elstromdichte[v] \cdot \intflaeche[v] \\
	& \divergenz \tetB[v] = 0
		&&	\oiint\limits_{\oberfl(\volumen)} \tetB[v]\cdot \intflaeche[v] = 0\\
 & \divergenz
    \elstromdichte[v] = 0 && 
                             \oiint\limits_{\oberfl(\volumen)}
                             \elstromdichte[v] \cdot   \intflaeche[v] = 0
\end{align*}
 }
\uncover<3->{für \alert{homogene, lineare, isotrope} Medien:
  $\tetB[v]=\mu_0\mu_r \magfeld[v]=\mu \magfeld[v]$
\begin{align*}
	& \rotation \tetB[v] = \mu\elstromdichte[v]
		&&	\oint\limits_{\rand(\flaeche)} \tetB[v]
                   \cdot \intweg[v] = \mu\iint\limits_{\flaeche}
                   \elstromdichte[v] \cdot \intflaeche[v] \\
	& \divergenz \tetB[v] = 0
		&&	\oiint\limits_{\oberfl(\volumen)} \tetB[v]\cdot \intflaeche[v] = 0
\end{align*}
}
\uncover<4->{$\divergenz \tetB[v] = 0 \Rightarrow \tetB[v] =
  \rotation \magvekpot[v] \to \rotation\tetB[v] =
  \alert{ \rotation\rotation \magvekpot[v] =
    \mu\elstromdichte[v]}$ (führt später zur \alert{Eichung})}
\end{frame}

\begin{frame}
  \frametitle{B.3 Stationäres Strömungsfeld: $\divergenz \elstromdichte[v] = 0, \elstromdichte[v]=\elstromdichte[v]_L=\kappa\efeld[v]$}

\uncover<2->{Grundgleichungen des Stationären Strömungsfeldes in Gebieten \alert{ohne Elektromotorische Kraft} (siehe \enquote{Stationäres Elektrisches Strömungsfeld})}
\only<1|handout:0>{
\begin{align*}
	& \rotation \efeld[v] = \vec{0}
		&&\oint\limits_{\rand(\flaeche)} \efeld[v]
                   \cdot \intweg[v] = 0\\
	& \rotation \magfeld[v] = \elstromdichte[v]
		&&	\oint\limits_{\rand(\flaeche)} \magfeld[v]
                   \cdot \intweg[v] = \iint\limits_{\flaeche}
                   \elstromdichte[v] \cdot \intflaeche[v] \\
	& \divergenz \tetB[v] = 0
		&&	\oiint\limits_{\oberfl(\volumen)} \tetB[v]\cdot \intflaeche[v] = 0\\
	& \divergenz \verschiebung[v] = \laddichte{V}
		&&\oiint\limits_{\oberfl(\volumen)}
                   \verschiebung[v]\cdot \intflaeche[v] =
                   \iiint\limits_{\volumen} \laddichte{V} \intvolumen\\
  \\
 \hline
 \\
  & \divergenz
    \elstromdichte[v] = 0 && 
                             \oiint\limits_{\oberfl(\volumen)}
                             \elstromdichte[v] \cdot   \intflaeche[v] = 0
\end{align*}
}
\only<2->{
\begin{align*}
	& \rotation \efeld[v] = \vec{0}
		&&\oint\limits_{\rand(\flaeche)} \efeld[v]
                   \cdot \intweg[v] = 0\\
	& \rotation \magfeld[v] = \elstromdichte[v]
		&&	\oint\limits_{\rand(\flaeche)} \magfeld[v]
                   \cdot \intweg[v] = \iint\limits_{\flaeche}
                   \elstromdichte[v] \cdot \intflaeche[v] \\
 & \divergenz
    \elstromdichte[v] = \kappa\divergenz \efeld[v]= 0 && 
                             \oiint\limits_{\oberfl(\volumen)}
                             \elstromdichte[v] \cdot   \intflaeche[v]
                                                         = \oiint\limits_{\oberfl(\volumen)}
                             \kappa\efeld[v] \cdot   \intflaeche[v]
                                                         = 0 
\end{align*}
 }
 \uncover<3->{
   \begin{itemize}
   \item Das E-Feld ist divergenz- und rotationsfrei
   \item Positive und negative Ladungen sind kompensiert: $\laddichte{V} = 0$
     \end{itemize}
 }
\end{frame}

\begin{frame}
  \frametitle{C Quasistationäres Feld}

  \begin{itemize}[<+->]
    \item Es sind zeitliche Änderungen vorhanden; diese sind aber
      \enquote{langsam}.
      \item Momentanaufnahmen der Felder entsprechen den statischen
        Feldern
        \item Wirkungen werden \enquote{verzögerungsfrei} übermittelt
          $\to$ Keine \alert{Retardierung}
        \item Keine Abstrahlung
        \item Orts- und Zeitabhängigkeit sind entkoppelt
          \item Berechnungen mit den Lösungsmethoden der Statik (aber zeitabhängig)
          \item zwei Fälle:
            \begin{itemize}[<+->]
            \item Quasi-Elektrostatik (Induktion vernachlässigen)
              $$
              \dfrac{\partial}{\partial t} \verschiebung[v] \neq \vec{0}
              \text{ und } \dfrac{\partial}{\partial t} \tetB[v] \to \vec{0}
              \Rightarrow \rotation\efeld[v] \to \vec{0}
              $$
              \item Quasi-Magnetostatik (Verschiebungsstrom vernachlässigen)
              $$
              \dfrac{\partial}{\partial t} \verschiebung[v] \to \vec{0}
              \text{ und } \dfrac{\partial}{\partial t} \tetB[v] \neq \vec{0}
              \Rightarrow \rotation\magfeld[v] \to \elstromdichte[v]
              $$
              \end{itemize}

    \end{itemize}
\end{frame}

\begin{frame}
  \frametitle{C.1 Quasi-Elektrostatik}
\begin{align*}
	& \alert{\rotation \efeld[v] = \vec{0}}
		&&	\alert{\oint\limits_{\rand(\flaeche)} \efeld[v]
                   \cdot \intweg[v] = 0}\\
	& \rotation \magfeld[v] = \elstromdichte[v] +\dfrac{\partial}{\partial t} \verschiebung[v]
		&&	\oint\limits_{\rand(\flaeche)} \magfeld[v]
                   \cdot \intweg[v] = \iint\limits_{\flaeche}
                   \elstromdichte[v] \cdot \intflaeche[v] + \dfrac{\upd}{\upd t}
                   \iint\limits_{\flaeche} \verschiebung[v] \cdot \intflaeche[v]\\
	& \divergenz \tetB[v] = 0
		&&	\oiint\limits_{\oberfl(\volumen)} \tetB[v]\cdot \intflaeche[v] = 0\\
	& \divergenz \verschiebung[v] = \laddichte{V}
		&&	\oiint\limits_{\oberfl(\volumen)}
                   \verschiebung[v]\cdot \intflaeche[v] =
                   \iiint\limits_{\volumen} \laddichte{V} \intvolumen
  \\
  \\
  \hline
  \\
  & \dfrac{\partial \laddichte{V}}{\partial t} + \divergenz
    \elstromdichte[v] = 0 && \dfrac{\upd}{\upd t}
                             \iiint\limits_{\volumen} \laddichte{V}
                             \intvolumen +
                             \oiint\limits_{\oberfl(\volumen)}
                             \elstromdichte[v] \cdot   \intflaeche[v] = 0
\end{align*}
\end{frame}

\begin{frame}
  \frametitle{C.2 Quasi-Magnetostatik}
\begin{align*}
	& \rotation \efeld[v] = -\dfrac{\partial}{\partial t} \tetB[v]
		&&	\oint\limits_{\rand(\flaeche)} \efeld[v]
                   \cdot \intweg[v] = -\dfrac{\upd}{\upd t}
                   \iint\limits_{\flaeche} \tetB[v] \cdot \intflaeche[v]\\
	& \alert{\rotation \magfeld[v] = \elstromdichte[v]}
		&&	\alert{\oint\limits_{\rand(\flaeche)} \magfeld[v]
                   \cdot \intweg[v] = \iint\limits_{\flaeche}
                   \elstromdichte[v] \cdot \intflaeche[v] }\\
	& \divergenz \tetB[v] = 0
		&&	\oiint\limits_{\oberfl(\volumen)} \tetB[v]\cdot \intflaeche[v] = 0\\
	& \divergenz \verschiebung[v] = \laddichte{V}
		&&	\oiint\limits_{\oberfl(\volumen)}
                   \verschiebung[v]\cdot \intflaeche[v] =
                   \iiint\limits_{\volumen} \laddichte{V} \intvolumen
  \\
  \\
  \hline
  \\
  & \dfrac{\partial \laddichte{V}}{\partial t} + \divergenz
    \elstromdichte[v] = 0 && \dfrac{\upd}{\upd t}
                             \iiint\limits_{\volumen} \laddichte{V}
                             \intvolumen +
                             \oiint\limits_{\oberfl(\volumen)}
                             \elstromdichte[v] \cdot   \intflaeche[v] = 0
\end{align*}
\end{frame}

\begin{frame}
  \begin{itemize}[<+->]
    \item Der allgemeine Satz der Maxwell-Gleichungen ist zwar immer
      korrekt, aber \alert{häufig unnötig kompliziert}
      \item Das Auffinden der Lösungen wird stark vereinfacht, wenn
        \alert{geeignete Annahmen} gemacht werden.
        \item Die vorgestellt Einteilung der Vereinfachungen kann auch
          noch weiter granularisiert werden.
      \item In den folgenden Einheiten werden die Lösungsverfahren für
        die verschiedenen Fälle eingeführt und die zugehörigen
        Lösungen diskutiert.
      \item \alert{Lernziele:}
        \begin{itemize}[<+->]
        \item Sie \alert{kennen} die Maxwell-Gleichungen
          \item Sie können die Beziehungen zwischen den Gleichungen
            \alert{erklären}
          \item Sie können vereinfachte Grundgleichungen \alert{entwickeln}
            \item Sie können Problemstellungen \alert{analysieren} und
              Lösungen finden
              \item Sie können Modelle für neue
                Problemstellungen \alert{entwickeln}
                \item Sie können für neue
                  Problemstellungen geeignete Lösungsmethoden \alert{auswählen}
          \end{itemize}
      \end{itemize}
  \end{frame}
 \input{finalframe.inc}
\end{document}