\input{head.inc}

% Präambelbefehle für die Präsentation
\title[TET: Elektrostatik-IV: Formale Lösung]{Elektrostatik-IV: Formale Lösung}

\begin{document}
% 
% Frontmatter 
% 
%%%%%%%%%%%%%%%%%%%%%%%%%%%%%%%%%%%%%%%%%%%%%%%%%%%%%%%%%%%%%%%%%%%%%%%%%%%%%%%%%%%%%%%%%%%%%%%%%%%%%%%%%%%%%%%%%%%%%%%%%%%%% 

%% inserts the title page and the table of contents
\maketitle

% 
% Content 
% 
%%%%%%%%%%%%%%%%%%%%%%%%%%%%%%%%%%%%%%%%%%%%%%%%%%%%%%%%%%%%%%%%%%%%%%%%%%%%%%%%%%%%%%%%%%%%%%%%%%%%%%%%%%%%%%%%%%%%%%%%%%%%% 
\section{Elektrostatik-IV: Formale Lösung}

\begin{frame}

  \frametitle{Greensche Funktionen - Fundamentallösung}

  \begin{itemize}[<+->]
  \item Sei $\mathcal{L}=\mathcal{L}(x)$ ein \alert{linearer Differentialoperator} der auf Funktionen (besser Distributionen) angewendet wird, die auf einen Teilraum $\Omega$ des $\mathbb{R}^n$ definiert sind.
  \item  Eine \alert{Greensche Funktion} oder \alert{Fundamentallösung} $G=G(x,s)$ zum Operator $\mathcal{L}$ am Punkt $s\in \Omega$ ist \alert{jede Lösung} der Gleichung
    $$
    \mathcal{L}G(x,s) = \delta^n(x-s)
    $$
  \item Die Bedeutung der Greenschen Funktionen für die Lösung inhomogener partieller Differentialgleichungen wird folgendermaßen klar:
  \item Multiplikation mit beliebiger Funktion $f(s)$ und Integration über $s$:
    $$
    \int \mathcal{L}G(x,s)f(s) \upd^n s = \int \delta^n(x-s) f(s) \upd^n s = f(x)
    $$
  \item Der Operator $\mathcal{L}$ ist linear und wirkt nur auf x:
    $$
    \mathcal{L}\left[ \int G(x,s)f(s) \upd^n s  \right] = f(x)
    $$
  \item Sucht man also eine Lösung der inhomogenen lineare Differentialgleichung $\mathcal{L} y(x) = f(x)$, und kennt eine Greensche Funktion (Fundamentallösung) $G(x,s)$ - also $\mathcal{L}G(x,s)=\delta^n(x-s)$, so gilt:
    $$
    \boxed{y(x) = \int G(x,s)f(s)\upd^n s \text{ ist Lösung von } \mathcal{L} y(x) = f(x)}
    $$
  \end{itemize}
  
\end{frame}


\begin{frame}
  \frametitle{Eigenschaften von Greenschen Funktionen}
  \begin{itemize}[<+->]
  \item Wenn der \alert{Kern} von $\mathcal{L}$ nicht trivial (nur die Null-Funktion) ist, hat das Problem $\mathcal{L}G(x,s)=\delta^n(x-s)$ \alert{unendlich} viele Lösungen $G(x,s)$ (Addition von Elementen des Kerns)
  \item Im Allgemeinen hat ein Operator $\mathcal{L}$ keine eindeutige Greensche Funktion
  \item Diese Mehrdeutigkeit werden wir bei der Lösung von \alert{Randwertproblemen} ausnutzen!
  \item Ohne Beweis: Greensche Funktionen sind \alert{adjungiert symmetrisch}: $G(x,s)=G^\dag(s,x)$
  \item Ohne Beweis, wichtiger: Ist $\mathcal{L}$ eine \alert{selbstadjungierter Operator}, also $\mathcal{L} = \mathcal{L}^\dag$, dann sind seine Greenschen Funktionen \alert{symmetrisch}:
    $$
    \boxed{\mathcal{L} = \mathcal{L}^\dag \Rightarrow G(x,s) = G(s,x)  }
    $$
  \end{itemize}  
  \end{frame}

  \begin{frame}
  \frametitle{Randwertproblem der Elektrostatik}

  \begin{itemize}[<+->]
  \item Poisson-Gleichung: $\laplace_r \phi(\Ortsr[v]) = -\frac{1}{\varepsilon_0} \laddichte{V}(\Ortsr[v]) \Rightarrow \mathcal{L} = \laplace_r \text{ selbstadjungiert}$
  \item Bestimmungsgleichung für Greensche Funktion: $\laplace_rG(\Ortsr[v],\Ortsr[vs]) = -\frac{1}{\varepsilon_0} \delta^3(\Ortsr[v]-\Ortsr[vs])$ (konstanter Faktor unbedeutend!)
  \item Bekannte Identität: $\delta^3(\Ortsr[v]-\Ortsr[vs]) = -\frac{1}{4\pi}\laplace_r \frac{1}{|\Ortsr[v]-\Ortsr[vs]|}$
  \item Damit ist \alert{eine} Greensche Funktion des Laplace-Operators bereits gefunden:
    $$
   \boxed{ \laplace_r G(\Ortsr[v],\Ortsr[vs]) = -\frac{1}{\varepsilon_0} \delta^3(\Ortsr[v]-\Ortsr[vs]) \text{ gilt für } G(\Ortsr[v],\Ortsr[vs]) = \frac{1}{4\pi\varepsilon_0}\frac{1}{|\Ortsr[v]-\Ortsr[vs]|} } 
   $$
 \item Zu \alert{dieser speziellen} Greenschen Funktion kann jedes Element $\Gamma(\Ortsr[v],\Ortsr[vs])$ des Kerns des Laplace-Operators ($\laplace_r \Gamma(\Ortsr[v],\Ortsr[vs]) = 0$) addiert werden, wobei $\Gamma$ eine symmetrische Funktion sein muss (Laplace ist selbstadjungiert):
   $$
   \boxed{G(\Ortsr[v],\Ortsr[vs]) = \frac{1}{4\pi\varepsilon_0}\frac{1}{|\Ortsr[v]-\Ortsr[vs]|} + \Gamma(\Ortsr[v],\Ortsr[vs]) \text{ mit } \laplace_r \Gamma(\Ortsr[v],\Ortsr[vs]) = 0,\quad \Gamma(\Ortsr[v],\Ortsr[vs]) =\Gamma(\Ortsr[vs],\Ortsr[v])} 
   $$
    \end{itemize}
  \end{frame}

  \begin{frame}
  \frametitle{Skalarpotential berechnen}

  \begin{itemize}[<+->]
     \item Wir setzen (wieder) in die 2. Greenschen Identität ein: $f\to\phi(\Ortsr[vs])$ und $g\to G(\Ortsr[v],\Ortsr[vs])$:
       \begin{align*}
         \int_V \left[  \phi(\Ortsr[vs]) \laplace_{\Ortsr[s]} G(\Ortsr[v],\Ortsr[vs])  - G(\Ortsr[v],\Ortsr[vs]) \laplace_{\Ortsr[s]}  \phi(\Ortsr[vs]) \right] \upd^3\Ortsr[s] &=  -\frac{1}{\varepsilon_0} \int_V  \phi(\Ortsr[vs]) \delta^3 (\Ortsr[v] - \Ortsr[vs]) \upd^3r' \\
         &+ \frac{1}{\varepsilon_0} \int_V G(\Ortsr[v],\Ortsr[vs])\rho(\Ortsr[vs]) \upd^3r'\\
         &= \oint_{O(V)} \left[ \phi\frac{\partial G(\Ortsr[v],\Ortsr[vs])}{\partial n'}
                 - G(\Ortsr[v],\Ortsr[vs])\frac{\partial\phi}{\partial n'} \right] \upd^2r'
       \end{align*}
     \item Somit:
       $$
      \boxed{ \phi(\Ortsr[v]) = \int_V
   \laddichte{V}(\Ortsr[vs]) G(\Ortsr[v],\Ortsr[vs]) \upd^3\Ortsr[s] + \varepsilon_0 \oint_{O(V)} \left[ G(\Ortsr[v],\Ortsr[vs]) \frac{\partial\phi}{\partial n'} - \phi\frac{\partial G(\Ortsr[v],\Ortsr[vs])}{\partial n'}\right] \upd^2r'}
 $$
 \item Freie Wahl von $\Gamma$! $\to$ Anpassung an gegebene Randbedingungen!
     \end{itemize}
\end{frame}   

\begin{frame}
  \frametitle{Dirichlet-Randbedingungen; $\phi$ auf $O(V)$ vorgegeben}
       $$
      \boxed{ \phi(\Ortsr[v]) = \int_V
   \laddichte{V}(\Ortsr[vs]) G(\Ortsr[v],\Ortsr[vs]) \upd^3\Ortsr[s] + \varepsilon_0 \oint_{O(V)} \left[ G(\Ortsr[v],\Ortsr[vs]) \frac{\partial\phi}{\partial n'} - \phi\frac{\partial G(\Ortsr[v],\Ortsr[vs])}{\partial n'}\right] \upd^2r'}
 $$
 \begin{itemize}
 \item<2-> Wunsch: Wähle $\Gamma(\Ortsr[v],\Ortsr[vs])$ so, dass
   $$
   \oint_{O(V)} G(\Ortsr[v],\Ortsr[vs]) \frac{\partial\phi}{\partial n'} \upd^2r' = 0
   $$
 \item<3-> Naheliegender (und häufig verwendeter) Ansatz:
   $$
   G(\Ortsr[v],\Ortsr[vs]) = 0 \text{ für } \Ortsr[vs] \in O(V) 
   $$
 \item<4-> Damit:
   $$
   \boxed{ \phi(\Ortsr[v]) = \int_V \laddichte{V}(\Ortsr[vs]) G(\Ortsr[v],\Ortsr[vs]) \upd^3\Ortsr[s] - \varepsilon_0 \oint_{O(V)}  \phi\frac{\partial G(\Ortsr[v],\Ortsr[vs])}{\partial n'} \upd^2r'}
   $$
\end{itemize}
\end{frame}


\begin{frame}
  \frametitle{Neumann-Randbedingungen; $\frac{\partial \phi}{\partial n}=-\EFeld[v]\cdot\vec{n}$ auf $O(V)$}
       $$
      \boxed{ \phi(\Ortsr[v]) = \int_V
   \laddichte{V}(\Ortsr[vs]) G(\Ortsr[v],\Ortsr[vs]) \upd^3\Ortsr[s] + \varepsilon_0 \oint_{O(V)} \left[ G(\Ortsr[v],\Ortsr[vs]) \frac{\partial\phi}{\partial n'} - \phi\frac{\partial G(\Ortsr[v],\Ortsr[vs])}{\partial n'}\right] \upd^2r'}
 $$
 \begin{itemize}
 \item<2-> Wunsch: Wähle $\Gamma(\Ortsr[v],\Ortsr[vs])$ so, dass
   $$
   \oint_{O(V)} \phi\frac{\partial G(\Ortsr[v],\Ortsr[vs])}{\partial n'} \upd^2r' = 0
   $$
 \item<3-> Naheliegender (und NICHT funktionierender) Ansatz:
   $$
   \frac{\partial G(\Ortsr[v],\Ortsr[vs])}{\partial n'} = 0 \text{ für } \Ortsr[vs] \in O(V) 
   $$
 \item<4-> Einerseits:
   $$
   \int_V \laplace_{r'}G(\Ortsr[v],\Ortsr[vs]) \upd^3\Ortsr[s] = -\frac{1}{\varepsilon_0} \int_V \delta^3(\Ortsr[v]-\Ortsr[vs]) \upd^3\Ortsr[s] = -\frac{1}{\varepsilon_0}  
   $$
 \item<5-> Andererseits:
  $$
   \int_V \laplace_{r'}G(\Ortsr[v],\Ortsr[vs]) \upd^3\Ortsr[s] = \oint_{O(V)} \gradient_{r'}G(\Ortsr[v],\Ortsr[vs]) \cdot\vec{n}' \upd^2\Ortsr[s] = \oint_{O(V)} \frac{\partial G(\Ortsr[v],\Ortsr[vs])}{\partial n'} \upd^2\Ortsr[s]  
   $$
 \ 
\end{itemize}
\end{frame}

\begin{frame}
  \frametitle{Neumann-Randbedingungen; $\frac{\partial \phi}{\partial n}=-\EFeld[v]\cdot\vec{n}$ auf $O(V)$}
       $$
      \boxed{ \phi(\Ortsr[v]) = \int_V
   \laddichte{V}(\Ortsr[vs]) G(\Ortsr[v],\Ortsr[vs]) \upd^3\Ortsr[s] + \varepsilon_0 \oint_{O(V)} \left[ G(\Ortsr[v],\Ortsr[vs]) \frac{\partial\phi}{\partial n'} - \phi\frac{\partial G(\Ortsr[v],\Ortsr[vs])}{\partial n'}\right] \upd^2r'}
 $$
 \begin{itemize}
 \item<2-> Nächstbester Ansatz: Wähle $\Gamma(\Ortsr[v],\Ortsr[vs])$ so, dass
   $$
   -\varepsilon_0 \oint_{O(V)} \phi\frac{\partial G(\Ortsr[v],\Ortsr[vs])}{\partial n'} \upd^2r' = \phi_0 \text{ mit } \phi_0 = \text{const.}
   $$
 \item<3-> Häufig gewählt:
   $$
   \frac{\partial G(\Ortsr[v],\Ortsr[vs])}{\partial n'} = -\frac{1}{\varepsilon_0 S} \text{ für } \Ortsr[vs] \in O(V), \quad S = \text{ Oberfläche} 
   $$
 \item<4-> damit:
   $$
   \phi_0 = \frac{1}{S}\oint_{O(V)} \phi(\Ortsr[vs]) \upd^2r'  
   $$
 \item<5-> Das ergibt dann:
  $$
      \boxed{ \phi(\Ortsr[v]) -\phi_0 = \int_V
   \laddichte{V}(\Ortsr[vs]) G(\Ortsr[v],\Ortsr[vs]) \upd^3\Ortsr[s] + \varepsilon_0 \oint_{O(V)} G(\Ortsr[v],\Ortsr[vs]) \frac{\partial\phi}{\partial n'} \upd^2r'}
 $$
\end{itemize}
\end{frame}

\begin{frame}
  \frametitle{Konkrete Lösungsverfahren}
  \begin{itemize}[<+->]
\item \alert{Wenn} die Funktion $\Gamma$ (und damit $G$) gefunden werden kann, dann ist das Poisson-Problem gelöst.
\item Konkrete Methoden zur Lösung werden im Folgenden betrachtet
\item \alert{Spiegelungsmethode}, \alert{Bildladungsmethode}: Platziere zusätzliche Ladungen \alert{außerhalb} von $V$. Deren Feld ist das gesuchte $\Gamma$, wenn die Überlagerung die Randbedingungen erfüllt
\item \alert{Entwicklung nach orthogonalen Funktionen:} Für geeignete Geometrien können die Randbedingungen durch eine geeignete funktionale Basis erfüllt werden. Lösungen sind dann Überlagerungen dieser Basis.
  \item \alert{Separation der Variablen}: Aus der Problemgeometrie abgeleiteter Ansatz für die Lösung, der in geeigneten Fällen zu einer direkt berechenbaren Lösung führt.
\end{itemize}
\end{frame}


\input{finalframe.inc}
   
\end{document}